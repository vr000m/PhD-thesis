%% Select the dissertation mode on
% See the documentation for more information about the available class options
% If you give option 'draft' or 'draft*', the draft mode is set on
%\documentclass[dissertation,final,nocontribution]{aaltoseries}

\documentclass[dissertation,final]{aaltoseries} %draft*->final
%\documentclass[dissertation]{aaltoseries}
\usepackage[utf8]{inputenc}

% Set the document languages
\usepackage[english]{babel}        % British hyphenation patterns

%\usepackage[ELEC, RGB]{aaltologo}

\usepackage[babel=true]{microtype} % http://ctan.org/tex-archive/macros/latex/contrib/microtype

\usepackage{graphicx}
 \graphicspath{{graphs/}}
 \DeclareGraphicsExtensions{.pdf}
 
\usepackage{amsmath}% http://ctan.org/pkg/amsmath
%%,amsfonts,amsthm,amsxtra,amsbsy,amssymb,mathrsfs}   % Mathematics
%\usepackage{bm} % bold math argument, http://www.ctan.org/pkg/bm

\usepackage{url}                   
                % http://ctan.org/tex-archive/macros/latex/contrib/url
\usepackage[caption=false,font=footnotesize]{subfig}                
                % http://ctan.org/tex-archive/macros/latex/contrib/subfig
\usepackage[labelfont=bf]{caption}  % , justification=centering
\usepackage{acronym}               
                % http://ctan.org/tex-archive/macros/latex/contrib/acronym
\usepackage{booktabs}              
                % http://ctan.org/tex-archive/macros/latex/contrib/booktabs
\usepackage{bytefield}             
                % http://ctan.org/tex-archive/macros/latex/contrib/bytefield
\usepackage{listings}              
                % http://ctan.org/tex-archive/macros/latex/contrib/listings
\usepackage{float}                 % Needed for hyperref - see note above
\usepackage{hyperref}              % Problematic - see note above
\usepackage{algorithm}             
                % http://ctan.org/tex-archive/macros/latex/contrib/algorithms
\usepackage{algorithmic}           % (also part of the algorithms bundle)

\usepackage{longtable}             % for tables that span across pages

\usepackage[footnote,draft,silent,nomargin]{fixme} %added FIXME!

\usepackage{multirow}   % for multirows
\usepackage{rotating}

\usepackage{setspace} 

%\floatstyle{plain}
%\newfloat{program}{thp}{figures}[chapter]
%\floatname{program}{Program}

\usepackage{chngcntr}
\counterwithout{footnote}{chapter}


\hyphenation{tele-presence}
\hyphenation{multi-media}

\hypersetup{
  pdfauthor={Varun Singh},
  pdftitle={Protocols and Algorithms for Adaptive Multimedia Systems},
  pdfsubject={RTP Congestion Control},
  pdfkeywords={RTP} {Congestion Control} {MPRTP} {geo-location} 
              {Multipath RTP} {Interactive Multimedia}
}

%%%%%%%%%%%%%%%%%%%%%%%%%%%%%%%%%%%%%%%%%%%%%%%%%%%%%%%%%%%%%%%%%%%%%%%%%%%%%%

% The author of the dissertation
\author{Varun Singh}
% The title of the thesis: Algorithms and Protocol extensions for Enabling Adaptive Multimedia Applications
\title{Protocols and Algorithms for Adaptive Multimedia Applications}

%%%%%%%%%%%%%%%%%%%%%%%%%%%%%%%%%%%%%%%%%%%%%%%%%%%%%%%%%%%%%%%%%%%%%%%%%%%%%%

\begin{document}

%% The abstract of the dissertation in English
% Use this command!
\draftabstract{

The deployment of WebRTC and telepresence systems is going to start wide-scale
adoption of high quality real-time communication. Delivering high quality
video usually corresponds to an increase in required network capacity and also
requires an assurance of network stability. A real-time multimedia application
that uses the Real-time Transport Protocol (RTP) over UDP needs to implement
congestion control since UDP does not implement any such mechanism. This
thesis is about enabling congestion control for real-time communication and
deploying it on the public Internet containing a mixture of wired and wireless
links.

A congestion control algorithm relies on congestion cues, such as RTT and
loss. Hence, in this thesis, we first propose a framework for classifying
congestion cues. We classify the congestion cues as a combination of: where
they are measured or observed? And, how is the sending endpoint notified?  For
each there are two options, i.e., the cues are either observed and reported by
an in-path or by an off-path source, and, the cue is either reported in-band
or out-of-band, which results in four combinations. Hence, the framework
provides options to look at congestion cues beyond those reported by the
receiver.

We propose a sender-driven, a receiver-driven and a hybrid congestion control
algorithm. The hybrid algorithm relies on both the sender and receiver co-
operating to perform congestion control. Lastly, we compare the performance of
these different algorithms. We also explore the idea of using capacity
notifications from middleboxes (e.g., 3G/LTE base stations) along the path as
cues for a congestion control algorithm. Further, we look at the interaction
between error-resilience mechanisms and show that FEC can be used in a
congestion control algorithm for probing for additional capacity.

We propose Multipath RTP (MPRTP) an extension to RTP, which uses multiple
paths for either aggregating capacity or for increasing error-resilience.  We
show that our proposed scheduling algorithm works in diverse scenarios (e.g.,
3G and WLAN, 3G and 3G, etc.) with paths with varying latencies.

Lastly, we propose a network coverage map service (NCMS), which aggregates
throughput measurements from mobile users consuming multimedia services. The
NCMS sends notifications to its subscribers about the upcoming network
conditions, which take these notifications into account when performing
congestion control.

In order to test and refine the ideas presented in this thesis, we have
implemented most of them in proof-of-concept prototypes, and conducted
experiments and simulations to validate our assumptions and gain new insights.


}
% Let's add another one in Finnish
%\draftabstract[finnish]{\lipsum[4-6]}
% And yet another one in Swedish
%\draftabstract[swedish]{\lipsum[7-9]}

%%%%%%%%%%%%%%%%%%%%%%%%%%%%%%%%%%%%%%%%%%%%%%%%%%%%%%%%%%%%%%%%%%%%%%%%%%%%%%

%% Preface
% If you write this somewhere else than in Helsinki, use the optional location.
\begin{preface}[Espoo]

This dissertation is based on several years of research at Aalto University
(erstwhile Helsinki University of Technology) and standardization at the
Internet Engineering Task Force (IETF). My sincere thanks to all the people
who helped me with my research. Foremost, I would like to thank Prof. Jörg Ott
for encouraging me to pursue the interaction of multimedia and congestion
control in more detail. He is an outstanding guide, his timely interactions
kept spirits high when faced with disappointments and for his patience to stay
the course. An equal measure of gratitude is due to all my co-authors. 

% First, Igor Curcio of Nokia Research Center for introducing me to the
% challenges faced by multimedia on mobile devies. 
% Saba Ahsan for helping with the design of the Multipath RTP scheduling
% algorithm. 

\end{preface}

%%%%%%%%%%%%%%%%%%%%%%%%%%%%%%%%%%%%%%%%%%%%%%%%%%%%%%%%%%%%%%%%%%%%%%%%%%%%%%

%% Table of contents of the dissertation
\tableofcontents

% For article dissertations, remove if you write a monograph dissertation.
\publicationdataorder{author}{title}{publication}{information}{month}{year and page}
\publicationpunctuation[]{,}{}{,}{,}{,}{}
\publicationformatting[]{}{}{\em}{}{}{}
\publicationadditionalpreformatting{}{``}{}{}{}{}
\publicationadditionalpostformatting{}{,''}{}{}{}{}

\listofpublications

%\listoffixmes

%% Add lists of figures and tables as you usually.
 \listoffigures
 \listoftables
% \listofalgorithms
% \listofequations

%% Add list of abbreviations, list of symbols, etc., using your preferred
%% package/method.

%\include{listofabbreviations}
%\include{listofsymbols}

%%%%%%%%%%%%%%%%%%%%%%%%%%%%%%%%%%%%%%%%%%%%%%%%%%%%%%%%%%%%%%%%%%%%%%%%%%%%%%

\chapter*{List of Abbreviations}
\begin{longtable}{ll}
3G		& Third Genereation \\
ACM		& Association for Computing Machinery \\
AVPF	& Audio-Visual Profile with Feedback \\
CB		& Circuit Breaker \\
DiffServ	& Differentiated Services \\
DSCP 	& Differentiated Services Code Points \\
ECN		& Explicit Congestion Control \\
FEC		& Forward Error Correction \\
HTML	& HyperText Markup Language \\
IETF	& Internet Engineering Task Force \\
IP		& Internet Protocol \\
LTE 	& Long Term Evolution \\
MPRTP 	& Multipath RTP \\
MPRTCP 	& Multipath RTCP \\
NAT		& Network Address Translation \\
QoS		& Quality of Service \\
QoE		& Quality of Experience \\
RFC 	& Request for Comments \\
RTP 	& Real-time Transport Protocol \\
RTCP 	& RTP Control Protocol \\
RTMFP	& Real-Time Media Flow Protocol \\
TCP		& Transmission Control Protocol \\
UDP		& User Datagram Protocol \\
WebRTC 	& Web Real-Time Communications \\
XR		& eXtended Reports \\
\end{longtable}


%% The main matter, one can obviously use \input or \include
\chapter{Introduction}
\label{chap1}


\section{Research Methodology}

\section{Structure of the Thesis}

This thesis describes techniques to adapt media to changing network
characteristics for different types of multimedia systems. The work is mainly
a summary of scientific-papers, but is also supported by additional body of
work. We have co-authored a number of Internet Drafts\footnote{at the time of
writing, several of these are still in the ID state, but will be published as
RFCs (Request for Comments) shortly} that complement the  sceintific results
discussed in the thesis. The chapters describing the various parts of the
congestion control framework discuss both our scientific and engineering work,
while associating it with the pertinent related work in the area. The
remainder of the thesis is organized as follows.

Chapter~\ref{chap2} discribes the research goal of the thesis, which consists
of creating a framework for congestion control that meets the  requirements
for multimedia systems.

Chapter~\ref{chap3} provides the neccessary background information to RTP
(Real-time Transport Protocol).

Chapter~\ref{chap4}

Chapter~\ref{chap5}

Chapter~\ref{chap6}

Chapter~\ref{chap7}

Chapter~\ref{chap8} concludes the thesis and we analyse if the framework meets
the requirements.


%\lipsum[11-12]

%% Examples of article references, remove these from your manuscript!
% Uncomment them, if you want to see the results of these commands in this example document

 % Refer to the Journal paper 1 of this example document
%\citepub{j1} \& \cpub{j1} \& \cp{j1} \& \pageref{j1} \& \ref{j1}

% Refer to the Conference paper of this example document
\citepub[p.~2]{c1} \& \cpub[Sec.~ 1]{c1} \&  \cp[pp.~1--2]{c1} \& \pageref{c1} \& \ref{c1} 






% \chapter{Research Goal}
% \label{chap:rg}
% %Enabling adaptive multimedia applications/systems.

The research goal of this thesis is to discuss congestion control for real-
time media. To achieve a suitable end-user experience, a multimedia
system\footnote{the transmitting endpoint or a classifier in the network} can:
1) associate a DiffServ Code Point (DSCP)~\cite{rfc2474} to the media packets;
therby, enabling Quality of Service (QoS). Using DSCP poses some challenges,
which are discussed in Section~\ref{rg.ch.dscp}. 2) instruct the encoder to
modify the encoding rate to a certain target rate. To achieve media rate-
adaption the endpoint needs to monitor and respond to congestion cues.
Additionally, we  discuss requirements for media rate-adaptation.


\section{Challenges with DSCP Markings}
\label{rg.ch.dscp}

DiffServ assigns each data packet to a traffic class and the network manages
each traffic class differently, thereby some traffic classes receive
preferential treatment (e.g., lower delay, lower losses)~\cite{rfc2475}. The
routers overcome congestion between traffic classes by implementing
\emph{priority queuing}, \emph{fair queuing}, or \emph{weighted fair queuing
(WFQ)}~\cite{rfc5865}; for congestion within the same traffic class the
router discards packets using \emph{tail drop} or \emph{Random Early Detection
(RED)}~\cite{Floyd:RED}.

Consequently, DiffServ needs to be implemented on every router along the data
path and configured to have the same forwarding policy (i.e., belong to the
same DiffServ administrative domain). 


\section{Congestion Cues}



\chapter{RTP: Real-time Transport Protocol}
\label{chap:rtp}
% intro

Real-time Transport Protocol (RTP)~\cite{rfc3550} is suitable for multimedia
telephony (voice-over-IP, video conferencing, telepresence systems),
multimedia streaming (video-on-demand, live streaming), and multimedia
broadcast. RTP's design is based on the fundamental principles of \textit
{application-layer framing} and \textit{integrated layer
processing}~\cite{clark:alf}, i.e., it provides the following mechanisms:
source and payload type identification, packet playout time, stream
synchronization, packet loss and re-ordering, media stream monitoring.

Figure~\ref{fig:3:rtp.hdr} describes the RTP packet header format, the
\textit{`synchronisation source'} (SSRC) assists in determining the source
endpoint, typically useful when an endpoint sends multiple media streams that
need to be synchronized (e.g., Audio/Video lip-sync). The \textit{`RTP
timestamp'} assists in playing out the received packets at the appropriate
instance of time and recomposing the media frame from RTP packets. The
\textit{`RTP sequence number'} assists in identifying the lost packets and re-
ordering packets in case of out-of-order packet arrival. Lastly, RTP uses
\textit{`payload type'} (PT) to describe the encoding of the media data it is
carrying. Consequentely, each codec needs to specify its corresponding payload
format.

\begin{figure}[!htbp]
\centerline{\includegraphics[width=\columnwidth]{fig_hdr_rtp}}
\caption{RTP header}
\label{fig:3:rtp.hdr}
\end{figure}

RTP utilizes RTP Control Protocol (RTCP) to monitor the performance of the
media stream. Using RTCP reports, the endpoints report loss fraction, jitter,
highest sequence number received, and calculate RTT. The RTCP reports also
assist in synchronizing the media streams (audio and video) by relating the
RTP timestamps of the individual media streams to the wall clock time and
measuring RTT. Figure~\ref{fig:3:rtcp.hdr} describes the RTCP packet header
format for a unicast media stream.

\begin{figure}
\centerline{\includegraphics[width=\columnwidth]{fig_hdr_rtcp}}
\caption{RTCP header}
\label{fig:3:rtcp.hdr}
\end{figure}

\section{RTCP Reporting Interval}
% timing

A closed control loop is formed by sending RTP media packets and receiving
RTCP feedback packets. The RTCP reporting interval is determined by the number
of SSRCs in the session, and the chosen session bandwidth. Typically, in a
unicast session there are two SSRCs, one for each participant, but the number
of SSRCs can be higher if they send multiple media streams. The interval
between reports tends to be on the order of a few seconds, and is randomized
to avoid snchronization of reports from multiple endpoints. Formally, to
ensure that the RTCP reports are not sent too frequently, the endpoints limit
the feedback rate to $5\%$ of the session \textit{media rate}, half for each
participant. However, if the endpoint detects packet loss or onset of
congestion midway through a reporting interval, the base RTP
specification~\cite{rfc3550} (AVP profile) does not allow sending the RTCP
report early and the endpoint has to wait for the next scheduled RTCP report.
Hence, the slow control loop can cause instability and oscillation in the
media bit rate.

% avpf

To use rapid feedback, the endpoints need to implement the Extended RTP
Profile for RTCP-Based Feedback (AVPF profile)~\cite{rfc4585}. This profile
allows the endpoint to adjust the RTCP reporting interval to send the RTCP
feedback reports earlier than the next scheduled RTCP report, sometimes even
immidiately. As long as the average reporting duration remains unchanged.

Along with the immediate feedback, the AVPF profile also defines a suite of
error-resilience mechanisms, namely, Negative Acknowledgements (NACK), Picture
Loss Indication (PLI), Slice Loss Indication (SLI), Reference Picture
Selection Indication (RPSI). 

\section{Extended Reports (XRs)}

% Transport Metrics

% The RTCP Extended Reports (XR) [RFC3611] allow reporting of more
% complex and sophisticated reception quality metrics, but do not
% change the RTCP timing rules.  RTCP extended reports of potential
% interest for congestion control purposes are the extended packet
% loss, discard, and burst metrics [RFC3611],
% [I-D.ietf-xrblock-rtcp-xr-discard],
% [I-D.ietf-xrblock-rtcp-xr-discard-rle-metrics],
% [I-D.ietf-xrblock-rtcp-xr-burst-gap-discard],
% [I-D.ietf-xrblock-rtcp-xr-burst-gap-loss]; and the extended delay
% metrics [RFC6843], [RFC6798].


\section{Codec Control}
% codec control

% relationship with SDP
\section{RTP Extensions}
\label{rtp.ext}


\chapter{Congestion Control Framework for Real-time Communication}
\label{chap:cc.fw}

RTP transmits the media data over IP using a variety of transport layer
protocols such as UDP, TCP, and Datagram Congestion Control Protocol (DCCP).
Consequetly, congestion control for RTP media flows can be implemented either
in the application or the media flows are transmitted over congestion-
controlled transport (TCP or DCCP). While using a congestion controlled
transport may be safe for the network, it is suboptimal for the media quality
unless the congestion-controlled transport is designed to carry media flows.
On the other hand, using a non-congestion controlled transport (e.g., UDP),
the rate-adaptation is implemented in the application. In this thesis, we
consider congestion control for unicast RTP traffic running over best-effort
IP network. To implement congestion control the sending endpoint needs to rely
on RTCP feedback from the receiver, but unlike other delay-based variants of
TCP there is insufficient RTCP bandwidth to provide feedback on a per-packet
basis.

% CC should not cause queuing delay. Or define low-delay operation of
% multimedia cc.

\section{Framework}
\label{fw.fw}


\begin{figure}
\includegraphics[scale=1.0]{chap2-fw-outline}
\caption{Congestion Cue Framework}
\label{fig:4:fw}
\end{figure}


\section{Requirements for Congetsion Control}
\label{fw.cc.eval}

\chapter{Congestion Control for Interactive Multimedia}
\label{chap:cc}
Figure~\ref{fig:4:fw} in Chapter~\ref{chap:cc.fw} shows the structure of the
congestion control framework described in this thesis. The framework
categorizes \emph{in-path} sources and \emph{in-band} signaling for
implementing congestion control (corresponds to \emph{Block A} in
Figure~\ref{fig:4:fw}), which are discussed in this chapter. This chapter is
based on our work on congestion control for interactive multimedia
applications, which is documented in \citepub{c:3grc}, \citepub{c:hetrc},
\citepub{c:eval}, \cite{rfc7097},
\cite{draft.xr.bytes.discarded}, \cite{singh:2010.thesis} and
\cite{Singh:control.loops.api}.

In \citepub{c:3grc}, we propose a new congestion control algorithm for the
mobile (e.g., 3G) environment, to be deployed in the IP Multimedia System
(IMS). The main distinction between mobile (e.g., 3G, LTE) and other wireless
environments (e.g., 802.11x) is that the media streams are transmitted using
the \emph{unacknowledged mode}; the packets corrupted due to bit-errors (e.g.,
wireless interference) are not re-transmitted. Hence, the packets incur low
delay, compared to Wireless LAN where corrupted packets are retransmitted by
the link layer. We propose a sender-driven and a receiver-driven congestion
control, and evaluate the performance of the proposed congestion control
algorithm in a simulated environment (in ns-2) using real-world 3G
traces~\cite{s4.eval.bitrate, 3gppSim}. In \citepub{c:hetrc}, we extend the
approach in \citepub{c:3grc} for deployment on the Internet and show that the
congestion control scheme is deployable. In \cite{rfc7097} and
\cite{draft.xr.bytes.discarded}, we propose RTCP XR block extensions that
indicate the number of bytes discarded and run-length encoding of discarded
packets, respectively. These packets are discarded by the receiver because
they arrived too early or too late to be played out by the receiver. This
information is used as a congestion cue by the sender.

\cite{Singh:control.loops.api} discusses the application and API requirements
for interactive multimedia congestion control. It describes the two control
loops: a) between the receiver and the sender, and b) between the media
encoder and the sending agent. In the first loop, the receiver notifies the
sender about the current network characteristics. In the second loop, the
sending agent requests a new media bit rate, and the encoder tries at best to
meet it, sometimes under-shooting or over-shooting the requested rate.

Lastly, in \citepub{c:eval} we evaluate the performance of a congestion
control algorithm proposed by Google for WebRTC. We evaluate the performance
in diverse scenarios measuring scalability (\emph{how quickly is the
congestion control able to utilize the available capacity}), self-fairness and
competing against bursty cross-traffic. We evaluate the performance of
web-browsers implementing the congestion control algorithm in our testbed that
emulates the diverse scenarios.

\section{Schemes of Congestion Control}

The congestion control algorithm can be implemented at the sender, at the
receiver, or the sender and receiver operate co-operatively. The
\emph{sender-driven} scheme requires that the receiver measures the current
network condition and signals the observed congestion cues to the sender, which
calculates the sender's estimate and uses it as the new sending rate. In the
\emph{receiver-driven} scheme, the receiver calculates the new sending rate
(receiver's estimate) based on the observed congestion cues, and signals the
new rate to the sender, which on receiving the new rate, adapts the media bit
rate to received value. The \emph{co-operative} scheme is an extension to the
\emph{sender-driven} scheme, in this case, the receiver calculates the
receiver's estimated rate and signals it along with the observed congestion
cues, the sender at its end calculates the sender's estimate based on the
congestion cues and chooses a new sending rate, typically, between the
sender's estimate and the receiver's estimate. Figure~\ref{fig:cc:scheme}
shows the interaction of the sender and receiver for each scheme. The figure
merely shows the media flow in one direction, however, it should be noted that
the media in the simulation and the emulated testbed actually flow in both
directions unless explicitly mentioned. This is mainly done for the
convenience of representation and followed throughout the remainder of the
thesis.

\subsection{Sender-driven Congestion Control Schemes}

\begin{figure}[!t]
  \centerline{
    \subfloat[Sender-driven Scheme]{
      \includegraphics[width=0.9\textwidth]
      {chap5-fig-cc-scheme-s}
    }
  }
  \centerline{
    \subfloat[Receiver-driven Scheme]{
      \includegraphics[width=0.9\textwidth]
      {chap5-fig-cc-scheme-r}
    }
  }
  \centerline{
    \subfloat[Co-operative Scheme]{
      \includegraphics[width=0.9\textwidth]
      {chap5-fig-cc-scheme-c}
    }
  }
  \caption{Congestion control schemes a) sender-driven, b) receiver-driven
and c) co-operative.}
  \label{fig:cc:scheme}
\end{figure}

TCP Friendly Rate Control (TFRC) is an equation based congestion control
algorithm implemented at the sender~\cite{tfrc_347397} and is also implemented
as a profile~\cite{rfc4342} in the Datagram Congestion Control Protocol
(DCCP)~\cite{rfc4340}. TFRC uses the average packet size, round trip time
($RTT$), loss ratio ($p$)~\cite{rfc3448} to calculate the new sending rate.
Formally, the TFRC is calculated as follows:

\begin{align*}
 TFRC = &\; \frac{8 \times avg\_packet\_size}
{RTT \times \sqrt[]{\frac{2 \times b \times p}{3}} + t_{RTP} \times 
\left( 3 \times \sqrt[]{\frac{3 \times b \times p}{8}}\right) \times p \times
\left( 1+32 \times p^2 \right)}\\
where,\; b = &\; 1\\
t_{RTO} = &\; 4 \times RTT
\end{align*}

TFRC cannot be directly be applied to RTP because TFRC requires per-packet
feedback and in RTP, the RTCP feedback is not necessarily sent that
often~\cite{draft.rmcat.feedback}. Therefore, \cite{draft.rtp.tfrc} maps the
TFRC timing rules defined in~\cite{rfc4828, rfc5348} to that of RTP/RTCP
feedback loop, it also proposes extensions to the timing rules in
AVPF-profile~\cite{rfc4585} for very short RTTs ($<20ms$).
\cite{Gharai06:ICME} and \cite{VladBalan:2007dq} show that TFRC is stable on
paths with longer RTTs than those with smaller RTTs, but it too exhibits
saw-tooth behavior~\cite{saurin:2006:thesis}. Any algorithm that consistently
produces a sawtooth media rate is not well suited for real-time communication
because it generates a poor user-experience~\cite{Gharai:2002wt,
Zink03subjectiveimpression}.

Other sender-driven congestion control algorithms that we explored are: The
Rate Adaption Protocol (RAP)~\cite{rap:752152} uses a windowed-approach and
this too exhibits a sawtooth-type of behavior. Zhu~\textit{et
al.}~\cite{rmcat-nada} use Pre-Congestion Notification (PCN), Explicit
Congestion Notification (ECN) and loss rate to get an accurate estimate delay
estimate for implementing congestion control. In this case, they assume all
packets marked by ECN and PCN as lost. Instead of just relying on RTT and loss
for congestion control, Garudadri~\textit{et al.}~\cite{4397059} also use the
receiver playout buffer to detect underutilization and overuse, i.e., the
receiver signals to the sender the current receiver buffer occupancy.
O'Hanlon~\textit{et al.}~\cite{rmcat-dflow} propose using a delay-based
estimate when competing with similar traffic and using a windowed-approach
when competing with TCP-type cross traffic, they switch modes by using a
threshold on the observed end-to-end delay, the idea is similar to the one
discussed in~\cite{budzisz2011fair}.




\subsection{Receiver-driven Congestion Control Schemes}

In receiver-driven congestion control, the receiver estimates the rate
notifies the sender about the new sending rate. Temporary Maximum Media Bit-
rate Request (TMMBR) is defined as a codec control messages in \cite{rfc5104}.
It is generated by the receiver in a point-to-point video call. The receiver
calculates the new estimate (available capacity) based on the average inter-
arrival time of RTP packets (\emph{video frames}). When the inter-arrival time
of the video frames increases beyond the expected arrival time in an observed
period, the receiver senses \emph{over-use}. When the frames arrive early, the
receiver senses \emph{under-use}. If the over-use and under-use occur at short
timescales, mainly because of I-frames, the receiver ignores the congestion
event. The I-frames are large frames because they are spatially compressed and
are not temporally correlated to previous frames. Hence, these I-frames are
expected to observe queuing delay. The receiver, on detecting link \emph{over
utilization} or \emph{under utilization}, modifies the \emph{receiver's
capacity estimate}. The receiver sends the TMMBR message to the sender
indicating the maximum sending rate. Currently, interactive multimedia sessions
in 3GPP~\cite{3gpp.26.114} use TMMBR messages to notify the sender of the
expected sending rate. In WebRTC~\cite{jennings:2013:webrtc}, TMMBR is
expected to be used initially, before RTP congestion control is standardized
by the IETF~\cite{rtp-usage}. The expectation is that different WebRTC clients
may develop proprietary receiver-driven algorithms and use TMMBR as the
standardized mechanism to communicate the capacity estimate to the sender,
which will blindly follow it.


\subsection{Co-operative Congestion Control Schemes}
\label{cc:co-op}

Next Application Data Unit (NADU)~\cite{nadu.1070341,nadu.1530486} is designed
for rate adaptation for video streaming in 3GPP~\cite{3gpp.26.234}. A NADU
receiver measures the playout delay (as a measure of buffer occupancy in time)
and signals it to the sender along with the next sequence number to be played
out. Conversational NADU (C-NADU) is an extension to NADU for congestion
control for interactive multimedia and is described in \citepub{c:3grc} and
\citepub{c:hetrc}. In C-NADU, the receiver also calculates the
\emph{receiver's capacity estimate} by measuring the frame inter-arrival time
and signals that along with the NADU report. If the video frame arrives at the
expected time, the receiver assumes no ongoing congestion, and if it arrives
later than the expected time, the frame is considered late and the receiver
diagnoses congestion. If the frame is delayed and misses its playout time, it
is discarded and in this case the receiver estimates congestion. Based on the
above cases, the receiver estimates the current capacity and signals it to the
sender. At the sender, the C-NADU controller calculates the TCP-friendly rate,
measures the variation in RTT ($75$ and $90$ percentile values) and calculates
the fraction of video frames that missed their playout deadline. Based on
these congestion cues and the receiver estimate, the sender chooses a new
sending rate.


Receiver-side Real-Time Congestion Control (RRTCC) is described in
\cite{draft.rrtcc} and is proposed as one of the solution candidates for
WebRTC by Google. Like C-NADU, RRTCC also has a receiver- and sender-side
component. The receiver-side measures the capacity overuse and underuse by
monitoring the timestamp jitter of the incoming frames. The arrival-times are
modeled as a white Gaussian process; when the mean is 0 there is no
congestion, the mean is expected to increase when there is ongoing congestion
and expected to decrease when the congestion abates. Based on this
expectation, the receiver calculates the capacity estimate and signals it to
the sender. The sender calculates its estimate based on TFRC and, finally
chooses the new sending rate as a value between the TFRC rate calculated by
the sender and receiver estimate. Full details of the algorithm proposed by
Google are documented in an Internet-Draft~\cite{draft.rrtcc}.



\begin{figure}[!t]
  \centerline{
    \subfloat[TFRC]{
      \includegraphics[width=0.5\textwidth, clip=true, trim=3cm 0 4.5cm 0]
      {chap5_graph_sl_tfrc}
    }
    \subfloat[TFRC]{
      \includegraphics[width=0.5\textwidth, clip=true, trim=3cm 0 4.5cm 0]
      {chap5_graph_3g_tfrc_1}
    }
  }
  \centerline{
    \subfloat[TMMBR]{
      \includegraphics[width=0.5\textwidth, clip=true, trim=3cm 0 4.5cm 0]
      {chap5_graph_sl_tmmbr}
    }
    \subfloat[TMMBR]{
      \includegraphics[width=0.5\textwidth, clip=true, trim=3cm 0 4.5cm 0]
      {chap5_graph_3g_tmmbr_u}
    }
  }
  \centerline{
    \subfloat[C-NADU]{
      \includegraphics[width=0.5\textwidth, clip=true, trim=3cm 0 4.5cm 0]
      {chap5_graph_sl_cnadu}
    }
    \subfloat[C-NADU]{
      \includegraphics[width=0.5\textwidth, clip=true, trim=3cm 0 4.5cm 0]
      {chap5_graph_3g_cnadu}
    }
  }
  \caption{The plots show the performance of TFRC, TMMBR and C-NADU in a slow
  time-varying link (a, c, e) and 3G network (b, d, f).}
  \label{fig:3grc}
\end{figure}


\begin{table}[!t]
\centering{
\begin{tabular}{cccc}
\hline
 & Avg. goodput & PLR & Avg. PSNR\\
 & (kbps)  &(\%) & (dB)\\
\hline
TFRC & 84.1 & 6.9\%& 29.3 \\ %
TMMBR & 89.8 & 3.7\% & 30.5 \\ %
%NADU & 106 & 93 & 29.9& 6.3\%\\%
C-NADU & 92 & 2.2\% & 31.9 \\%
\hline
\end{tabular}
}
\caption{Comparing TFRC, TMMBR, C-NADU for calls over mobile nodes (180s
simulations using 3G traces).}
\label{table:3grc}
\end{table}

\section{Performance Analysis of TFRC, TMMBR, C-NADU, and RRTCC}

This section briefly discusses the performance of each congestion control
algorithm, the detailed analysis can be found in the respective papers.

Our results in \citepub{c:3grc} shows that TFRC produces a sawtooth sending
rate, similar to the performance in~\cite{saurin:2006:thesis}. When the media
stream is the only flow on the end-to-end path, we also observe the average
bandwidth utilization (ABU) is between 30-40\,\%, i.e., TFRC underutilizes the
link and the loss ratio is about 6\,\% which results in a lower media quality
(approximated by measuring PSNR) compared to the other two schemes (see
Table~\ref{table:3grc}). TMMBR-based congestion control utilizes the link
better than TFRC (ABU between 50-70\,\%) and produces a lower loss ratio
($\approx$3\,\%). Lastly, C-NADU has comparable bandwidth utilization
(ABU=55-60\,\%) and loss ratio ($\approx$2\,\%) to TMMBR.
Figure~\ref{fig:3grc} shows the performance of TFRC, TMMBR and C-NADU over two
types of bottleneck links, a slow time-varying link and a 3G link.


\begin{figure}[!t]
\centerline{
%\hfill
\subfloat [Call 1 vs Call 2] {\label{fig_sim-mixed-1-1}\includegraphics[width=0.5\textwidth]{chap5-graph-5rtp_uc1_12}%
}
%\hfill
\subfloat [Call 1 vs Call 3] {\label{fig_sim-mixed-1-2}\includegraphics[width=0.5\textwidth]{chap5-graph-5rtp_uc1_13}%
}
%\hfill
}
\centerline{
%\hfill
\subfloat [Call 1 vs Call 4] {\label{fig_sim-mixed-1-3}\includegraphics[width=0.5\textwidth]{chap5-graph-5rtp_uc1_14}%
}
%\hfill
\subfloat [Call 1 vs Call 5] {\label{fig_sim-mixed-1-4}\includegraphics[width=0.5\textwidth]{chap5-graph-5rtp_uc1_15}%
}
%\hfill
}
\caption{The plots show the performance of five C-NADU calls competing for
capacity on a shared bottleneck in a heterogeneous network. Each calls needs
to quickly adapt to changes in 3G link capacity and fairly share the
bottleneck link.}
\label{fig:hetrc}
\end{figure}

\begin{table}[!t]
\centering{
\scalebox{0.9}{
\begin{tabular}{cccccc}
\hline
 & 3G Capacity & $Goodput_{avg}$ & PLR & $PSNR_{avg}$ & ABU \\
 & Pattern & (kbps) & (\%) & (dB) & (\%) \\ 
\hline
Call 1 & Excellent-Poor-Elevator & 140.10 & 2.15\% & 31.4 ($\sigma=0.39$) & 70.1\% \\ 
Call 2 & Good-Good-Poor & 133.55 & 1.61\% & 31.9 ($\sigma=0.62$)& 66.8\% \\ 
Call 3 & Poor-Poor-Poor & 35.18 & 1.55\% & 22.2 ($\sigma=1.13$)& 17.59\% \\ 
Call 4 & Fair-Fair-Poor & 114.96 & 2.75\% & 31.1 ($\sigma=0.75$)& 57.5\% \\ 
Call 5 & Excellent-Elevator-Poor & 130.23 & 2.25\% & 31.3 ($\sigma=0.13$)& 65.1\% \\ 
\hline
\end{tabular}
}}
\caption{C-NADU: Five calls in a heterogeneous network with end-to-end latency
between \emph{60-120ms} and 0.5\% link-layer losses.}
\label{table:hetrc}
\end{table}

In \citepub{c:hetrc}, we show that C-NADU is self-fair with other C-NADU flows
in both wired and wireless environments~\cite{singh:2010.thesis} and in
\citepub{c:fecrc} we show that it competes fairly with TCP cross-traffic, both
long and short (bursty) TCP flows. Figure~\ref{fig:hetrc} show 5 video calls,
each sender uses an independent 3G link into a common bottleneck to the
receivers. The 3G links are based on radio link traces and have different
capacities. Hence, at some instances of time the 3G link is the constraint and
at other times the shared bottleneck link. Table~\ref{table:hetrc} shows that
4 calls have comparable performance (see PSNR and goodput) and 1 call suffers
due to poor connectivity (the 3G link has insufficient capacity which affects
the quality).

\begin{figure}[!t]
  \centerline{
    \subfloat{
      \includegraphics[width=0.8\textwidth]
      {chap5-graph-rrtcc-latency}
    }
   }
   \centerline{
    \subfloat{
      \includegraphics[width=0.8\textwidth]
      {chap5-graph-rrtcc-plr}
    }
  }
  \caption{The plots show the performance of RRTCC on a link with varying
  delay and fractional loss rate. We observe that by the sending rate
  decreases when increasing link latency or bit-error loss. }
  \label{fig:rrtcc-single}
\end{figure}

\begin{table}[!t]
\begin{center}{
  \scalebox{0.9}{
\begin{tabular}{ ccccc }
\hline
 & Goodput & Residual  & PLR\\
 & (kbps)  & Loss (\%) & (\%)\\
\hline
 0\% & 1949.7$\pm233.62$ & 0.011 & 0.011 \\ 
 5\% & 1568.74$\pm178.52$ & 0.23 & 9.77 \\ 
 10\% & 1140.82$\pm161.92$ & 0.49 & 19.02 \\ 
 20\% & 314.4$\pm61.98$ & 2.43 & 36.01 \\ \hline
\end{tabular}
}}
\end{center}
\caption{RRTCC: Metrics for a bottleneck with different packet loss rates.}
\label{tab:rrtcc-loss}
\end{table}

In \citepub{c:eval}, we evaluate the performance of RRTCC in several
scenarios: by itself on a bottleneck link, competing with other RRTCC flows
and competing with TCP cross-traffic. Figure~\ref{fig:rrtcc-single} shows an
example plot of the performance of RRTCC when increasing latency and
fractional loss. For example, in Figure~\ref{fig:rrtcc-single}(a) by
increasing the bottleneck link latency reduces the sending rate of RRTCC.
Similarly, Figure~\ref{fig:rrtcc-single}(b) shows that increasing the loss
rate also affects the sending rate. However, Table~\ref{tab:rrtcc-loss} shows
that even though the link has high loss rate, the residual loss rate is low
(even when the loss is 20\,\%), mainly due to the use of NACKs, PLI and FEC.


\begin{figure}[!t]
\centerline{
  \subfloat[Start together]
    {\includegraphics[width=0.8\textwidth]
    {chap5-graph-rrtcc-three-calls-sync}}
  }
  \centerline{
  \subfloat[Start 30s apart]
    {\includegraphics[width=0.8\textwidth]
    {chap5-graph-rrtcc-three-calls-async}}
  }
   \caption{The plots show the variation in receiver rate of three RRTCC
   flows, a) starting together, b) starting 30s apart. The total duration of
   the call is 5 mins (300s).}
\label{fig:rrtcc-self-fair}
\end{figure}

\begin{table}[!t]
\begin{center}
  \scalebox{0.9}{
\begin{tabular}{ccccc}
\hline
& Rate  & RTT & Residual & PLR\\
& (kbps)& (ms) & Loss (\%) & (\%)\\ \hline
 3 calls &  809.07$\pm202.38$ &   31.48$\pm24.93$ & 0.21 & 0.23 \\  
 3 calls (time shifted) &  1154.32$\pm250.54$ &   35.15$\pm27.88$ & 0.08 & 0.91 \\ \hline
\end{tabular}
}
\end{center}
    \caption{RRTCC competing with similar cross-traffic on the bottleneck link.}
    \label{tab:self-fair}
\end{table}

Next, we emulate three calls sharing a common bottleneck, in this case the
individual media rates do not reach their individual maximum rate of 2Mbps.
Figure~\ref{fig:rrtcc-self-fair}(a) shows three calls ramp-up at about the
same rate, reach a peak and drop their rate simultaneously. The sending rates
synchronize, even though the flows originate from different endpoints using
independent WebRTC stacks.

Lastly, instead of the three calls starting together, each call starts at 30s
intervals. We observe that while the media rate per call on average is higher,
the first call has a disadvantage and in all the cases, temporarily starves
when a new flow appears and after a few minutes starts to ramp-up again.
Figure~\ref{fig:rrtcc-self-fair}(b) shows the instantaneous rates of each of
the calls. The first call temporarily starves when new flows appear because
when it starts it is the only flow on the bottleneck and does not encounter
any queues, it observes a certain RTT and uses that as the baseline. When the
second flow appears, the second flow already observes queues from the existing
stream and competes with it, while the initial flow observes an increase in
queues and reduces the sending rate to avoid congestion.

\section{Summary}

In this chapter we describe congestion control implemented using congestion
cues from in-band sources and signaled in path (in RTCP). We further
categorize the congestion control algorithms based on \emph{where they are
implemented}: sender-driven scheme (e.g., TFRC), receiver-driven scheme (e.g.,
TMMBR), or co-operative scheme (combination of sender and receiver, e.g.,
C-NADU, RRTCC) and compare the performance of an algorithm in each scheme. We
observe that the performance of TFRC is bursty, which may lead to poor  user-
experience, whilst TMMBR, C-NADU and RRTCC have a more stable throughput.
Lastly, TMMBR appears to be conservative with very low packet loss, while
RRTCC appears to be aggressive with a lot more packet loss. C-NADU appears to
be in between the two schemes with higher throughput than TMMBR and much lower
packet loss compared to RRTCC.

%% An example for changing the running header (the optional parameter)

 \chapter{Interaction of Error-Resilience and Congestion Control}
 \label{chap:er-cc}
Error-resilience is typically a topic discussed orthogonally to congestion
control and the main reason is that, error-resilience caters to handling
packet loss while congestion control caters to the amount of information sent
over the network. This chapter is based on our work on unifying
error-resilience and congestion control.

In \citepub{c:err}, we evaluate the performance of the various
error-resilience schemes available for use in interactive multimedia
communication (mainly applicable to H.264). These are: using Negative
Acknowledgement (NACK) or Packet Loss Indication (PLI), Forward Error
Correction (FEC) or Unequal Level of Protection (ULP), slice size adaption
(SSA), and Reference Pictures Selection Indication (RPSI). We evaluate the
performance of the proposed mechanisms in diverse scenarios in a simulated
environment (in ns-2) using real-world 3G loss patterns~\cite{3gppSim}.
Lastly, based on our observations, we define the applicability of the various
error-resilience with respect to end-to-end delay and packet loss.

In \citepub{c:fecrc}, we propose using FEC not only for error-resilience but
also for congestion control. Instead of probing for available capacity by
increasing the sending rate of the media flow, we propose introducing
redundancy. If a packet gets lost and the added FEC packet arrives in time the
receiving endpoint would recover the lost packet. However, if the packet is
not lost, by introducing the FEC packet the sender not only discovers that
there is additional available capacity, but also has a sense of the magnitude
(at minimum) of the available capacity. We compare our proposal with our
previous work in \citepub{c:3grc} and \citepub{c:hetrc}, and Google's
congestion control~\cite{draft.rrtcc}. We evaluate the performance of the
mechanisms in diverse scenarios implemented in a simulation environment (in
ns-2) and in our testbed.

\section{Error-resilience Schemes}
% explain all 4 and the adaptivity

\begin{figure}
\centerline {
\includegraphics[width=0.9\textwidth]{chap6_apply_err}
}
\caption{Applicability of the error-resilience schemes in heterogeneous
environment containing both wireless and wired links.}
\label{chap6:fig_err}
\end{figure}

\section{Using FEC for Congestion Control}

% Figure with the idea: FEC for CC

% Table of FEC, C-NADU and RRTCC :)

\chapter{Mobility, Offloading, Multihoming, and Overlays}
\label{chap:mprtp}
Figure~\ref{fig:4:fw} in Chapter~\ref{chap:cc.fw} shows the structure of the
congestion control framework described in this thesis. The framework
categorizes \emph{Out-of-path} sources and \emph{in-band} signaling for
implementing congestion control (corresponds to \emph{Block C} in
Figure~\ref{fig:4:fw}), which are discussed in this chapter. This chapter is
based on our work on Multipath RTP (MPRTP), which is documented in
\citepub{c:mprtp}, in \cite{draft.mprtp}, \cite{draft.mprtp.sdp},
\cite{Globisch:AsymGrpComm}, and \cite{draft.rtcp.overlay}.

In \citepub{c:mprtp}, we propose the following: design goals to implement a
multipath protocol for multimedia, protocol details, scheduling algorithm to
send media packets over multiple paths, a dejitter buffer implementation to
playout packets smoothly even when the path skew is high. We evaluate the
performance of the proposed mechanisms in diverse scenarios in our testbed.
Lastly, we discuss the system consideration for deployment.

In \cite{draft.mprtp}, we describe the requirements, functional blocks and
protocol formats to extend RTP for enabling multipath capabilities. However,
this document does not define a scheduling algorithm and allows for multiple
proposals. In \cite{draft.mprtp.sdp}, we describe SDP and RTSP extensions
required to setup MPRTP sessions and also ICE extensions to advertise MPRTP
interfaces and perform NAT traversal.

In \cite{Globisch:AsymGrpComm}, we propose using a network topology with
multiple distribution trees to distribute video streams in a very large call
video conference with few active speakers and many passive participants
listening in. The multiple distribution trees carry separate MPRTP subflows
and participants are members of multiple distribution trees, but actively
forward media flows in one of the distribution trees. Therefore, a node is an
overlay node in a few (e.g., one) distribution trees and a leaf node in the
rest. In the paper, we use a centralized focus to manage the joining, leaving
and inserting the participant in the appropriate position in the distribution
tree. \cite{draft.rtcp.overlay} is a work in progress and proposes protocol
extensions to perform tree constructions in a distribution tree without the
need of a centralized conferencing focus.


\section{Multipath RTP (MPRTP)}

The Internet backbone has evolved over the past decades to a mesh of service
providers with manifold peerings that are generally capable of offering a
number of (independent) paths between two nodes. Networks often use multiple
attachment points for resilience purposes, such as data enterprise networks or
data centers and even routers for SOHO networks support multiple access
networks~\cite{draft.fun.multi, draft.homenet.arch}. Additionally, many hosts
today feature multiple network interfaces (e.g., WLAN and 3G on mobile
devices), this may yield the possibility for two endpoints to communicate via
multiple paths. While exploiting multipath characteristics
\cite{Wischik:2008:RPP} has been explored for TCP (e.g.,
MPTCP~\cite{rfc6824}), the requirements for real-time traffic differs notably
and TCP can at best serve real-time communication within tight delay
constraints of the network~\cite{Brosh:tcp-real-time}. In the multipath case,
the scheduling algorithms do not consider real-time bounds when spreading data
segments across different paths and diverse paths may lead to worst case delay
and thus even longer buffering time.

\begin{figure}
\centerline {
\includegraphics[width=0.9\textwidth]{chap7-fig_mprtp-1}
}
\caption{System Overview: A sender uses multiple paths to stream media
  to a receiver.  The receiver uses a dejitter buffer to reorder
  packets and sends per-path characteristics to the sender that
  distributes the packets based on the reported values.}
\label{chap7:fig_mprtp}
\end{figure}

We propose Multipath RTP (MPRTP) as a backwards compatible extension to
RTP~\cite{rfc3550}, it is documented in \cite{draft.mprtp} and defines the
basic mechanisms to operate across multiple parallel paths.
Figure~\ref{chap7:fig_mprtp} shows a macroscopic system overview of MPRTP. The
primary use-case for MPRTP is transporting media flows between multi-homed
endpoints. Such endpoints could be residential IPTV or telepresence devices
that connect to the Internet through two different Internet service providers
(ISPs), or mobile devices that connect to the Internet through 3G and WLAN
interfaces. By allowing RTP to use multiple paths for transmission, the
following gains can be achieved:

\begin{enumerate}
\setlength{\itemsep}{5pt}

\item \textbf{\texttt{Higher quality}}: Pooling the resource capacity of
multiple Internet paths allows higher bit-rate and higher quality codecs to be
used. From the application perspective, the available bandwidth between the
two endpoints increases.

\item \textbf{\texttt{Load balancing}}: Transmitting an RTP stream over
multiple paths reduces the bandwidth usage on a single path, which in turn
reduces the impact of the media stream on other traffic on that path. Also by
seamlessly offloading a flow from one path to another allows for some gains,
for example, reduces energy consumption, reduces access costs, or reduces
network latency.

\item \textbf{\texttt{Fault tolerance}}: Using multiple paths in conjunction
with redundancy mechanisms (FEC, re-transmissions, etc.), outages on one path
have less impact on the overall perceived quality of the stream. This can also
enable seamless handover in the case of mobility, i.e., moving from one
network to another.

\end{enumerate}


\begin{figure}
\centerline {
\includegraphics[width=0.75\textwidth]{chap7-fig-mprtp-stack}
}
\caption{The RTP and MPRTP stack working alongside each other. SSRC $\#1$ uses
MPRTP while SSRC $\#2$ and SSRC $\#3$ uses single path RTP.}
\label{chap7:fig_mprtp_arch}
\end{figure}

Figure~\ref{chap7:fig_mprtp_arch} compares the network stack of a single path
and a multipath-capable endpoints. SSRC \#2 and SSRC \#3 use a single path,
while SSRC \#1 uses multiple paths (with two subflows for the two interfaces).
Subflow \#1 and \#2 are expected flow over IP address \#1 and \#2,
respectively. To discover its multiple interfaces, the multimedia application
either uses the ICE procedures (hence, STUN) or polls the kernel repeatedly.

The design goals for MPRTP from our perspective are: MPRTP-enabled system to
be able to make use of multiple paths and adapt to their relative capacity
changes by redistributing the load. As different paths will likely exhibit
different RTTs, mechanisms must be developed to overcome the resulting skew.
Furthermore, the choice of suitable transmission paths should reflect the
demands of the application. From a protocol perspective, RTP must be extended
to perform these functions, yet maintain backwards compatibility.


Specifically for multimedia, Liang \emph{et al.}~\cite{Liang01} show that
transmitting redundant voice traffic over multiple paths perform better than a
FEC protected single stream. While Chesterfield \emph{et al.}~\cite{1498479}
show that by sending media over one 3G interface and Unequal Protection (UEP)
packets over a separate 3G interface can compensate for losses on the first
path. Chebrolu \emph{et al.}~\cite{1599407} propose bandwidth aggregation for
multimedia applications by computing the earliest delivery time for each
packet. They further propose to drop less important frames (e.g., B-frames) if
the available capacity is smaller than the current encoding
rate~\cite{1313320}. Jurca \emph{et al.}~\cite{4130370:jurca} propose a
frame-aware scheduling algorithm that sends key-frames and other important
media packets over less lossy paths and this approach is similar to the one
proposed in this paper. However, they also propose sending future packets over
high latency paths by reading ahead in the media stream. While this is an
interesting concept, it would require larger buffers and more state at the
sender (typically, RTSP servers) to read ahead the stored media stream and
this would not work for interactive multimedia and live video streams where it
cannot read ahead. Our proposed scheduling algorithm in \citepub{c:mprtp}
calculates the per path rate based on the the following: a) characterize the
path based on the observed network behavior, b) choosing performant paths from
the available paths for active transmission, c) follows packet scheduling
rules that can be described by the multimedia application. We do not use
B-frames and do not discard any packets at the sender. Furthermore, we try to
maintain optimal playout by choosing paths that meet the latency constraints
and try to maintain a very short de-jitter buffer (<500\,\emph{ms}), so that
the scheduling algorithm can be extended to include interactive applications.
\subsection{Multipath Scheduling and Adaptive Playout}

The scheduling algorithm at startup assigns equal fractional distributions and
the per-path distribution changes depending on the observed path
characteristics. Hence, the MPRTP sender first calculates the estimated
receiver rate for each path based on the Subflow Receiver
Reports~\cite{draft.mprtp}. Second, the sender characterizes the paths based
on the observed packet discards and losses. Third, the sender chooses a set of
\emph{active paths} from the available paths. Forth, the sender follows a set
of packet scheduling rules. Fifth, the sender calculates the timescales at
which it will re-calculate the fractional distribution and lastly, it
calculates the per-path fractional distribution.

A path that reports discards and losses in a single or consecutive intervals
is considered \emph{mildly congested}. If this behavior is observed over 3
successive intervals, it is considered \emph{congested}. Furthermore, if a
path reports only losses and no discards in successive intervals, it is
considered \emph{lossy}. A path without losses or discards is considered
\emph{non-congested}.

A multipath sender chooses the paths that meet the capacity and latency
requirements. Next, it groups the paths based on the path latencies--bandwidth
is additive for paths with similar latencies~\cite{Wischik:2008:RPP}.
Subsequently, it sorts the path groups in decreasing order of
$\frac{bandwidth}{latency}$, so that groups with high bandwidth and low delay
are preferred. The endpoint chooses the set of paths from the groups that meet
the capacity requirements and marks these paths as \emph{`active'} and the
rest are marked \emph{`passive'}and used when the chosen paths fail. Depending
on the amount of packet loss (caused due to bit-error corruption) may affect
the quality of experience. Therefore, an MPRTP sender should avoid scheduling
packets on paths with losses. The scheduler observes the following rules:

\begin{itemize}
\setlength{\itemsep}{0pt}

  \item If the next scheduled frame is an I-frame then the resulting RTP
  packets are assigned to the path with the highest $\frac{bandwidth}{delay}$,
  bandwidth and lowest loss rate.

  \item On receiving a NACK, transmit the requested packet on the path with
  the highest $\frac{bandwidth}{delay}$, least RTT and lowest loss rate.

  \item Reduce the fractional traffic distribution on the \emph{mildly
  congested} and \emph{congested} paths in an attempt to reduce congestion on
  those paths.

\end{itemize}

To compensate for the difference in path latencies, the receiver calculates:
1) Packet skew based on the path jitter, 2) the Path Skew, based on media
value of packet skew on each path, and 3) the Playout Delay, based on the per
Path Skew. First, the endpoint calculates the packet skew of each packet
received on a path by subtracting the difference between reception timestamps
($TR$) and RTP timestamps ($TS$), $Packet\ Skew = (TR_j - TR_i) - (TS_j -
TS_i)$, where `i' and `j' are consecutive packets received on a path.

For each path the receiver maintains a Drift Window (DW), which is a sliding
window of 2 seconds of media packets or 100 packets, whichever is lower. We
chose a relatively small window size to avoid the receiver from under-flowing
by changing the playout very late. Every time the endpoint receives a packet
on a path it calculates the drift and inserts it in to the window. The
receiver then sorts the window and chooses the median ($\widetilde{DW}$) value
for calculating the path skew: $Path\ Skew = 0.01 \times \widetilde{DW} + 0.99
\times PathSkew_{prev}$. 

The path skew values are fed into the regular playout delay
calculation~\cite{Fober05,Colin03}: $Playout_{delay} = \frac{MAX([SW]) + 124
\times Playout_{prev}}{125}$.




\begin{table}
  \begin{center}
  \begin{tabular}{cccc} \hline
   & Avg. PSNR & $\sigma_{PSNR}$ & PLR\\ \hline
  \multirow {2}{*}{} 
  1-Path (no loss) & 48.427 & 0.00 & 0.00 \\ 
  2-Path (no loss) & 48.427 & 0.00 & 0.00 \\
  3-Path (no loss) & 48.427 & 0.00 & 0.00 \\ \hline
  \multicolumn{4}{c}{Variable losses per path} \\ \hline	
  \multirow {2}{*}{} 
  1-Path (0.5\% loss) & 40.887 & 0.506 & 0.49 \\
  1-Path (1\% loss) & 36.172 & 0.705 & 1.01 \\ %\hline
  2-Path (0-0.5\%) & 43.4 & 1.9 & 0.24 \\
  3-Path (0-1.0\%) & 40.5 & 0.49 & 0.48\\ \hline	
  \multicolumn{4}{c}{Variable RTT per path} \\ \hline
  2-Path & 48.303 & 0.278 & 0.004 \\ \hline
  3-Path & 48.164 & 0.32 & 0.0121\\ \hline
\end{tabular}
\caption{Comparing performance of using a single path with using multiple
paths}
\label{table-var-path}
\end{center}
\end{table}

In \citepub{c:mprtp}, we show that by using the performance of an MPRTP
endpoint does not deteriorate when comparing to the performance of a flow
using just a single interface. In our experiment in the testbed, we use the
results from a single-path media flow as the benchmark to compare the
performance of MPRTP. Table~\ref{table-var-path} shows that the performance of
endpoints implementing MPRTP compared to single path is not adversely
affected. When none of the paths exhibit any losses, the performance of MPRTP
was exactly the same, except that MPRTP induces an overhead because it uses
additional extension for identifying, monitoring and reporting subflows. In
our experiments in \citepub{c:mprtp} the RTP overhead for a 1 Mbps media flow
is an additional 1.275\,\emph{kbps} and the Multipath RTCP (MPRTCP) accounted
for $\approx$70\,\% of the total RTCP bandwidth ($\approx$0.25\,\emph{kbps}).
When we introduce losses, the PSNR drops for the single path, however, for the
multipath case the PSNR is significantly higher because the paths do not
necessarily exhibit losses at the same instance in time, hence, the MPRTP
scheduling algorithm is able to redistribute the capacity and preferring the
path with lower loss rate. Additionally, when the paths have dissimilar RTTs
(up to 150\,\emph{ms} of skew across paths), yet again the receiver is able to
playout packets across all paths and performs (compare PSNR) at par with the
single path. The scheduling algorithm and the adaptive dejitter buffer to
playout packets across different path skews is discussed in detail in
\citepub{c:mprtp}.


\section{Call Establishment and NAT Traversal}

When endpoints want to use multiple paths or offload traffic onto another path
(or interface) or move between networks, it requires the endpoint to either
change its IP address or use multiple IP addresses at the same time.
Typically, an endpoint changing its IP addresses breaks some of the higher
level protocols (e.g., TCP, RTP), unless the higher level protocol is designed
to be oblivious to the changes in IP address (e.g., SCTP~\cite{rfc4960}).

Various techniques exist for handling mobility, such as, Mobile IP, Proxy
Mobile IP, Locator/ID Separation Protocol (LISP), but these techniques are not
useful for enabling multipath because they attempt to assign a static IP
address to the endpoint and hence disables the use of multiple paths.
Endpoints will generally use a signaling protocol to establish a media
session. With the existence of such a signaling relationship, two alternatives
become available to advertise an endpoint's multiple interfaces:
\emph{in-band} (over the media path) or \emph{out-of-band} (over the signaling
path).

Typically, performing interface advertisement is tightly coupled with NAT and
firewall traversal. Endpoints implement NAT and FW traversal using Interactive
Connectivity Establishment (ICE)~\cite{rfc5245} procedures, which enables the
endpoints to ascertain connectivity between the endpoints by performing
connectivity tests before transmitting media. The endpoint usually advertises
the multiple interfaces in SDP, which usually couples the interface
advertisement to the offer/answer mechanism. The offer/answer mechanism is
excessive in this case, because a declarative mechanism would suffice. The
endpoint mainly wants to notify the other endpoints of its multiple
interfaces. Likewise, when multiple interfaces become available at the other
endpoint, it would notify its peers.

To summarize, in \cite{draft.mprtp}, we define an \emph{in-band} mechanism to
advertise interfaces in RTCP. The endpoint is able to update its existing
interfaces or advertise new ones, whenever the RTCP interval expires.
Advertising in-band is mainly useful when the endpoints are not deployed
behind NATs or the ICE agent works together with the MPRTP
stack~\cite{draft.mice}. In \cite{draft.mprtp.sdp}, we define the \emph{out-of
band} mechanism in SDP. The endpoint in this case performs the first round of
offer/answer exactly like it would do for a multimedia session using a single
path, but indicating it supports MPRTP and containing multiple \emph{ICE
candidates}. Later, when the connectivity checks for more than one path are
successful, each endpoint advertises its MPRTP interfaces.
Figure~\ref{chap7:fig_mprtp_arch} show the interworking of the MPRTP stack
with an ICE Agent implementing STUN connectivity checks. Irrespective of the
presence of a NAT, in \citepub{c:mprtp} we show that advertising the multiple
interfaces \emph{in-band} leads to a establishing the call (with MPRTP
capabilities) more quickly than when advertising the same interfaces
\emph{out-of-band}.


\section{Offloading and Multihoming}

In \citepub{c:mprtp}, we focus on spreading a constant bit rate (CBR) media
stream across multiple paths, for which we present a scheduling algorithm for
allocating traffic based on path characteristics. We use an adaptive dejitter
buffer at the receiver so that the endpoint can playback media packets from
paths with diverse characteristics. In our experiments, the application
configures the scheduling algorithm for maximum end-to-end latency of
400\,\emph{ms} and maximum path skew set to 200\,\emph{ms}. However, our work
is orthogonal to rate adaptation--which would just change the total media rate
to spread across each subflow.


\begin{figure}
    \centerline{
        {\includegraphics[width=0.8\textwidth] %clip=true, trim=0 1cm 0 1.5cm]
        {chap7-graph_variable_bw_13073-2p5-2}}
    }
    \caption{The plot shows MPRTP offloading media from a path with reducing
    capacity to another path with more capacity.}
    \label{chap7:fig_sim_var_bw}
\end{figure}

\textbf{\texttt{Offloading}}: In this scenario, the e2e capacity on one path
is variable, it demonstrates the sensitivity of the scheduling algorithm to
the changes in network capacity, which may be caused by \emph{cross-traffic}.
Path B in Figure~\ref{chap7:fig_sim_var_bw} shows the link with variable e2e
capacity and the per-instant bandwidth utilization by an MPRTP subflow. Note
that the scheduling algorithm uses cues on one path to reallocate the media on
to the other paths (observe the points where the link rate drops). The
scheduling algorithm also tries to probe the network, so that an equilibrium
state of fair sharing can be achieved. However, this is done at long
time-scales (order of seconds) so that the per-path load does not oscillate.

\textbf{\texttt{Multihoming}}: Figure~\ref{chap7:fig_sim_bb_3g} shows the
bandwidth utilization of a WLAN and 3G path and the overall bandwidth
distribution between the paths. The bandwidth is more evenly shared except
when the 3G path is constrained, the scheduling algorithm offloads the
remaining media on to the WLAN path, however, it does not quickly reallocate
the bandwidth it took away from the link to avoid bandwidth oscillations. This
is a useful feature for the scheduling algorithm because it can then use the
passive or idle paths for fallback. Moreover, the 3G path encounters packet
losses more often than on the WLAN path, which makes the scheduling algorithm
prefer sending more media over the WLAN path. Despite using two lossy paths
the PSNR of the media stream (see Table~\ref{table-bb-3g}) in this scenario is
close to optimum.

\begin{figure}
    \centerline{
        {\includegraphics[width=0.8\textwidth] %clip=true, trim=0 1cm 0 1.5cm]
        {chap7_graph_bb_3g_s17075-2p3-2}}
    }
    \caption{The plot shows a multihomed endpoint load-balancing a media flow
    over WLAN and 3G paths.}
    \label{chap7:fig_sim_bb_3g}
\end{figure}

\begin{table}[!t]%htbp]
\centering
% \resizebox{\textwidth}{!}{%\scalebox{0.75}
{
\begin{tabular}{ccccc} \hline
Path Characteristic & Avg. PSNR & $\sigma_{PSNR}$ & PLR\\ \hline
Offloading & 42.93 & 2.23 & 0.772 \\
Multihoming & 46.7173 & 0.21 & 0.33 \\ \hline
\end{tabular}
}
\caption{Performance of multipath scheduling when offloading (from a
constrained path) and multihoming (with WLAN and 3G paths)}
\label{table-bb-3g}
\end{table}

% \textbf{Conclusions}: We have explored the criteria for assigning traffic
% shares as a function of the diverse path properties and presented
% considerations for scheduling algorithms. Our evaluation shows that our
% design 1) allows exploiting multiple paths without performance degradation
% compared to suitable single-path cases—so that it is safe to deploy—and 2)
% enables load distribution and capacity aggregation in diverse scenarios.
% Mobile users (and operators) may benefit from aggregating or dynamically
% shifting load between different wireless interfaces of their mobile devices
% and MPRTP may assist well in bundling multiple wireless access networks for
% vehicular Internet access.

\section{Applying MPRTP to Group Communication}


Various conference architectures can be used to distribute the media in a
many-to-many communication scenario: centralized, unicast receive with
multicast send, full mesh, overlays, and trees
\cite{Li2010a,Noh2008,Singh2001}. When developing a conferencing systems for a
specific use-case, the scalability, reliability, quality and delay
characteristics, for each of these architecture needs to be considered. In the
case of of very large video conferences, such as, massive open online course,
seminars or conferences, we assume a low peer churn i.e., all participants
arrive and leave roughly at the same time. Also, active participants produce
the media flows, while the dormant/passive participant consume it. In
\cite{Globisch:AsymGrpComm}, we propose using multiple Application Layer
Multicast (ALM) distribution trees to broadcast the media from an active
speaker to the participants listening in to the conference. The ALM tree
network minimizes the end-to-end delay and reduces the load on the active
participants.

The use of multiple Application Layer Multicast (ALM) trees for media delivery
minimizes the end-to-end delay and results in asymmetric relationships between
participants and introduces complex forwarding. Chu \emph{et al.} show ALM as
a viable solution for real-time conferencing over the Internet~\cite{Chu2001}.
Banerjee et al.\cite{Banerjee2002} present an ALM protocol that has a
hierarchical control structure with low overhead. Noh \emph{et
al.}~\cite{Noh2008} use multiple trees to reduce the end-to-end delay and
determine the optimal number of ALM trees depending on specific network
characteristics. They conclude that the fan-out of a peer influences the
trade-off between the propagation delay and the queuing delay. Li et
al.~\cite{Li2010a} describes the use of multiple trees as a mechanism to scale
to more clients by introducing multiple focus-mixer structures where each
structure is dedicated to serving a set of clients in a region.

In \cite{draft.rtcp.overlay}, instead of using a centralized conferencing
server to maintain the media sessions and inserting peers or nodes in the
appropriate location, we propose protocol extensions (e.g., to RTCP) to help
nodes re-arrange themselves based on their pairwise connectivity, i.e.,
reconstructing the tree by preferring links with better network
characteristics.



 \chapter{Network-assisted Congestion Control}
 \label{chap:cc.nw}
 Figure~\ref{fig:4:fw} in Chapter~\ref{chap:cc.fw} shows the structure of the
congestion control framework described in this thesis. The framework
categorizes \emph{In-path} and \emph{Out-of-path} sources and
\emph{out-of-band} signaling for implementing congestion control, which are
discussed in this chapter. This chapter is based on our work on congestion
control for interactive multimedia applications, which is documented in
\citepub{c:3grc}, \citepub{c:glass}.

In \citepub{c:3grc}, when the available link capacity changes at a routers (in
this case, the 3G base stations), it notifies the endpoint connected to it
about the current capacity. Based on the notifications the sender adapts the
sending rate of the video call.

In \citepub{c:glass}, we explore the use of coverage maps for congestion
control. The map server collect throughput information from the mobile
clients, which also add geo-location information along with the throughput
information. This assists the map server to build a bandwidth and coverage map
and is queried by the mobile to predict coverage outage.

\section{In-path Sources}

ECN, PCN, BW indication


\begin{figure}
  \centerline{
    \subfloat{
      \includegraphics[width=0.5\textwidth, clip=true, trim=3cm 0 4.5cm 0]
      {chap8_graph_3g_tmmbr_a}
    }
    \subfloat{
      \includegraphics[width=0.5\textwidth, clip=true, trim=3cm 0 4.5cm 0]
      {chap8_graph_3g_tmmbr_b}
    }
  }
  \caption{Performance of C-NADU in a slow time-varying link and 3G network.}
  \label{fig:cnadu}
\end{figure}

\section{Out-of-path Source}

Congestion Maps

\chapter{Conclusions}
\label{chap:conc}
 
In this thesis, we describe our proposed classification of congestion control
cues (framework) in \ref{fw.fw}. When describing the different parts of the
framework, we also discuss the related work and our contributions in those
areas. In Chapter~\ref{chap:rg}, we aimed to classify congestion control cues
for real-time communication based on: \emph{where they are measured?} (by
in-path or out-of-path sources) and \emph{how they are reported?} (via in-band
or out-of-band signaling). In Chapter~\ref{chap:cc.fw}, we describe other
fundamental choices needed to implement congestion control. Choice of
congestion cues, reporting frequency and circuit-breakers. Additionally, we
also describe an evaluation suite for measuring the performance of the
proposed congestion control algorithms.

In chapter~\ref{chap:cc}, we describe congestion control implemented using
congestion cues from in-band sources and signaled in path (in RTCP). We
further categorize the congestion control algorithms based on \emph{where they
are implemented}: sender-driven scheme (e.g., TFRC), receiver-driven scheme
(e.g., TMMBR), or co-operative scheme (combination of sender and receiver,
e.g., C-NADU, RRTCC) and compare the performance of an algorithm in each
scheme. We observe that the performance of TFRC is bursty, which may lead to
poor user-experience, whilst TMMBR, C-NADU and RRTCC have a more stable
throughput. Lastly, TMMBR appears to be conservative and RRTCC appears to be
aggressive. C-NADU attempts to trade-off throughput for losses, but achieves
slightly higher throughput than TMMBR.


In chapter~\ref{chap:er-cc}, we address two problems: applicability of
error-resilience schemes, and using FEC to probe for available capacity to
implement congestion control for interactive multimedia flows. To the first
problem, we show that NACK, Reference Picture Selection, Unequal Error
Protection and adaptive slice-size can be used for different levels of
latencies and observed packet loss ratio. To the second problem: we propose a
congestion control scheme where instead of increasing the rate when network
conditions seem stable, the sender introduces FEC for one RTCP interval. If
the FEC and the media packets are received successfully, the sender increases
the sender rate with the amount of FEC rate. The trade-off is that we get a
smoother ramp-up and if a packet gets lost, it may be recovered by the FEC
packet. The sender also implements a variable FEC interval, i.e., it varies
the number of packet for which FEC is generated. Hence, if the sender thinks
that it is underutilizing the link by large margin, it introduces a shorter
FEC interval (up to 33\,\% redundancy) and therefore ramps up quickly.
Consequently when it thinks that it is closer to the bottleneck link capacity,
it introduces a longer FEC interval (up to 8\,\% redundancy) and therefore is
conservative in probing for available capacity. Our experiments show that by
using adaptive FEC for probing, the endpoints are able to recover 15-25\,\% of
the lost packets and $\approx$90\,\% of the time using FEC subsequently
results in an increase in the media rate. These results are comparable to our
earlier experiments using a fixed FEC interval throughput the duration of the
call for error-resilience, where we were able to 20-24\;\% of the lost
packets. We believe using FEC for congestion control in interactive multimedia
has not been explored in depth by the community, partly because interactive
multimedia flows have very tight delay constraints and FEC may not arrive in
time for recovering the packet.

In chapter~\ref{chap:mprtp}, we propose using the multiple interfaces of an
endpoint (e.g., mobile device, tablet, SOHO gateways) for increasing
throughput and robustness. It corresponds to using congestion cues from
out-of-band sources (different paths) and signaled in path (in RTCP). We
design a protocol extension (MPRTP) to RTP that is backwards compatible and
exploits multipath capabilities. We implement a scheduling algorithm that
takes application requirements and current path characteristics into
consideration to send packets over multiple paths. At the receiver, we
implement a per-path and aggregate de-jitter buffer, which interwork to
playout packets smoothly even when the path skew is high. Our experiments show
that the performance of MPRTP is not degraded compared to single path RTP, so
that it is safe to deploy. It enables load distribution and capacity
aggregation, which enables features like mobility, offloading, multihoming.

In chapter~\ref{chap:cc.nw}, we describe two congestion control mechanisms,
both use out-of-band signaling, but use in-path and out-of-path sources,
respectively. First, the in-path sources is based on receiving congestion cues
from middleboxes (e.g., 3G or LTE base-stations) and use standard RTP
extensions (e.g., TMMBR) for signaling . Second, the out-of-path sources are
crowd-sourced 3G coverage maps that collect throughput and geo-location
information from users and then notifies endpoints about the available
capacity in the region. We propose a system to collect and query coverage
maps. The endpoints query the coverage server using out-of-band signaling to
discover areas of poor coverage, based on these notifications the endpoints
vary their sending rate.

In terms of innovativeness, the congestion control framework provides options
to look beyond using congestion cues reported by the receiver on a given path.
The framework offers the congestion control algorithm to use multiple paths to
either aggregate capacity or for increasing resilience, to use feedback
notifications from smart middleboxes along the path or to build a coverage or
map that provides congestion notification as a third party services. Each of
these techniques apply to one of the four areas described in the framework.
However, the definitions of each area is generalized enough to allow
application of a broad-range of techniques not just those proposed in this
thesis.


\section{Future Work}

The algorithms described in each area of the framework currently works
independently of each-another and for each of the proposed technique, we make
deployment recommendations. However, making them interwork together deserves
further investigation.

%%%%%%%%%%%%%%%%%%%%%%%%%%%%%%%%%%%%%%%%%%%%%%%%%%%%%%%%%%%%%%%%%%%%%%%%%%%%%%

%% The following commands are for article dissertations, remove them if you
%% write a monograph dissertation.

% Errata list, if you have errors in the publications.
%\errata

%%%%%%%%%%%%%%%%%%%%%%%%%%%%%%%%%%%%%%%%%%%%%%%%%%%%%%%%%%%%%%%%%%%%%%%%%%%%%%

%% The first publication (journal article)
% Set the publication information.
% This command musts to be the first!

\addpublication[conference]{V. Singh, S. McQuistin, M. Ellis, C.
Perkins}{Circuit Breakers for Multimedia Congestion Control}{IEEE Packet
Video}{San Jose, USA}{Dec}{2013}{IEEE}{c:cb}

% Add the dissertation author's contribution to that publication.

\addcontribution{The author of this dissertation was the main author of this
paper~\cite{Singh:CB.eval}. He equally contributed to the ideas and concepts
discussed in the paper and he also contributed to implementing the algorithm
and the overall system, conducting experiments for the interactive video
scenario and analyzing the results.}

% Add the errata of the publication, remove if there are none (the order can
% be interchanged with \addauthorscontribution).
%\adderrata{This is wrong}
% Add the publication pdf file, the filename is the parameter (must be the
% last).

% \addpublicationpdf{add/9_cb.pdf}

%%%%%%%%%%%%%%%%%%%%%%%%%%%%%%%%%%%%%%%%%%%%%%%%%%%%%%%%%%%%%%%%%%%%%%%%%%%%%%

%% The second publication (conference article, note the optional parameter)
% Set the publication information.

\addpublication[conference]{V.Singh, J. Ott, I. D. D. Curcio}{Rate adaptation
for 3G Conversational Video}{IEEE Infocom Workshops}{Rio de Janeiro,
Brazil}{04}{2009}{IEEE}{c:3grc}

% Add the dissertation author's contribution to that publication.

\addcontribution{The author of this dissertation was the main author of the
paper~\cite{Singh:3gRC}. He equally contributed to the ideas and concepts
discussed in the paper, and also implemented the algorithm and the overall
system, conducted the experiments and analyzed the results.}

% \addpublicationpdf{add/2_3g_ra.pdf}

%%%%%%%%%%%%%%%%%%%%%%%%%%%%%%%%%%%%%%%%%%%%%%%%%%%%%%%%%%%%%%%%%%%%%%%%%%%%%%

%% The third publication (another journal paper, accepted for publication,
%% note the optional parameter)
% Set the publication information, detailed information can be empty

\addpublication[conference]{V. Singh, J. Ott, I. D. D. Curcio}{Rate Adaption
for Conversational Video in Heterogeneous Environments}{IEEE World of
Wireless, Mobile and Multimedia (WoWMoM)}{San Francisco,
USA}{06}{2012}{IEEE}{c:hetrc}

% Add the dissertation author's contribution to that publication.

\addcontribution{The author of this dissertation was the main author of the
paper~\cite{Singh:HetRC}. He equally contributed to the ideas and concepts
discussed in the paper, and also implemented the algorithm and the overall
system, conducted the experiments and analyzed the results.}

% \addpublicationpdf{add/3_het_ra.pdf}

%%%%%%%%%%%%%%%%%%%%%%%%%%%%%%%%%%%%%%%%%%%%%%%%%%%%%%%%%%%%%%%%%%%%%%%%%%%%%%

\addpublication[conference]{V. Singh, A. A. Lozano, J. Ott}{Performance
Analysis of Receive-Side Real-Time Congestion Control for WebRTC}{IEEE Packet
Video}{San Jose, USA}{Dec}{2013}{IEEE}{c:eval}

% Add the dissertation author's contribution to that publication.

\addcontribution{The author of this dissertation was the main author of this
paper~\cite{Singh:rrtcc.eval}. His contribution consisted of designing the
experiments and analyzing the results discussed in the paper.}

% \addpublicationpdf{add/10_rrtcc.pdf}

%%%%%%%%%%%%%%%%%%%%%%%%%%%%%%%%%%%%%%%%%%%%%%%%%%%%%%%%%%%%%%%%%%%%%%%%%%%%%%

\addpublication[conference]{Devadoss J., Singh V., Liu C., Ott J., Wang Y-K.,
Curcio I.}{Evaluation of Error Resilience Mechanisms for 3G Conversational
Video}{IEEE ISM}{Berkley, USA}{Dec}{2008}{IEEE}{c:err}

% Add the dissertation author's contribution to that publication (the order
% can be interchanged with \adderrata).

\addcontribution{The author of this dissertation was one of the co-authors of
the paper~\cite{Devadoss:3gErr}. His contribution consisted of providing 2 out
of the 4 ideas (NACK, Slice-size adaptation) discussed in the paper,
implementing the corresponding algorithms, conducting the  experiments and 
co-editing the paper.}

% \addpublicationpdf{add/1_3g_er.pdf}

%%%%%%%%%%%%%%%%%%%%%%%%%%%%%%%%%%%%%%%%%%%%%%%%%%%%%%%%%%%%%%%%%%%%%%%%%%%%%%

\addpublication[submitted]{M. Nagy, V. Singh, J. Ott, L. Eggert}{Rate-control
using FEC for Interactive Multimedia Communication}{ACM Multimedia Systems (MMSys)}
{Singapore,Singapore}{03}{2014}{ACM}{c:fecrc}

% Add the dissertation author's contribution to that publication.

\addcontribution{The author of this dissertation was one of the co-authors of
the paper~\cite{Singh:FECRC}. His contribution consisted of discussing the
main idea and concept for the paper with the lead author, designing the
experiments, implementing the comparative algorithms and providing the
corresponding results, and co-editing the paper.}

% \addpublicationpdf{add/6_fec_ra.pdf}

%%%%%%%%%%%%%%%%%%%%%%%%%%%%%%%%%%%%%%%%%%%%%%%%%%%%%%%%%%%%%%%%%%%%%%%%%%%%%%

\addpublication[conference]{V. Singh, S. Ahsan, J. Ott}{MPRTP: Multipath
Considerations for Real-time Media}{ACM Multimedia Systems (MMSys)}{Oslo,
Norway}{02}{2013}{ACM}{c:mprtp}

% Add the dissertation author's contribution to that publication.

\addcontribution{The author of this dissertation was the main author of the
paper~\cite{Singh:MPRTP}. His contribution consisted of providing the main
idea for the paper, providing the configurations for the experiments,
performing the statistical analysis, and acting as the main editor of the
paper.}

% \addpublicationpdf{add/5_mprtp.pdf}

%%%%%%%%%%%%%%%%%%%%%%%%%%%%%%%%%%%%%%%%%%%%%%%%%%%%%%%%%%%%%%%%%%%%%%%%%%%%%%

%% The fourth publication (yet another journal paper, submitted for
%% publication, note the optional parameter) Note that you are allowed to use
%% this option only when submitting the dissertation for pre-examination!
% Set the publication information, detailed information is not printed

\addpublication[conference]{V. Singh, J. Ott, I. D. D. Curcio}{Predictive
Buffering for Streaming Video in 3G Networks}{IEEE World of Wireless, Mobile
and Multimedia (WoWMoM)}{San Francisco, USA}{06}{2012}{IEEE}{c:glass}

% Add the dissertation author's contribution to that publication.

\addcontribution{The author of this dissertation was the main author of the
paper~\cite{Singh:Glass}. He equally contributed to the ideas and concepts
discussed in the paper, and also implemented the algorithm and the overall
system, conducted the experiments and analyzed the results.}

% \addpublicationpdf{add/4_glass.pdf}

%%%%%%%%%%%%%%%%%%%%%%%%%%%%%%%%%%%%%%%%%%%%%%%%%%%%%%%%%%%%%%%%%%%%%%%%%%%%%%

%\bibliographystyle{IEEEtran} %plain, amsalpha

\bibliographystyle{plain}

% argument is your BibTeX string definitions and bibliography database(s)
%\bibliography{IEEEabrv,../allpapers}

\bibliography{bib/diss,bib/rfc}

\end{document}
