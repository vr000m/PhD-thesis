Figure~\ref{fig:4:fw} in Chapter \ref{chap:cc.fw} shows the structure of the
congestion control framework described in this thesis. The framework includes
\emph{In-path} sources and \emph{In-band} signaling for implementing
congestion control. All these are discussed in this section.

This chapter is based on our work on congestion control for interactive
multimedia applications, which is documented in \citepub{c:3grc},
\citepub{c:hetrc}, \citepub{c:eval}, \cite{draft.xr.discard.rle},
\cite{draft.xr.discard} and \cite{Singh:control.loops.api}.

In \citepub{c:3grc}, we propose a new congestion control algorithm for the
mobile (e.g., 3G) environment, to be deployed in IP Multimedia System (IMS).
The main distinction between mobile (e.g., 3G, LTE) and other wireless
environments (e.g., 802.11x) is the media streams are transmitted using the
\emph{unacknowledged mode}; the packets corrupted due to bit-errors (e.g.,
interference) are not re-transmitted and hence, packets incur low delay. We
compare a sender-driven congestion control with receiver-driven and
network-assisted congestion control. In \citepub{c:hetrc}, we extend the
approach in \citepub{c:3grc} for deploying on the Internet and show the 