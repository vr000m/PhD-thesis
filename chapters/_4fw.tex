In the forthcoming deployment of WebRTC systems, we speculate that high
quality\footnote{normally, corresponds to increase in required bandwidth}
video confrencing will see wide adoption. Normally, to assure stability of the
network (and avoid congestion collapse), these real-time communication systems
will need to implement some kind of congestion control for their RTP-based
media traffic.

RTP transmits the media data over IP using a variety of transport layer
protocols such as UDP, TCP, and Datagram Congestion Control Protocol (DCCP).
Consequetly, congestion control for RTP-based media flows can be implemented
either in the application or the media flows are transmitted over congestion-%
controlled transport (TCP or DCCP). While using a congestion controlled
transport may be safe for the network, it is suboptimal for the media quality
unless the congestion-controlled transport is designed to carry media flows or
operates in a very low latency network ($<100$ms)~\cite{Brosh:tcp-real-time}.
On the other hand, using a non-congestion controlled transport (e.g., UDP),
the rate-adaptation is implemented in the application.  In this thesis, we
consider congestion control for unicast RTP traffic running over best-effort
IP network.

% CC should not cause queuing delay. Or define low-latency operation of
% multimedia cc.

Endpoints rely on RTCP feedback from the receiver to implement congestion
control. Hence, the design of congestion control algorithm needs to be aware
of the limits on the timing of the feedback. Normally congestion control
requires a tight control loop, which means that the receiving endpoint should
be able to provide feedback at very short intervals. For example in TCP, the
receiver sends an \emph{acknowledgement} packet in response to every packet
(or every few packets) it receives. Whereas, RTCP encourages infrequent
feedback and specifies an upper-bound on the fraction of the session media
bitrate that the feedback packets can use\footnote{The specified feedback rate
is $5\%$ for each multimedia session}. Hence, the congestion control should
take into 3 aspects into its design: congestion cues to report, block size of
each report, and frequency of these feedback reports.


\cite{draft.rmcat.feedback} discusses three options for the short report
intervals, they are:
\section{Congestion Cues: Reporting Frequency}

\textbf{Per-packet feedback report}: sends RTCP feedback every time the
endpoint receives a packet. For low bitrate media sessions (e.g., audio
streams) this would be quite difficult to achieve because the size of the
feedback packet would be comparable to the size of the media packet, i.e., the
feedback bitrate would larger than the $5\%$ fraction specified for it. If an
endpoint receives packets in a burts or at very short time intervals, the
endpoints will not be able to meet the timing requirements for per-packet
feedback because the RTCP timing interval calculation has a randomization
factor to avoid synchronizing feedback from multiple endpoints.

\textbf{Per-frame feedback report}: sends RTCP feedback every time the
endpoint recives a complete frame. This is mainly applicable to video where a
single video frame would be fragmented into multiple packets because the frame
size exceeds MTU size. Typically, an average size of an RTCP packet size in a
two-party call is $156$-$176$ bytes\footnote{The breakdown in bytes is:
UDP=16, IPv4=20 or IPv6=40, RTCP=8, SR=20, RR=24, SDES=28, one or more XR
blocks (20 each).}. For a 30 FPS biderectional video stream, the $rtcp\_bw
\approx 75$ \emph{kbps}, which requires the media session bitrate be set to a
value higher than $1.5$ \emph{Mbps}. Consequently, it would not be possible to
perform per-frame for sessions with lower media rates. It should be noted that
the requirements for the media session butrate needs to be re-calculated if
the number of participants change, or the number of reported blocks change or
the frame rate changes.

\textbf{Per-RTT feedback report}: sends RTCP feedback at regular intervals
based on the RTT estimate. The requirement for the media session rate would be
lower, if the RTT is higher than the frame inter-arrival time. The calculation
of the RTCP interval for the per-frame still applies, except that the frame
rate is replaced by the RTT estimate.

Any feedback interval above these three lower-limits would require a lower
media session bitrate.

\section{Congestion Cues: Sources and Signaling}
\label{fw.fw}


\begin{figure}[!h]
\includegraphics[scale=1.0]{chap2-fw-outline}
\caption{Congestion Cue Framework}
\label{fig:4:fw}
\end{figure}


\section{Requirements}
\label{fw.cc.req}

\section{Evaluation Criteria}
\label{fw.cc.eval}

% Flow scenarios
% Link properties
% Router properties