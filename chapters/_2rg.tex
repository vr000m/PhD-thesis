%Enabling adaptive multimedia applications/systems.

The research goal of this thesis is to discuss congestion control for real-
time media. To achieve a suitable end-user experience, a multimedia
system\footnote{the transmitting endpoint or a classifier in the network} can:
1) associate a DiffServ Code Point (DSCP)~\cite{rfc2474} to the media packets;
therby, enabling Quality of Service (QoS). Using DSCP poses some challenges,
which are discussed in Section~\ref{rg.ch.dscp}. 2) instruct the encoder to
modify the encoding rate to a certain target rate. To achieve media rate-
adaption the endpoint needs to monitor and respond to congestion cues.
Additionally, we  discuss requirements for media rate-adaptation.


\section{Challenges with DSCP Markings}
\label{rg.ch.dscp}

DiffServ assigns each data packet to a traffic class and the network manages
each traffic class differently, thereby some traffic classes receive
preferential treatment (e.g., lower delay, lower losses)~\cite{rfc2475}. The
routers overcome congestion between traffic classes by implementing
\emph{priority queuing}, \emph{fair queuing}, or \emph{weighted fair queuing
(WFQ)}~\cite{rfc5865}; for congestion within the same traffic class the
router discards packets using \emph{tail drop} or \emph{Random Early Detection
(RED)}~\cite{Floyd:RED}.

Consequently, DiffServ needs to be implemented on every router along the data
path and configured to have the same forwarding policy (i.e., belong to the
same DiffServ administrative domain). 


\section{Congestion Cues}

