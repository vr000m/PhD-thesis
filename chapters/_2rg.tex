%Enabling adaptive multimedia applications/systems.

The research goal of this thesis is to discuss congestion control for real-
time media. To achieve a suitable end-user experience, a multimedia
system\footnote{the transmitting endpoint or a classifier in the network} can:
1) associate a DiffServ Code Point (DSCP) to the media packets; therby,
enabling Quality of Service (QoS). Using DSCP poses some challenges, which are
discussed in Section~\ref{rg.ch.dscp}. 2) instruct the encoder to modify the
encoding rate to a certain target rate. To achieve media rate adaption the
endpoint needs to monitor and respond to congestion cues, we discuss the
sources of these cues in Section~\ref{rg.ch.cc.fw}


\section{Challenges with DSCP Markings}
\label{rg.ch.dscp}




\section{Congestion Control Framework}
\label{rg.ch.cc.fw}
