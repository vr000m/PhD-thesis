%Enabling adaptive multimedia applications/systems.

The research goal of this thesis is to discuss congestion control for real-
time media. To achieve a suitable end-user experience, a multimedia
system\footnote{the transmitting endpoint or a classifier in the network} can:
1) associate a DiffServ Code Point (DSCP)~\cite{rfc2474} to the media packets;
therby, enabling Quality of Service (QoS). Using DSCP poses some challenges,
which are discussed in Section~\ref{rg.ch.dscp}. 2) instruct the encoder to
modify the encoding rate to a certain target rate. To achieve media rate-
adaption the endpoint needs to monitor and respond to congestion cues.
Additionally, we  discuss requirements for media rate-adaptation.


\section{Multimedia Ecosystem}
%definition of Multimedia System

\textbf{Endpoint}:

\textbf{Multimedia System}:

\textbf{Multimedia Environment}:

\section{Challenges with DSCP Markings}
\label{rg.ch.dscp}

DiffServ assigns each data packet to a traffic class on a hop-by-hop basis and
the routers manage each traffic class differently, thereby some traffic
classes receive preferential treatment (e.g., lower delay, lower losses) in
the network~\cite{rfc2475}. The routers overcome congestion between traffic
classes by implementing \emph{priority queuing}, \emph{fair queuing}, or
\emph{weighted fair queuing (WFQ)}~\cite{rfc4594}; for congestion within the
same traffic class the router discards packets using \emph{tail drop} or
\emph{Random Early Detection (RED)}~\cite{Floyd:RED}.

Consequently, DiffServ needs to be implemented on every router along the data
path and configured to have the same forwarding policy, i.e., the routers have
to belong to the same DiffServ administrative domain for the packets to be
treated in exactly the same way at each hop. However, if packets traverses
across DiffServ domains, typically between Internet Service Providers (ISPs),
it is quite possible that the ISPs do not implement a policy for each
corresponding traffic class; when this happens the routers use the default
policy to forward packets, and as a result lose any opportunity of
differentiated service. Especially with video traffic which can be marked in
varying ways depending on the type of application (multicast, broadcast,
streaming, conversational, with each category having its own policy), the ISP
sometimes choose to ignore these different categories for video and chooses
just one and marks all video packets with the same DSCP~\cite{rfc5865}. These
\emph{generic} markings may be contrary to the intended DSCP of the multimedia
system.

% generic markings

% cases where it might work

Despite the above listed challenges, DSCP markings can help in some
environments~\cite{draft.rtcweb.qos}:

\begin{itemize}
	\item If the congested link is the last-but-one-hop broadband router that
	often connects a residential or Small Office/Home Office (SOHO) then the 
	DSCP markings can help in prioritizing the data traffic. 
	\item If the packets traverse a congested wireless link and the service 
	provider did not scrub out the DSCP markings, the markings may help.
\end{itemize}

\section{Enabling Adaptive Multimedia Systems}

\section{Contribution to Knowledge}