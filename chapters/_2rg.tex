
Compared to video streaming, video communication is more challenging. First,
it requires implementing a NAT traversal mechanism for communicating between
peers that may be behind a NATs or firewalls. Second, it requires agreeing on
a set of common codecs, protocols and formats to avoid negotiation failure.
Lastly, to provide  a decent end-user experience, it requires either the end-
point or a classifier to associate a DiffServ Code Point (DSCP) to the media
packets; therby, enabling Quality of Service (QoS). Additionally, it should be
able to adapt the media quality to the end-to-end path characteristics.

The research goal of this thesis is to discuss congestion control for real-time
media.

\section{Challenges with DSCP Markings}


\section{Congestion Control Framework}
