% intro

The Real-time Transport Protocol (RTP)~\cite{rfc3550} is designed for
multimedia telephony (voice-over-IP, video conferencing, telepresence
systems), multimedia streaming (video-on-demand, live streaming), and
multimedia broadcast. RTP's design is based on the fundamental principles of
\textit {application-layer framing} and \textit{integrated layer
processing}~\cite{clark:alf}. To this end, RTP provides the following
mechanisms: source and payload type identification, stream synchronization,
packet loss and re-ordering, and media stream monitoring. RTP utilizes the RTP
Control Protocol (RTCP) to report the performance of the media stream.
Figure~\ref{fig:3:rtp:model} describes the features provided by RTP and RTCP.
The media sender transmits encoded media encapsulated in RTP; in addition it
also sends RTCP Sender Reports (SR) to facilitate playback synchronization of
different media streams (typically audio and video). The receiver maintains a
dejitter buffer to reorder media packets and play them out as per the timing
information encoded in the packet. If a packet is missing the receiver attempts
to either recover the lost packet or conceal the error. Lastly, the receiving
endpoint reports rough or detailed statistics that enables the media sender to
adapt its media encoding rate, change to a better codec, or vary the amount of
forward error correction.

\begin{figure}[!h]
\centering{
  \includegraphics[width=\textwidth]{chap3-fig-rtp-rtcp}
}
\caption{RTP and RTCP for adaptive real-time applications.{\scriptsize Source:
J\"org Ott, ``Networked Multimedia Protocols and Systems''}.}
\label{fig:3:rtp:model}
\end{figure}

Figure~\ref{fig:3:rtp.hdr} describes the RTP packet header format. The
\textit{synchronization source} (SSRC) assists in determining the source
endpoint, typically useful when an endpoint sends multiple media streams that
need to be synchronized (e.g., audio and video lip-sync). The \textit{RTP
timestamp} assists in playing out the received packets at the appropriate
instance of time and recomposing the media frame from RTP packets. The
\textit{RTP sequence number} assists in identifying the lost packets and 
re-ordering packets in the case of out-of-order packet arrival. Lastly, RTP uses
the \textit{`payload type'} (PT) to describe the encoding of the media data it is
carrying. Consequently, each codec needs to specify its corresponding payload
format.

% \begin{figure}[!h]
% \centerline{\includegraphics[width=\textwidth]{chap3_fig_hdr_rtp}}
% \caption{shows the RTP packet format that encapsulates the media data.}
% \label{fig:3:rtp.hdr}
% \end{figure}

\begin{figure}[!h]
\begin{spacing}{0.5}
\centering
{\small
\begin{verbatim}
    0                   1                   2                   3
    0 1 2 3 4 5 6 7 8 9 0 1 2 3 4 5 6 7 8 9 0 1 2 3 4 5 6 7 8 9 0 1
   +-+-+-+-+-+-+-+-+-+-+-+-+-+-+-+-+-+-+-+-+-+-+-+-+-+-+-+-+-+-+-+-+
   |V=2|P|X|  CC   |M|     PT      |       sequence number         |
   +-+-+-+-+-+-+-+-+-+-+-+-+-+-+-+-+-+-+-+-+-+-+-+-+-+-+-+-+-+-+-+-+
   |                           timestamp                           |
   +-+-+-+-+-+-+-+-+-+-+-+-+-+-+-+-+-+-+-+-+-+-+-+-+-+-+-+-+-+-+-+-+
   |           synchronization source (SSRC) identifier            |
   +=+=+=+=+=+=+=+=+=+=+=+=+=+=+=+=+=+=+=+=+=+=+=+=+=+=+=+=+=+=+=+=+
   |            contributing source (CSRC) identifiers             |
   |                             ....                              |
   +-+-+-+-+-+-+-+-+-+-+-+-+-+-+-+-+-+-+-+-+-+-+-+-+-+-+-+-+-+-+-+-+
\end{verbatim}
}
\end{spacing}
\caption{The RTP packet format that encapsulates the media data.}
\label{fig:3:rtp.hdr}
\end{figure}


\begin{figure}[!h]
\begin{spacing}{0.5}
{\footnotesize
\begin{verbatim}
          0                   1                   2                   3
          0 1 2 3 4 5 6 7 8 9 0 1 2 3 4 5 6 7 8 9 0 1 2 3 4 5 6 7 8 9 0 1
         +-+-+-+-+-+-+-+-+-+-+-+-+-+-+-+-+-+-+-+-+-+-+-+-+-+-+-+-+-+-+-+-+
  header |V=2|P|    RC   |   PT=SR=200   |             length            |
         +-+-+-+-+-+-+-+-+-+-+-+-+-+-+-+-+-+-+-+-+-+-+-+-+-+-+-+-+-+-+-+-+
         |                         SSRC of sender                        |
         +=+=+=+=+=+=+=+=+=+=+=+=+=+=+=+=+=+=+=+=+=+=+=+=+=+=+=+=+=+=+=+=+
  sender |              NTP timestamp, most significant word             |
  info   +-+-+-+-+-+-+-+-+-+-+-+-+-+-+-+-+-+-+-+-+-+-+-+-+-+-+-+-+-+-+-+-+
         |             NTP timestamp, least significant word             |
         +-+-+-+-+-+-+-+-+-+-+-+-+-+-+-+-+-+-+-+-+-+-+-+-+-+-+-+-+-+-+-+-+
         |                         RTP timestamp                         |
         +-+-+-+-+-+-+-+-+-+-+-+-+-+-+-+-+-+-+-+-+-+-+-+-+-+-+-+-+-+-+-+-+
         |                     sender's packet count                     |
         +-+-+-+-+-+-+-+-+-+-+-+-+-+-+-+-+-+-+-+-+-+-+-+-+-+-+-+-+-+-+-+-+
         |                      sender's octet count                     |
         +=+=+=+=+=+=+=+=+=+=+=+=+=+=+=+=+=+=+=+=+=+=+=+=+=+=+=+=+=+=+=+=+
  report |                 SSRC_1 (SSRC of first source)                 |
  block  +-+-+-+-+-+-+-+-+-+-+-+-+-+-+-+-+-+-+-+-+-+-+-+-+-+-+-+-+-+-+-+-+
    1    | fraction lost |       cumulative number of packets lost       |
         +-+-+-+-+-+-+-+-+-+-+-+-+-+-+-+-+-+-+-+-+-+-+-+-+-+-+-+-+-+-+-+-+
         |           extended highest sequence number received           |
         +-+-+-+-+-+-+-+-+-+-+-+-+-+-+-+-+-+-+-+-+-+-+-+-+-+-+-+-+-+-+-+-+
         |                      interarrival jitter                      |
         +-+-+-+-+-+-+-+-+-+-+-+-+-+-+-+-+-+-+-+-+-+-+-+-+-+-+-+-+-+-+-+-+
         |                         last SR (LSR)                         |
         +-+-+-+-+-+-+-+-+-+-+-+-+-+-+-+-+-+-+-+-+-+-+-+-+-+-+-+-+-+-+-+-+
         |                   delay since last SR (DLSR)                  |
         +=+=+=+=+=+=+=+=+=+=+=+=+=+=+=+=+=+=+=+=+=+=+=+=+=+=+=+=+=+=+=+=+
  report |                 SSRC_2 (SSRC of second source)                |
  block  +-+-+-+-+-+-+-+-+-+-+-+-+-+-+-+-+-+-+-+-+-+-+-+-+-+-+-+-+-+-+-+-+
    2    :                               ...                             :
         +=+=+=+=+=+=+=+=+=+=+=+=+=+=+=+=+=+=+=+=+=+=+=+=+=+=+=+=+=+=+=+=+
         |                  profile-specific extensions                  |
         +-+-+-+-+-+-+-+-+-+-+-+-+-+-+-+-+-+-+-+-+-+-+-+-+-+-+-+-+-+-+-+-+
\end{verbatim}
}
\end{spacing}
\caption{The RTCP packet format for carrying the Sender Report (SR) and
the Receiver Report (RR). The SR carries transport statistics and enables 
stream synchronization, while the RR carries the receiver transport 
characteristics.}
\label{fig:3:rtcp.hdr}
\end{figure}

% \begin{figure}[!h]
% \centering{\includegraphics[width=\textwidth]{chap3_fig_hdr_rtcp}}
% \caption{shows the RTCP packet format for carrying the Sender Report (SR) and
% the Receiver Report (RR). The SR carries transport statistics and enables 
% stream synchronization, while the RR carries the receiver transport 
% characteristics.}
% \label{fig:3:rtcp.hdr}
% \end{figure}

The receiver measures the incoming streams and reports the coarse-grained
transport statistics in an RTCP Receiver Report (RR). The RTCP RR contains the
current loss fraction, jitter, and the highest sequence number received, and it
facilitates in calculating the round-trip time (RTT). The sender uses RTCP Sender Reports (SRs)
to assist in synchronizing the media streams (audio and video) by relating the
RTP timestamps of the individual media streams to the wall clock time (NTP)
and notifying the receiver about the current packet rate and bit rate.
Figure~\ref{fig:3:rtcp.hdr} shows the RTCP packet header format for a
interactive unicast media stream (i.e., both sending and receiving media).

\section{RTP Payload Formats}

The general principle for defining payload formats/types is to 
identify the encoding of the media packets. These encodings are either  
codec-specific (e.g., H.264, H.263, H.261, MPEG-2, JPEG, G.711, G.722, AMR, etc.),
or generic (e.g., Forward Error Correction (FEC), NACK, multiplexed streams).
Typically, a payload document specifies a well-defined packet format for media
codecs; it also defines \emph{aggregation rules} for codecs that produce
several small frames (e.g., audio) compared to the IP Maximum Transmission
Unit (MTU), and \emph{fragmentation rules} for codecs that produce large frames
(e.g., I-frames by video codecs). The main reason for fragmenting large
frames into smaller packets and not rely on IP fragmentation is that IP
fragmented packets are commonly discarded in the network, especially by NATs
or firewalls.

% Usually the RTP header is immediately followed by
% payload-specific header (payload format) and then by the media data. This
% allows the sending endpoint to semantically fragment large packets, which
% simplifies processing and decoding at the receiver (i.e., be able to decode
% individual packets without relying on receiving other packets).

\begin{figure}[!h]
\begin{spacing}{0.5}
{\footnotesize
\begin{verbatim}
    0                   1                   2                   3
    0 1 2 3 4 5 6 7 8 9 0 1 2 3 4 5 6 7 8 9 0 1 2 3 4 5 6 7 8 9 0 1
   +-+-+-+-+-+-+-+-+-+-+-+-+-+-+-+-+-+-+-+-+-+-+-+-+-+-+-+-+-+-+-+-+
   |V=2|P|X|  CC   |M|     PT      |       sequence number         |
   +-+-+-+-+-+-+-+-+-+-+-+-+-+-+-+-+-+-+-+-+-+-+-+-+-+-+-+-+-+-+-+-+
   |                           timestamp                           |
   +-+-+-+-+-+-+-+-+-+-+-+-+-+-+-+-+-+-+-+-+-+-+-+-+-+-+-+-+-+-+-+-+
   |           synchronization source (SSRC) identifier            |
   +=+=+=+=+=+=+=+=+=+=+=+=+=+=+=+=+=+=+=+=+=+=+=+=+=+=+=+=+=+=+=+=+
   |                 Payload Format-Specific Header                |
   +               +-+-+-+-+-+-+-+-+-+-+-+-+-+-+-+-+-+-+-+-+-+-+-+-+
   |               |                                               |
   +-+-+-+-+-+-+-+-+                                               +
   |                           Media Data                          |
   +                                                               +
   |                                                               |
   +-+-+-+-+-+-+-+-+-+-+-+-+-+-+-+-+-+-+-+-+-+-+-+-+-+-+-+-+-+-+-+-+
\end{verbatim}
}
\end{spacing}
\caption{Packet structure of an RTP packet encapsulating the
payload-specific header and the associated media data.}
\label{fig:3:pt.fmt}
\end{figure}

% \begin{figure}[!t]
% \centerline{\includegraphics[width=\textwidth]{chap3_fig_hdr_pt_fmt}}
% \caption{shows the packet structure of an RTP packet encapsulating the
% payload-specific header and the associated media data.}
% \label{fig:3:pt.fmt}
% \end{figure}

\section{RTP Header Extensions}

RTP header extensions carry media-independent information, i.e., data that may
be generically applicable to multiple payload formats (e.g., timing
information), and needs to be reported more frequently than RTCP reports are
emitted. A commonly-cited example is the sending of NACK packets for interactive video,
where media flows in both directions and RTP packets are generated every tens
of milliseconds. In this case, the RTP header extension can indicate which sequence numbers
were correctly received or lost, thereby not completely relying on the RTCP
receiver reports to send NACKs or ACKs.

The advantage of using header extensions is that they are backwards
compatible, i.e., an endpoint that does not understand them is able ignore
them. Some current use-cases for RTP header extensions include reporting the
network send timestamp: instead of bursting packets from a large frame on to
the network, the sender paces these packets.  Another example is equalizing a
client's audio levels across multiple streams in a video conference. Lastly,
RTP header extensions are generic, there is no need to redefine the same extension
for each media codec.

\section{RTCP Reporting Interval}
% timing

A closed control loop is formed by sending RTP media packets and receiving
RTCP feedback packets. The RTCP feedback interval is typically limited to a
small fraction of the session bandwidth (\emph{session\_bw}) as not to affect the media traffic.
The RTCP reporting interval is determined by the number of SSRCs in the
session (denoting the session size), and the chosen session bandwidth. The
session bandwidth (\emph{session\_bw}) is expected to be divided amongst the participants, but
oftentimes it is calculated as the sum of the average throughput of the
senders expected to be concurrently active. In the case of an audio conference,
the session bandwidth would be one sender's bandwidth, but for a video
conference, the session bandwidth would vary depending on the number of participants displayed on
the user interface. Consequently, the session bandwidth is supplied by the
session management layer so that the same value for the RTCP interval is calculated for each participant.


The recommended fraction of the session bandwidth allocated for control
traffic is 5\,\%. For many scenarios, including large conferences, where there
are a large number of receivers but a small number of senders, it is
recommended that a quarter of the reporting bandwidth (\emph{rtcp\_bw}) be
shared equally by the senders and the remaining three-quarters by the receivers. The
main reason for this allocation ratio is to allow newly-joining participants to quickly receive the
CNAME and synchronization timestamps from the Sender Reports (SRs). 
For new participants (even if they are just receivers), the RTCP
interval is halved to quickly declare their presence.  Lastly, the recommended
value for a fixed minimum RTCP interval is 5 seconds, while the value for a
reduced minimum is $\frac{360}{session\_bw}s$.  The fixed minimum RTCP
interval of \emph{5\,s} is suitable for unidirectional links or for sessions
that do not require monitoring of the reception quality statistics (e.g., IPTV),
while the reduced minimum RTCP interval is also suitable for participants in a
unicast bidirectional multimedia session. The reduced minimum RTCP interval
is suitable for sending timely feedback messages to either perform congestion
control or error repair; the interval is shorter than \emph{5\,s} for session
bandwidths greater than \emph{72\,kbps}.

% avpf

If an endpoint detects packet loss or the onset of congestion midway through a
reporting interval, the base RTP specification~\cite{rfc3550} (AVP profile)
does not allow the RTCP reports to be sent early and the endpoint has to wait for
the next scheduled RTCP report. In this case, the slow control loop causes
instability and oscillation in the media bit rate. To overcome this
shortcoming, endpoints implement the Extended RTP Profile for RTCP-Based
Feedback (AVPF profile)~\cite{rfc4585}, an extension to RTP's default timing
rules, to enable rapid feedback. This profile allows the endpoint to adjust the
RTCP reporting interval to send the RTCP feedback reports earlier than the
next scheduled RTCP report, sometimes even immediately, as long as the reporting
interval on average remains the same. Figure~\ref{fig:3:avpf.interval} shows
that with AVP profile, the endpoint reports at regular intervals, whereas with AVPF
the endpoint it gets the opportunity to send feedback early in every other reporting
interval. Along with the possibility of providing timely feedback, the AVPF
profile also defines a suite of error-resilience feedback messages, namely,
Negative Acknowledgments (NACK), Picture Loss Indication (PLI), Slice Loss
Indication (SLI), and the Reference Picture Selection Indication (RPSI).

\begin{figure}[!t]
\centering{\includegraphics[width=\textwidth]{chap3-fig-avpf-rtcp}}
\caption{The RTCP reporting interval as defined in a) AVP, b) AVPF.}
\label{fig:3:avpf.interval}
\end{figure}




\section{RTCP Extended Reports (XRs) for Performance Monitoring}

Endpoints use RTCP Extended Reports (XRs)~\cite{rfc3611} to describe complex
metrics that are not exposed by the RTCP Receiver Report (RR). Some examples of
XRs relevant to performance monitoring and congestion control are: de-jitter
buffer metrics~\cite{rfc7005}, Packet Delay Variation (PDV)~\cite{rfc6798},
delay metrics~\cite{rfc6843}, burst-gap discard~\cite{rfc7003}, burst-gap
loss~\cite{rfc6958}, Run-Length Encoded (RLE) loss~\cite{rfc3611}, discard
RLE~\cite{rfc7097}, the number of discarded packets~\cite{rfc7002} and
bytes~\cite{rfc7243}, summary statistics~\cite{rfc7004},
Quality of Experience (QoE)~\cite{draft.xr.qoe}, and loss
concealment~\cite{draft.xr.conceal}, etc. RTP allows for new metrics to be
defined; the main requirement is to document what is measured, how it is
measured and how it is reported to the other endpoints.


% The RTCP Extended Reports (XR) [RFC3611] allow reporting of more
% complex and sophisticated reception quality metrics, but do not
% change the RTCP timing rules.  RTCP extended reports of potential
% interest for congestion control purposes are the extended packet
% loss, discard, and burst metrics [RFC3611],
% [I-D.ietf-xrblock-rtcp-xr-discard],
% [I-D.ietf-xrblock-rtcp-xr-discard-rle-metrics],
% [I-D.ietf-xrblock-rtcp-xr-burst-gap-discard],
% [I-D.ietf-xrblock-rtcp-xr-burst-gap-loss]; and the extended delay
% metrics [RFC6843], [RFC6798].


\section{Codec Control Messages}
% codec control

Sometimes an endpoint needs to configure or notify the other endpoint's codec.
These messages are broadly classified as \emph{Transport Layer} and \emph
{Payload-specific} feedback messages~\cite{rfc4585, rfc5104}. The transport
layer messages are: Temporary Maximum Media Stream Bit Rate Request (TMMBR)
and Temporary Maximum Media Stream Bit Rate Notification (TMMBN). The
receiving endpoint uses the TMMBR message to configure the maximum encoding
bit rate of the media stream, while the sending endpoint uses the TMMBN to
inform the receiver of the updated bit rate. Therefore, transport layer
feedback messages are intended to transmit general purpose feedback messages,
independent of any particular codec or application.

On the other hand, the payload-specific feedback messages carry information
specific to a certain payload type and are acted upon by the codec layer. Some
examples of these type of messages are: Full Intra Request (FIR), 
temporal-spatial tradeoff, frame rate, frame size, maximum packet size or packet rate,
etc.~\cite{draft.avt.cop}.

\section{Reduced-Size RTCP Reports}
% non-compound feedback

An endpoint sends RTCP feedback as a \emph{compound}, or \emph{minimal}, RTCP
packet. A \emph{compound RTCP packet} as defined in~\cite{rfc3585} contains
at least a sender report (SR) or a receiver report (RR) or both, followed by
a Source Description (SDES) and any additional XR blocks. A \emph{minimal RTCP
packet} is one that contains an SR and/or RR, and is followed by an SDES
containing just the canonical name (CNAME)\footnote{The real name
(identifier) used to describe the source; it can be in any form desired by the
user. Of the SDES items (username, email, phone, geolocation, etc.), it is 
compulsory to include CNAME in every RTCP packet.}. Hence, every compound RTCP
packet is a minimal RTCP packet with additional report blocks. A typical RTCP
packet size for conversational multimedia streams is 80 bytes (RTCP=8, SR=20,
RR=24, SDES/CNAME =28).

Including any of the additional SDES items or adding XR blocks makes the
compound RTCP packet very large. On low bit rate links, these large compound
RTCP packets may introduce more delay. Therefore, it may be desirable to
logically fragment the report blocks in a compound RTCP packet and send them
independently. These fragmented report blocks are called \emph {reduced-size
RTCP packets}~\cite{rfc5506}. Unlike compound RTCP packets, to transmit a
reduced-size RTCP packet an endpoint does not need to include the minimal RTCP
report. However, when using reduced-size RTCP packets, minimal packets need to
be sent once in a while to keep the CNAME-SSRC binding alive.

Reduced-size RTCP reports are beneficial in wireless networks where the packet
loss rate increases with the packet size, i.e., larger-sized packets are more
susceptible to being dropped compared with smaller-sized packets. Additionally, smaller
packets have shorter serialization time, i.e., the amount of time it takes for
the endpoint to put the data packet onto the link is short.

The main reasons for the application to use reduced-size RTCP reports are:
1) to notify the other endpoint of events. Using the signaling channel would
incur at least one RTT while implementing it as an RTCP extension would merely
incur a one-way delay. 2) to send codec control (e.g., TMMBR) or feedback (e.g.,
NACK, RPSI) messages. These reduced-size messages are more likely to be
transmitted more often and with as little delay as possible, especially since
these types of messages are more likely to be sent when link conditions are
poor.


% relationship with SDP
\section{Session Setup}

% SDP O/A, declarative SDP, RTCP notify

There are several ways to set up an interactive or conversational multimedia
session, for example by implementing one of the following: H.323~\cite{H.323},
Session Initiation Protocol (SIP)~\cite{rfc3261}, or Jingle~\cite{XEP-0166}, an
extension to the Extensible Messaging and Presence Protocol
(XMPP)~\cite{rfc6120}.

SIP uses the Session Description Protocol (SDP)~\cite{rfc4566} to describe the
endpoint's transport and media capabilities. An SDP description defines a
single multimedia session, i.e., an association between a set of participants.
It may, however, carry multiple media streams in the session.

\begin{figure}[!h]
\centerline{\includegraphics[width=\textwidth]{chap3-fig-sig-med}}
\caption{The signaling and media paths between two endpoints engaging in
a video call.}
\label{fig:3:sig.media}
\end{figure}


The transport details in the SDP are mainly split into two parts: the protocol
for delivering media packets (currently, TCP, UDP, or SCTP), and the 
IP address of the endpoint. The protocol to deliver the media packets is
chosen by the application, but identifying the IP address and port of an
endpoint is a bit complex. It first requires gathering the endpoint's multiple
IP addresses and later exchanging them with the other endpoints to establish
connectivity.

Multiple IP addresses arise not only from multiple interfaces but also from
the presence of NAT devices in the network, which may change the IP address of
the outgoing packets. Since interactive media calls are between endpoints and
media streams may eventually traverse a NAT at both ends, in some cases, the
only way to deliver media packets between two endpoints would is by using
a relay\footnote{Traversal Using Relays around NAT (TURN)} on the public
Internet. Hence, the endpoint needs to discover its 2-tuple \texttt{[IP
address:port]} on the host, which is relatively easy followed by the public 
address, if behind a NAT~\cite{rfc5389}. Discovering an endpoint's public address
requires contacting a publicly-hosted STUN\footnote{Session Traversal
Utilities for NAT (STUN).} server and comparing the endpoint's
host addresses with the one observed by the STUN server. Lastly, the endpoint
discovers the address allocated by the TURN relay~\cite{rfc5766} and notifies
the other endpoints about its transport details. This collection of 2-tuples
is known as \emph{candidates}. First, each endpoint sorts its candidates
in decreasing order of priority and exchanges these candidate addresses in the
SDP with the other endpoint. On receiving the list of candidates, the endpoints
probes between each combination of addresses in the two candidate
lists; a pair of addresses is called a \emph{candidate pair}. The endpoint
chooses the first candidate pair that successfully establishes connectivity
(\emph{aggressive nomination}). This process of performing pair-wise
\emph{connectivity checks} is called Interactive Connectivity Establishment
(ICE)~\cite{rfc5245, rfc6544} and it relies on the STUN protocol~\cite{rfc5389} to establish connectivity
across a NAT. In case direct connectivity between the two endpoints fails 
to be established, the individual endpoints use the \emph{Traversal Using Relays
around NAT} (TURN) server and the associated protocol~\cite{rfc5766} to relay
media packets between them. Similar to the STUN server, the TURN
server is hosted on the public Internet.

% Depending on the type of NAT device (full-cone, address- or port-restricted
% cone, symmetric).

\begin{figure}[!h]
{\small
\begin{verbatim}
        v=0
        o=jdoe 2890844526 2890842807 IN IP4 10.0.1.1
        s=
        c=IN IP4 192.0.2.3
        t=0 0
        a=ice-pwd:asd88fgpdd777uzjYhagZg
        a=ice-ufrag:8hhY
        m=audio 45664 RTP/AVP 0
        a=rtpmap:0 PCMU/8000
        a=candidate:1 1 UDP 2130706431 10.0.1.1 8998 typ host
        a=candidate:2 1 UDP 1694498815 195.148.127.98 45664 typ srflx 
            raddr 10.0.1.1 rport 8998
\end{verbatim}
}
\caption{Sample SDP containing the sender's transport and media capabilities.}
\label{fig:3:sdp}
\end{figure}

The second part of SDP carries the media capabilities, together with the 
transport parameters, that bind the SDP to the Offer/Answer model. In the O/A
model, the sending endpoint \emph{offers} to the receiver a set of media
capabilities in decreasing order of preference, typically, multiple options of
audio/video codecs and ICE candidates. On receiving the sender's capabilities,
the receiver compares its media capabilities with the sender's and responds
with the one that best fits the receiver's requirements (\emph{answer}). The
\emph{offer is rejected} if the receiver is unable to pick any
of the options provided by the sender, or if the ICE connectivity checks fail.
Hence, the application at the sender needs to pick a minimum number of widely-available 
audio and video codecs to avoid negotiation failure. If the
\emph{offer is accepted}, both endpoints then know the following: 1) which
audio and/or video codecs to use, 2) the payload types of the encoded media
streams (possibly even their respective SSRCs), 3) to which IP address and
port number to send the media stream, 4) the media session bandwidth, if
indicated, and 5) the encryption keys, if encrypting traffic.

\begin{figure}
\centering{\includegraphics[width=0.9\textwidth]{chap3-fig-rtp-stack}}
\caption{Relationship of RTP with STUN, TURN, DTLS and the signaling protocol.}
\label{fig:3:rtp-stack}
\end{figure}

Figure~\ref{fig:3:rtp-stack} shows the relationship of the RTP stack with the
rest of the IP stack. RTP is usually transmitted over UDP, but under some
constrained situations (e.g., restrictive firewalls or NATs), RTP may be
encapsulated in TCP~\cite{rfc3550}. STUN~\cite{rfc5389} is used by ICE to
discover the presence of a NAT device and obtain the mapped (public) IP
address. When using a TURN relay server (in the presence of symmetric NATs or
when concealing the host address of the caller for privacy reasons), the RTP
packets are encapsulated inside the TURN's \emph{ChannelData}
message~\cite{rfc5766}. In Figure~\ref{fig:3:rtp-stack}, four media streams
(SSRC \#1-4) are transmitted by RTP over UDP, but two streams (SSRC \#1-2) are
relayed through a TURN server. Secure RTP (SRTP)~\cite{rfc3611} is a security
framework that provides confidentiality by encrypting RTP payload (not the RTP
headers) and supports source origin authentication. While SRTP is not the only
security mechanism for RTP~\cite{draft.srtp-not-must}, it is widely
applicable, especially to voice telephony and group communication. However,
the main challenge for SRTP is key management~\cite{draft.sec-opts}, since
many options exist (e.g., SRTP over DTLS in WebRTC~\cite{rfc5763}, MIKEY in
SIP~\cite{rfc3830}, Security Description in SDP~\cite{rfc4566},
ZRTP~\cite{RFC6189}, etc.).

\section{Summary}

In this chapter we introduced the basic features of RTP and RTCP; we discussed
the limits on reporting interval and the numerous RTP and RTCP extensions.
The introduction to RTP and RTCP largely provides context and helps understand
the design constraints  for multimedia congestion control which are discussed 
in more detail in the forthcoming chapters.
