% intro

Real-time Transport Protocol (RTP)~\cite{rfc3550} is suitable for multimedia
telephony (voice-over-IP, video conferencing, telepresence systems),
multimedia streaming (video-on-demand, live streaming), and multimedia
broadcast. RTP's design is based on the fundamental principles of \textit
{application-layer framing} and \textit{integrated layer
processing}~\cite{clark:alf}, i.e., it provides the following mechanisms:
source identification, packet playout time, packet loss and reordering, report
packet delay variation (jitter), and payload type identification. 

The `synchronisation source' (SSRC) assists in determining the source
endpoint, typically useful when an endpoint sends multiple media streams that
need to be synchronized (e.g., Audio/Video lip-sync). The `RTP timestamp'
assists in playing out the received packets at the appropriate instance of
time and recomposing the media frame from RTP packets. The `RTP sequence
number' assists in identifying the lost packets and re-ordering packets in
case of out-of-order packet arrival. Lastly, RTP uses `payload formats' to
describe the encoding of the media data it is carrying. Consequentely,
each codec needs to specify its corresponding payload format.

RTP utilizes RTP Control Protocol (RTCP) to monitor the performance of the
media stream. Using RTCP reports, the endpoints report loss fraction, jitter,
highest sequence number received, and calculate RTT. The RTCP reports also
assist in synchronizing the media streams (audio and video) by relating the
RTP timestamps of the individual media streams to the wall clock time and
measuring RTT.

% In this thesis, we consider congestion control for unicast RTP traffic
% flows.

RTP transmits the media data over IP using a variety of transport layer
protocols such as UDP, TCP, and Datagram Congestion Control Protocol (DCCP).
Consequetly, congestion control for RTP media flows can be implemented either
in the application or the media flows are transmitted over congestion-
controlled transport (TCP or DCCP). While using a congestion controlled
transport may be safe for the network, it is suboptimal for the media quality
unless the congestion-controlled transport is designed to carry media flows.
On the other hand, using a non-congestion controlled transport (e.g., UDP),
the rate-adaptation is implemented in the application based on RTCP reports.
i.e., the application is network aware.

% timing

% avpf

% codec control

% relationship with SDP
\section{RTP Extensions}
\label{rtp.ext}
