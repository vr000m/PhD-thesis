

\section{Research Methodology}

\section{Structure of the Thesis}

This thesis describes techniques to adapt media to changing network
characteristics for different types of multimedia systems. The work is mainly
a summary of scientific-papers, but is also supported by additional body of
work. We have co-authored a number of Internet Drafts\footnote{at the time of
writing, several of these are still in the ID state, but will be published as
RFCs (Request for Comments) shortly} that complement the  sceintific results
discussed in the thesis. The chapters describing the various parts of the
congestion control framework discuss both our scientific and engineering work,
while associating it with the pertinent related work in the area. The
remainder of the thesis is organized as follows.

Chapter~\ref{chap2} discribes the research goal of the thesis, which consists
of creating a framework for congestion control that meets the requirements for
multimedia systems.

%\chapter{RTP: Real-time Transport Protocol}

Chapter~\ref{chap3} provides the neccessary background information to RTP
(Real-time Transport Protocol). RTP together with RTCP (RTP Control Protocol)
forms the control loop that adapts media to the reported path characteristics.
We also provide an high-level overview to our proposed `Congestion Cues
Framework' and discuss criteria for evaluating congestion control for
multimedia systems. This section is based on the RTP protocol
suite~\cite{rfc3550, rfc4585, rfc3611, rfc5104, rfc5506} and our contributions
to it~\cite{draft.rmcat.evaluate}.

%RTP, AVPF, CCM, XR, reduced-size, 
%draft.rtp.cb, draft.rtp.tfrc, rrtcc
% \chapter{Rate-control for Interactive Multimedia}

Chapter~\ref{chap4} discusses the mechanisms available for congestion control
in interactive multimedia. We also discuss our contributions, which is
documented in \citepub{c:cb}, \citepub{c:3grc}, \citepub{c:hetrc},
\citepub{c:fecrc}, and \cite{draft.rtp.cb}.

% \chapter{Multihoming, Overlay and Mobility Consideration}

Chapter~\ref{chap5} discusses the impact of mobility 

% \chapter{Network-assisted Congestion Control}
Chapter~\ref{chap6}

% \chapter{Adaptive Error-Resilience and Congestion Control}
Chapter~\ref{chap7}

% \chapter{Conclusions}
Chapter~\ref{chap8} concludes the thesis and we analyse if the framework meets
the requirements.


%\lipsum[11-12]

%% Examples of article references, remove these from your manuscript!
% Uncomment them, if you want to see the results of these commands in this example document

 % Refer to the Journal paper 1 of this example document
%\citepub{j1} \& \cpub{j1} \& \cp{j1} \& \pageref{j1} \& \ref{j1}

% Refer to the Conference paper of this example document
\citepub[p.~2]{c1} \& \cpub[Sec.~ 1]{c1} \&  \cp[pp.~1--2]{c1} \& \pageref{c1} \& \ref{c1} 




