In recent years, video has emerged as the dominant traffic\footnote{In 2012,
51\,\% of mobile traffic was video.} on the Internet~\cite{cvni.13,dawn.zb},
partly due to the success of YouTube and other over-the-top media streaming
services (e.g., Netflix, Vimeo, Dailymotion, etc.). Video streaming emerged as
the dominant traffic only after it was easily accessible to Internet users in
the web-browser. The initial growth for video streaming is attributed to the
availability of Adobe Flash Video plugin and it became ubiquitous with the
inclusion of the video tag in HTML5. Currently, the same trend is apparent in
web-based real-time multimedia communication; web services either use Adobe's
Real-time Media Flow Protocol (RTMFP)~\cite{draft.rtmfp}\footnote{There are
other plugin-based services: Facebook Video based on Skype SDK, Google Talk,
Google's Hangout/Helpout services, etc.} or HTML5's Web-based Real-Time
Communication (WebRTC) stack~\cite{draft.webrtc}.


Real-time multimedia communication on the Internet is subject to the
unpredictability of the best-effort IP network. This uncertainty is mainly due
to packet loss, packet re-ordering, variable queuing delay. Additionally,
Buffer-bloat~\cite{gettys:bufferbloat} and drop-tail queues in the router may
cause long delays and bursty losses. Video streaming overcomes most of these
challenges by having larger playout buffers, this behavior is independent of
streaming over UDP (e.g., RTSP or SIP) or TCP (e.g., DASH). However, real-time
communication uses very small buffers and hence, endpoints modify the sending
rate to best match the available capacity and not cause excessive delay. For
real-time communication, TCP is at best suitable for paths with very low path
latencies (<100\,\emph{ms})~\cite{Brosh:tcp-real-time}, hence, UDP carries
real-time media flows.

% Furthermore, in video streaming, when the path characteristics change, the
% client switches between files with the same media content pre-encoded at
% different quality levels (\emph{rate-switching}). However, this strategy is
% not applicable to real-time communication, where the sending endpoint needs 
% to immediately adapt the media encoding rate to the available path capacity.

% While video communication has existed for the last decade (via Skype, etc.),
% there is no standardized congestion-control, but many have been proposed in
% the past. To tackle congestion control, the IETF has chartered a new working
% group, RMCAT\footnote{http://tools.ietf.org/wg/rmcat/} that aims at
% standardizing congestion-control for real-time communication, but it is
% expected to be a multi-year process\cite{jennings:2013:webrtc}.

%Compared to video streaming, video communication is more challenging. First,
%it requires implementing a NAT traversal mechanism for communicating between
%peers that may be behind a NATs or firewalls. Second, it requires agreeing on
%a set of common codecs, protocols and formats to avoid negotiation failure.
%Lastly, to provide  a decent end-user experience, it requires either the end-
%point or a classifier to associate a DiffServ Code Point (DSCP) to the media
%packets; therby, enabling Quality of Service (QoS). Additionally, it should be
%able to adapt the media quality to the end-to-end path characteristics.

This thesis is about enabling congestion control for real-time communication;
any congestion control also faces further challenges such as, variability in
the captured motion, variability in the size of the frames produced by the
media codec (I or P frames), the responsiveness of the codec to produce the
media stream for a new bit rate. Additionally, the congestion control needs to
adjust based on user's preferences, changes in application policy/settings and
by measuring the user's Quality of Experience (QoE). In this thesis, we
discuss the requirements for congestion control, define a framework and
categorize congestion control cues and lastly, propose congestion control
algorithms that work in diverse situations.

% video codecs routinely produce an I-frame (a full frame compressed picture)
% due to high motion (zooming, panning or the captured view itself contains
% motion) which causes a temporary burst on the network because these I-frames
% may be an order of magnitude larger than the other video packets (e.g.,
% P-frames are dependent frames).


We use Real-time Transport Protocol (RTP)~\cite{rfc3550} to carry media flows,
the congestion control algorithm is therefore built within the design
constraints of RTP. Consequently, wherever possible we have attempted to fix
these deficiencies by proposing extensions to RTP. This thesis is a bundle of
scientific papers that discuss various parts of the framework and this summary
puts them in context.

% classification of cues 
% is a summary of papers

\section{Multimedia Congestion Control}
% \label{rg.ch.dscp}

\subsection{Challenges with DSCP Markings}
% \label{rg.ch.dscp}

DiffServ assigns each data packet to a traffic class on a hop-by-hop basis and
the routers manage each traffic class differently, thereby some traffic
classes receive preferential treatment (e.g., lower delay, lower losses) in
the network~\cite{rfc2475}. The routers overcome congestion between traffic
classes by implementing \emph{priority queuing}, \emph{fair queuing}, or
\emph{weighted fair queuing (WFQ)}~\cite{rfc4594}; for congestion within the
same traffic class the router discards packets using \emph{tail drop} or
\emph{Random Early Detection (RED)}~\cite{Floyd:RED}.

% generic markings

Consequently, DiffServ needs to be implemented on every router along the data
path and configured to have the same forwarding policy, i.e., the routers have
to belong to the same DiffServ administrative domain for the packets to be
treated in exactly the same way at each hop. However, if the packets traverse
DiffServ domains, normally between Internet Service Providers (ISPs), it is
quite possible that the receiving ISP does not implement a policy for each
corresponding traffic class; when this happens the routers use the default
policy to forward packets, and as a result lose any opportunity of
differentiated service. Especially with video traffic which can be marked in
varying ways depending on the type of application. Multicast, broadcast,
streaming, conversational, each category has its own policy, the ISP sometimes
choose to ignore these different categories for video and marks all (or a
subset of) the video packets with the same DSCP~\cite{rfc5865}. These
\emph{generic} markings may be contrary to the intended DSCP of the multimedia
system. Again DSCP does not guarantee the intended behavior for the endpoint.
Despite the above listed challenges, marking packets with a DSCP code point
helps in some environments~\cite{draft.rtcweb.qos}:

% cases where it might work

\begin{itemize}

  \item If the bottleneck link is the broadband router that often connects a
  residential or Small Office/Home Office (SOHO) to the ISP, the DSCP markings
  helps in prioritizing the data traffic at this bottleneck.

  \item If the packets traverse a congested wireless link and the service
  provider did not scrub out the DSCP markings, the markings may help.

\end{itemize}

\section{Research Methodology}

This thesis aimed to produce original scientific work that would be widely
applicable in the Internet community. Based on the cultural styles defined by
ACM~\cite{Denning:CS.Method}, we use the \textit{abstraction} and
\textit{design} paradigms to conduct our scientific research. To elaborate,
the results that make up the core of the thesis were implemented as
simulations, proof-of-concept prototypes and in test-beds. Consequently, this
process helped us make better design and implementation decisions.

In order to make significant impact in the Internet community, researchers not
only have to produce significant results to motivate deployment but also solve
engineering issues. These engineering solutions may not fulfill the
requirements of being described in scientific papers, but are instead
described in standards documents, which facilitates interoperability and
enthuses deployment. For example, the protocol extensions to enable congestion
control (i.e., to signal the congestion cues) are discussed mainly in the
standards document, while the associated congestion control algorithm and the
performance analysis is discussed in the scientific papers. To summarize, in
this thesis, we discuss both our research work and our engineering work, but
emphasize more on the former.

\section{Contribution to Knowledge}

The following are the main contributions to knowledge of this dissertation:

\begin{itemize}
\setlength{\itemsep}{0pt}

% \item A criteria to evaluate multimedia congestion control algorithms in
% diverse usage scenarios and network topologies. These standardized scenarios
% form the basis of the performance evaluation in all our papers and improves
% the process of comparing our proposed solutions.

\item A mechanism to implement a rudimentary congestion control (circuit-
breaker) that aborts communication when it encounters congestion. By
implementing such a mechanism the endpoints limit the impact of a non-adaptive
media flow on other elastic traffic.

\item A study on implementing the congestion controller for real-time media
communication at the sender, receiver, or both. Additionally, we look at the
possibility of reacting to congestion cues sent by the network elements on the
media path. We also evaluate the performance of the congestion control
currently implemented by the Chrome Web-browser.

\item Applicability of an error-resilience scheme from a suite of
error-resilience mechanisms based on latency and loss rate. Consequently, we
also propose using Forward Error Correction (FEC) to perform congestion
control instead of just using it for error-resilience.

\item A mechanism to use multiple interfaces to send and receive real-time
multimedia. We also propose a scheduling and an adaptive playout algorithm
that takes into account the variability in path characteristics across diverse
paths.

\item A mechanism to create coverage maps, i.e., associate throughput to a
geo-location. So that endpoints detect areas of good and poor coverage, and
when passing through these areas adapt their sending rate to best fit the
network conditions.

\end{itemize}

% \vspace{-1cm}
\section{Summary of the Publications}

This thesis consists of an introductory part and eight original publications.
In~\citepub{c:cb}, we propose a set of circuit-breaker conditions which are
applied to non-adaptive media flows. At the moment, these media flows do not
implement congestion control and if deployed on the Internet, are expected to
cause congestion.

\citepub{c:3grc}, \citepub{c:hetrc}, \citepub{c:eval}, and \citepub{c:fecrc}
discuss about congestion control for interactive multimedia communication.
\citepub{c:3grc} performs congestion control in a wireless environment. In
this paper, we introduce three algorithms for implementing congestion control:
sender-driven, the sender decides the new sending rate; receiver-driven, the
receiver decides the new sending rate; and network-assisted, the network
notifies the endpoints about the available rate (for e.g., by an in-path
device, a 3G base station). \citepub{c:hetrc} extends the sender-driven
algorithm in \citepub{c:3grc} for deployment in heterogeneous environments.
\citepub{c:eval} evaluates the performance of Google's congestion control
algorithm proposed for WebRTC and remarks on the deployability on the
Internet.

\citepub{c:err} discusses about error resilience for interactive multimedia
communication in a mobile (3G) environment. In this paper, we experiment with
using different types of error resilience schemes, namely, Negative
Acknowledgment (NACK) or Packet Loss Indication (PLI), Forward Error
Correction (FEC) or Unequal Level of Protection (ULP), adaptive video slice
sizes, and Reference Pictures Selection Indication (RPSI). Lastly it
discusses, the applicability of these schemes based on observed packet loss
ratio and delay.

In \citepub{c:fecrc}, we propose unifying the concept of error-resilience and
congestion control. This new congestion control algorithm uses FEC to probe
for available capacity and would replace the two separate algorithms currently
implemented by existing interactive multimedia applications (e.g., Skype,
Google Hangouts, FaceTime).


In \citepub{c:mprtp}, we enable multihoming for real-time flows and extend the
current RTP system to be able to send media over multiple paths. In this
paper, we propose a scheduling algorithm for Multipath RTP (MPRTP) that sends
media over paths with widely different path characteristics and also propose a
de-jitter buffer algorithm that plays out packets smoothly when the path skew
is large. The paper also discuss system and implementation related issues
involved in implementing MPRTP.

In \citepub{c:glass}, we propose a system to enable network-assisted
congestion control for mobile clients by building a network coverage maps
(mainly, measuring throughput). This paper builds on the initial results
presented in \citepub{c:3grc}, where the middleboxes in the media path assist
in congestion control. However, in \citepub{c:glass}, mobile clients report
their media throughput and geolocation to a \emph{coverage map service}, which
collects, stores and summarizes this information per geo-location. The mobile
clients query the coverage map service for available capacity at future
geo-locations and make appropriate congestion control decisions.

\section{Structure of the Thesis}

This thesis describes techniques to modify the sending rate to changing
network characteristics for different types of multimedia systems. The work is
mainly a summary of scientific-papers, but is also supported by additional
body of work. We have co-authored a number of Internet Drafts\footnote{at the
time of writing this thesis, several of these documents are still in the
Internet Draft state, but will be published as RFCs shortly} that complement
the scientific results discussed in the thesis. The chapters describing the
various parts of the congestion control framework discuss both our scientific
and engineering work, while associating it with the relevant related work in
the area. The remainder of the thesis is organized as follows.

% Chapter~\ref{chap:rg} describes the research goal of the thesis, which
% consists of discussing terminology, challenges with DSCP markings,
% requirements for congestion control. This chapter is based on our contribution
% \cite{Singh:control.loops.api} and~\cite{draft.rmcat.evaluate}.

%\chapter{RTP: Real-time Transport Protocol}
% RTP, AVPF, CCM, XR, reduced-size, 

Chapter~\ref{chap:rtp} provides the necessary background information to Real-
time Transport Protocol (RTP). RTP together with RTP Control Protocol (RTCP)
forms the control loop that adapts media to the reported path characteristics.
This chapter is based on the RTP protocol suite~\cite{rfc3550, rfc4585,
rfc3611, rfc5104, rfc5506} and our contributions to
it~\cite{rfc7097, rfc7005, draft.xr.bytes.discarded}.

% \chapter{Rate-control Framework}

Chapter~\ref{chap:cc.fw} provides an high-level overview to our proposed
`Congestion Cues Framework', discusses congestion cues, options for reporting
intervals, criteria for evaluating congestion control. We also discuss the
circuit breaker--a minimal congestion control--conditions under which a
multimedia stream will be terminated. The circuit breaker is applicable to
applications that do not currently implement congestion control, are about to
be deployed on the wide Internet and do not want to cause a congestion
collapse. This chapter is based on our contributions, which is documented
in~\cite{Singh:control.loops.api, draft.rmcat.app.interaction,
draft.rmcat.evaluate, Singh:PhDFw, draft.rtp.cb}, \citepub{c:cb}.


% \chapter{Rate-control for Interactive Multimedia}
% draft.rtp.cb, draft.rtp.tfrc, draft.rrtcc

Chapter~\ref{chap:cc} discusses the mechanisms available for congestion
control in interactive multimedia. We consider sender-driven, receiver-driven
and co-operative congestion control algorithms. The chapter is based on our
contributions, which is documented in \citepub{c:3grc}, \citepub{c:hetrc},
\cite{singh:2010.thesis} and \citepub{c:eval}.

% \chapter{Adaptive Error-Resilience and Congestion Control}
% +ECN

Chapter~\ref{chap:er-cc} discusses the applicability of error-resilience
mechanisms for real-time communication. We also discuss using these
error-resilience techniques for congestion control. The chapter is based on
our contributions, which is documented in \citepub{c:err}, and
\citepub{c:fecrc}.

% \chapter{Multihoming, Overlay and Mobility Consideration}

Chapter~\ref{chap:mprtp} discusses using multihoming for real-time media
delivery and introduces Multipath RTP (MPRTP). The chapter is mainly based on
our contributions, which is documented in~\cite{draft.mprtp, draft.mprtp.sdp,
Globisch:AsymGrpComm, draft.rtcp.overlay}, and \citepub{c:mprtp}.


% \chapter{Network-assisted Congestion Control}

Chapter~\ref{chap:cc.nw} discusses network-assisted congestion cues, i.e.,
from middleboxes in the media path or from a service providing a map of
network coverage (collected via active or passive measurements). The chapter
is based on our contributions, which is documented in \citepub{c:3grc},
\citepub{c:glass} and \cite{glass:patent}.

% \chapter{Conclusions}

Chapter~\ref{chap:conc} concludes the thesis and we analyze the proposed
congestion control algorithms with the areas in the framework.
