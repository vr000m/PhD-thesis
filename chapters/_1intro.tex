In recent years, video has emerged as the dominant traffic\footnote{In 2012,
$51\%$ of mobile traffic was video.} on the Internet~\cite{cvni.13,dawn.zb},
partly due to the success of YouTube and other over-the-top media streaming
services (e.g., Netflix, Vimeo, Dailymotion, etc.). Video streaming emerged as
the dominant traffic only after it was easily accessible to Internet users in
the web-browser (initially, attributed to the availability of Adobe Flash
Video plugin and later, due to the inclusion of the HTML5 video tag).
Currently, the same trend can be observed with web-based conversational
multimedia, i.e., web services can use Adobe's Real-time Media Flow Protocol
(RTMFP)~\cite{draft.rtmfp}\footnote{There are other plugin-based services:
Facebook Video, Skype, Google Talk/Hangout, etc.} or HTML5's Web-based
Real-Time Communication (WebRTC) stack~\cite{draft.webrtc}.


Real-time communication on the Internet is subject to the unpredictability of
the best-effort IP network. This uncertainty is mainly due to packet loss,
packet re-ordering, variable queuing delay. Additionally,
Buffer-bloat~\cite{gettys:bufferbloat} and drop-tail queues in the router may
cause long delays and bursty losses. Video streaming overcomes most of these
challenges by having larger playout buffers, but some over-the-top streaming
applications now use TCP instead of UDP for delivering media data.
Unfortunately, TCP is only suitable for interactive multimedia for paths with
very low end-to-end delay ($<100$ms)~\cite{Brosh:tcp-real-time}. Furthermore,
in video streaming, if the path characteristics change, the client can switch
between files with the same media content pre-encoded at different quality
levels. However, this strategy is not applicable to interactive multimedia,
where the sending endpoint needs to immediately adapt the media encoding rate
to the available path capacity.

% While video communication has existed for the last decade (via Skype, etc.),
% there is no standardized congestion-control, but many have been proposed in
% the past. To tackle congestion control, the IETF has chartered a new working
% group, RMCAT\footnote{http://tools.ietf.org/wg/rmcat/} that aims at
% standardizing congestion-control for real-time communication, but it is
% expected to be a multi-year process\cite{jennings:2013:webrtc}.

%Compared to video streaming, video communication is more challenging. First,
%it requires implementing a NAT traversal mechanism for communicating between
%peers that may be behind a NATs or firewalls. Second, it requires agreeing on
%a set of common codecs, protocols and formats to avoid negotiation failure.
%Lastly, to provide  a decent end-user experience, it requires either the end-
%point or a classifier to associate a DiffServ Code Point (DSCP) to the media
%packets; therby, enabling Quality of Service (QoS). Additionally, it should be
%able to adapt the media quality to the end-to-end path characteristics.

This thesis is about enabling congestion control for real-time communication.
The idea was to build congestion control algorithms that takes into account
the network conditions, media codec capabilities and the demands of the
multimedia application. Additionally, the congestion control can adjust to the
user's preferences, changes in application policy/settings and by monitoring
the user's Quality of Experience (QoE). In this thesis, we discuss the
requirements for congestion control, define a framework and categorize
congestion control cues.


We use Real-time Transport Protocol (RTP)~\cite{rfc3550} to carry media data,
the congestion control algorithm is therefore built within the design
constraints of RTP. Consequently, wherever possible we have attempted to fix
these deficiencies by proposing extensions to RTP. This thesis is a bundle of
scientific papers that discuss various parts of the framework and this summary
puts them in context.

% classification of cues 
% is a summary of papers

\section{Research Methodology}

This thesis aimed to produce original scientific work that would be widely
applicable in the Internet community. Based on the cultural styles defined by
ACM~\cite{Denning:CS.Method}, we use the \textit{abstraction} and
\textit{design} paradigms to conduct our scientific research. To elaborate,
the results that make up the core of the thesis were implemented as
simulations, proof-of-concept prototypes and in test-beds. Consequently, this
process helped us make better design and implementation decisions.

In order to make significant impact in the Internet community, researchers not
only have to produce significant results to motivate deployment but also solve
engineering issues. These engineering solutions may not fulfill the
requirements of being described in scientific papers, but are instead
described in standards documents, which facilitates interoperability and
enthuses deployment. For example, the protocol extensions to enable congestion
control (i.e., to signal the congestion cues) are discussed mainly in the
standards document, while the associated congestion control algorithm and the
performance analysis is discussed in the scientific papers. To summarize, in
this thesis, we discuss both our research work and our engineering work, but
emphasize more on the former.



\section{Structure of the Thesis}

This thesis describes techniques to adapt media data to changing network
characteristics for different types of multimedia systems. The work is mainly
a summary of scientific-papers, but is also supported by additional body of
work. We have co-authored a number of Internet Drafts\footnote{at the time of
writing this thesis, several of these documents are still in the Internet
Draft state, but will be published as RFCs shortly} that complement the
scientific results discussed in the thesis. The chapters describing the
various parts of the congestion control framework discuss both our scientific
and engineering work, while associating it with the relevant related work in
the area. The remainder of the thesis is organized as follows.

Chapter~\ref{chap:rg} describes the research goal of the thesis, which
consists of creating a framework for congestion control that meets the
requirements for multimedia systems. This section is based on
\cite{draft.rmcat.req} and our contribution \cite{Singh:control.loops.api}.

%\chapter{RTP: Real-time Transport Protocol}
% RTP, AVPF, CCM, XR, reduced-size, 

Chapter~\ref{chap:rtp} provides the necessary background information to Real-
time Transport Protocol (RTP). RTP together with RTP Control Protocol (RTCP)
forms the control loop that adapts media to the reported path characteristics.
This section is based on the RTP protocol suite~\cite{rfc3550, rfc4585,
rfc3611, rfc5104, rfc5506} and our contributions to
it~\cite{draft.xr.discard.rle, draft.xr.jb, draft.xr.bytes.discarded}.

% \chapter{Rate-control Framework}

Chapter~\ref{chap:cc.fw} provide an high-level overview to our proposed
`Congestion Cues Framework' and discuss criteria for evaluating congestion
control for multimedia systems. We also discuss the circuit breaker--a minimal
congestion control--conditions under which a multimedia stream will be
terminated. The circuit breaker is applicable to applications that do not
currently implement congestion control, are about to be deployed on the wide
Internet and do not want to cause a congestion collapse. This section is based
on our contributions, which is documented in~\cite{draft.rmcat.evaluate,
Singh:PhDFw, draft.rtp.cb}, \citepub{c:cb}.


% \chapter{Rate-control for Interactive Multimedia}
% draft.rtp.cb, draft.rtp.tfrc, draft.rrtcc

Chapter~\ref{chap:cc} discusses the mechanisms available for congestion
control in interactive multimedia. We consider sender-driven, receiver-driven
and hybrid congestion control algorithms. The chapter is based on our
contributions, which is documented in \citepub{c:3grc}, \citepub{c:hetrc},
\cite{singh:2010:thesis} and \citepub{c:eval}.

% \chapter{Adaptive Error-Resilience and Congestion Control}
% +ECN

Chapter~\ref{chap:er-cc} discusses the mechanisms available for
error-resilience of interactive video. We also discuss using these
error-resilience techniques for congestion control. The chapter is based on
our contributions, which is documented in \citepub{c:err}, and
\citepub{c:fecrc}.


% \chapter{Network-assisted Congestion Control}

Chapter~\ref{chap:cc.nw} discusses using congestion cues from coverage maps,
i.e., overlay coverage or throughput information (collected by active/passive
measurements) on a geographical map. The chapter is based on our
contributions, which is documented in \citepub{c:glass}.

% \chapter{Multihoming, Overlay and Mobility Consideration}

Chapter~\ref{chap:mprtp} discusses using multihoming for real-time media
delivery and introduces Multipath RTP (MPRTP). The chapter is mainly based on
our contributions, which is documented in~\cite{draft.mprtp, draft.mprtp.sdp,
Globisch:AsymGrpComm, draft.rtcp.overlay}, and \citepub{c:mprtp}.

% \chapter{Conclusions}

Chapter~\ref{chap:conc} concludes the thesis and we analyze if the framework
meets the requirements.
