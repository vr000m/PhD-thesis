In recent years, video has emerged as the dominant traffic\footnote{In 2012,
51\,\% of mobile traffic was video.} on the Internet~\cite{cvni.13,dawn.zb},
partly due to the success of YouTube and other over-the-top media streaming
services (e.g., Netflix, Vimeo, Dailymotion, etc.). Video streaming emerged as
the dominant traffic only after it was easily accessible to Internet users in
the web-browser. The initial growth for video streaming is attributed to the
availability of Adobe Flash Video plugin and it became ubiquitous with the
inclusion of the video tag in HTML5. Currently, the same trend is apparent in
web-based real-time multimedia communication; web services either use Adobe's
Real-time Media Flow Protocol (RTMFP)~\cite{draft.rtmfp}\footnote{There are
other plugin-based services: Facebook Video based on Skype SDK, Google Talk,
Google's Hangout/Helpout services, etc.} or HTML5's Web-based Real-Time
Communication (WebRTC) stack~\cite{draft.webrtc}.


Real-time multimedia communication on the Internet is subject to the
unpredictability of the best-effort IP network. This uncertainty is mainly due
to packet loss, packet re-ordering, variable queuing delay. Additionally,
Buffer-bloat~\cite{gettys:bufferbloat} and drop-tail queues in the router may
cause long delays and bursty losses. Video streaming overcomes most of these
challenges by having larger playout buffers, this behavior is independent of
streaming over UDP (e.g., RTSP or SIP) or TCP (e.g., DASH). However, real-time
communication uses very small buffers and hence, endpoints modify the sending
rate to best match the available capacity and not cause excessive delay. For
real-time communication, TCP is at best suitable for paths with very low path
latencies (<100\,\emph{ms})~\cite{Brosh:tcp-real-time}, hence, UDP carries
real-time media flows.

% Furthermore, in video streaming, when the path characteristics change, the
% client switches between files with the same media content pre-encoded at
% different quality levels (\emph{rate-switching}). However, this strategy is
% not applicable to real-time communication, where the sending endpoint needs 
% to immediately adapt the media encoding rate to the available path capacity.

% While video communication has existed for the last decade (via Skype, etc.),
% there is no standardized congestion-control, but many have been proposed in
% the past. To tackle congestion control, the IETF has chartered a new working
% group, RMCAT\footnote{http://tools.ietf.org/wg/rmcat/} that aims at
% standardizing congestion-control for real-time communication, but it is
% expected to be a multi-year process\cite{jennings:2013:webrtc}.

%Compared to video streaming, video communication is more challenging. First,
%it requires implementing a NAT traversal mechanism for communicating between
%peers that may be behind a NATs or firewalls. Second, it requires agreeing on
%a set of common codecs, protocols and formats to avoid negotiation failure.
%Lastly, to provide  a decent end-user experience, it requires either the end-
%point or a classifier to associate a DiffServ Code Point (DSCP) to the media
%packets; therby, enabling Quality of Service (QoS). Additionally, it should be
%able to adapt the media quality to the end-to-end path characteristics.

This thesis is about enabling congestion control for real-time communication;
any congestion control also faces further challenges such as, variability in
the captured motion, variability in the size of the frames produced by the
media codec (I or P frames), the responsiveness of the codec to produce the
media stream for a new bit rate. Additionally, the congestion control needs to
adjust based on user's preferences, changes in application policy/settings and
by measuring the user's Quality of Experience (QoE). In this thesis, we
discuss the requirements for congestion control, define a framework and
categorize congestion control cues and lastly, propose congestion control
algorithms that work in diverse situations.

% video codecs routinely produce an I-frame (a full frame compressed picture)
% due to high motion (zooming, panning or the captured view itself contains
% motion) which causes a temporary burst on the network because these I-frames
% may be an order of magnitude larger than the other video packets (e.g.,
% P-frames are dependent frames).


We use Real-time Transport Protocol (RTP)~\cite{rfc3550} to carry media flows,
the congestion control algorithm is therefore built within the design
constraints of RTP. Consequently, wherever possible we have attempted to fix
these deficiencies by proposing extensions to RTP. This thesis is a bundle of
scientific papers that discuss various parts of the framework and this summary
puts them in context.

% classification of cues 
% is a summary of papers

\section{Research Methodology}

This thesis aimed to produce original scientific work that would be widely
applicable in the Internet community. Based on the cultural styles defined by
ACM~\cite{Denning:CS.Method}, we use the \textit{abstraction} and
\textit{design} paradigms to conduct our scientific research. To elaborate,
the results that make up the core of the thesis were implemented as
simulations, proof-of-concept prototypes and in test-beds. Consequently, this
process helped us make better design and implementation decisions.

In order to make significant impact in the Internet community, researchers not
only have to produce significant results to motivate deployment but also solve
engineering issues. These engineering solutions may not fulfill the
requirements of being described in scientific papers, but are instead
described in standards documents, which facilitates interoperability and
enthuses deployment. For example, the protocol extensions to enable congestion
control (i.e., to signal the congestion cues) are discussed mainly in the
standards document, while the associated congestion control algorithm and the
performance analysis is discussed in the scientific papers. To summarize, in
this thesis, we discuss both our research work and our engineering work, but
emphasize more on the former.



\section{Structure of the Thesis}

This thesis describes techniques to modify the sending rate to changing
network characteristics for different types of multimedia systems. The work is
mainly a summary of scientific-papers, but is also supported by additional
body of work. We have co-authored a number of Internet Drafts\footnote{at the
time of writing this thesis, several of these documents are still in the
Internet Draft state, but will be published as RFCs shortly} that complement
the scientific results discussed in the thesis. The chapters describing the
various parts of the congestion control framework discuss both our scientific
and engineering work, while associating it with the relevant related work in
the area. The remainder of the thesis is organized as follows.

Chapter~\ref{chap:rg} describes the research goal of the thesis, which
consists of discussing terminology, challenges with DSCP markings,
requirements for congestion control. This chapter is based on our contribution
\cite{Singh:control.loops.api} and~\cite{draft.rmcat.evaluate}.

%\chapter{RTP: Real-time Transport Protocol}
% RTP, AVPF, CCM, XR, reduced-size, 

Chapter~\ref{chap:rtp} provides the necessary background information to Real-
time Transport Protocol (RTP). RTP together with RTP Control Protocol (RTCP)
forms the control loop that adapts media to the reported path characteristics.
This chapter is based on the RTP protocol suite~\cite{rfc3550, rfc4585,
rfc3611, rfc5104, rfc5506} and our contributions to
it~\cite{rfc7097, rfc7005, draft.xr.bytes.discarded}.

% \chapter{Rate-control Framework}

Chapter~\ref{chap:cc.fw} provides an high-level overview to our proposed
`Congestion Cues Framework', discusses congestion cues, options for reporting
intervals, criteria for evaluating congestion control. We also discuss the
circuit breaker--a minimal congestion control--conditions under which a
multimedia stream will be terminated. The circuit breaker is applicable to
applications that do not currently implement congestion control, are about to
be deployed on the wide Internet and do not want to cause a congestion
collapse. This chapter is based on our contributions, which is documented
in~\cite{draft.rmcat.evaluate, Singh:PhDFw, draft.rtp.cb}, \citepub{c:cb}.


% \chapter{Rate-control for Interactive Multimedia}
% draft.rtp.cb, draft.rtp.tfrc, draft.rrtcc

Chapter~\ref{chap:cc} discusses the mechanisms available for congestion
control in interactive multimedia. We consider sender-driven, receiver-driven
and co-operative congestion control algorithms. The chapter is based on our
contributions, which is documented in \citepub{c:3grc}, \citepub{c:hetrc},
\cite{singh:2010.thesis} and \citepub{c:eval}.

% \chapter{Adaptive Error-Resilience and Congestion Control}
% +ECN

Chapter~\ref{chap:er-cc} discusses the applicability of error-resilience
mechanisms for real-time communication. We also discuss using these
error-resilience techniques for congestion control. The chapter is based on
our contributions, which is documented in \citepub{c:err}, and
\citepub{c:fecrc}.

% \chapter{Multihoming, Overlay and Mobility Consideration}

Chapter~\ref{chap:mprtp} discusses using multihoming for real-time media
delivery and introduces Multipath RTP (MPRTP). The chapter is mainly based on
our contributions, which is documented in~\cite{draft.mprtp, draft.mprtp.sdp,
Globisch:AsymGrpComm, draft.rtcp.overlay}, and \citepub{c:mprtp}.


% \chapter{Network-assisted Congestion Control}

Chapter~\ref{chap:cc.nw} discusses network-assisted congestion cues, i.e.,
from middleboxes in the media path or from a service providing a map of
network coverage (collected via active or passive measurements). The chapter
is based on our contributions, which is documented in \citepub{c:3grc},
\citepub{c:glass} and \cite{glass:patent}.

% \chapter{Conclusions}

Chapter~\ref{chap:conc} concludes the thesis and we analyze the proposed
congestion control algorithms with the areas in the framework.
