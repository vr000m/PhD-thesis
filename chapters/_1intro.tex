Video ...


\section{Research Methodology}

This thesis aimed to produce original scientific work that would be widely
applicable in the Internet community. The results that make up the core of the
thesis are documented in the scientific papers and were implemented as
simulations, proof-of-concept prototypes and in test-beds (in that order). Our
ideas evolved and became more refined by implementing them on a real system.
Consequently, this helped us make better decisions, based on preliminary
performance measurements.

In order to make significant impact in the Internet community, we not only
have to produce significant results to motivate deployment but also solve
engineering issues and also document them as standards documents. By
documenting these solutions, we create a more robust standard that facilitates
interoperability and therefore deployment. For example, the protocol
extensions to enable congestion control (i.e., to signal the congestion cues)
are discussed mainly in the standards document, while the associated
congestion control algorithm and the performance analysis is discussed in the
scientifc papers. To summarize, we discuss both our research work and our 
engineering work; however in thesis, we emphasize more on the former.



\section{Structure of the Thesis}

This thesis describes techniques to adapt media to changing network
characteristics for different types of multimedia systems. The work is mainly
a summary of scientific-papers, but is also supported by additional body of
work. We have co-authored a number of Internet Drafts\footnote{at the time of
writing, several of these are still in the ID state, but will be published as
RFCs (Request for Comments) shortly} that complement the  sceintific results
discussed in the thesis. The chapters describing the various parts of the
congestion control framework discuss both our scientific and engineering work,
while associating it with the relevant related work in the area. The remainder
of the thesis is organized as follows.

Chapter~\ref{chap2} discribes the research goal of the thesis, which consists
of creating a framework for congestion control that meets the requirements for
multimedia systems.

%\chapter{RTP: Real-time Transport Protocol}

Chapter~\ref{chap3} provides the neccessary background information to RTP
(Real-time Transport Protocol). RTP together with RTCP (RTP Control Protocol)
forms the control loop that adapts media to the reported path characteristics.
We also provide an high-level overview to our proposed `Congestion Cues
Framework' and discuss criteria for evaluating congestion control for
multimedia systems. This section is based on the RTP protocol
suite~\cite{rfc3550, rfc4585, rfc3611, rfc5104, rfc5506} and our contributions
to it~\cite{draft.rmcat.evaluate, draft.xr.discard.rle, draft.xr.jb}.

% RTP, AVPF, CCM, XR, reduced-size, 
% draft.rtp.cb, draft.rtp.tfrc, rrtcc
% \chapter{Rate-control for Interactive Multimedia}

Chapter~\ref{chap4} discusses the mechanisms available for congestion control
in interactive multimedia. We also discuss the circuit breaker--a minimal
congestion control--conditions under which a multimedia stream will be
terminated. The chapter is based on our contributions, which is documented
in~\cite{draft.rtp.cb, control.loops.api}, \citepub{c:cb}, \citepub{c:3grc},
\citepub{c:hetrc}.

% \chapter{Multihoming, Overlay and Mobility Consideration}

Chapter~\ref{chap5} discusses using multihoming for real-time media delivery
and introduces Multipath RTP (MPRTP). The chapter is mainly based on our
contributions, which is documented in~\cite{draft.mprtp, draft.mprtp.sdp,
globisch2011architecture, draft.rtcp.overlay}, and \citepub{c:mprtp}.

% \chapter{Network-assisted Congestion Control}

Chapter~\ref{chap6} discusses using congestion cues from coverage maps, i.e.,
overlay coverage or througput information (collected by active/passive
measurements) on a geographical map. The chapter is based on our contributions, 
which is documented in \citepub{c:glass}.

% \chapter{Adaptive Error-Resilience and Congestion Control}
% +ECN
Chapter~\ref{chap7} discusses the mechanisms available for error-resilience
of interactive video. We also discuss using these error-resilience techniques
for congestion control. The chapter is based on our contributions, which is
documented in \citepub{c:err}, and \citepub{c:fecrc}.

% \chapter{Conclusions}

Chapter~\ref{chap8} concludes the thesis and we analyse if the framework meets
the requirements.
