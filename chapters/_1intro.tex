In recent years, video has emerged as the dominant traffic\footnote{In 2012,
51\,\% of mobile traffic was video.} on the Internet~\cite{cvni.13,dawn.zb},
partly due to the success of YouTube and other over-the-top media streaming
services (e.g., Netflix, Vimeo, Dailymotion, etc.). Video streaming emerged as
the most prominent traffic on the network only after it was easily accessible
to Internet users in the web-browser. The initial growth for media streaming
is attributed to the Adobe Flash Video plugin, but it became ubiquitous when
the video tag was introduced into the HTML5 standard~\cite{html5-spec} and the
browsers natively supported rendering media streams. Currently, the same trend
is observable for real-time communication; present day web services either use
Adobe's Real-time Media Flow Protocol (RTMFP) \cite{draft.rtmfp} or write
their own plugins\footnote{There are other  plugin-based services: Facebook
Video based on Skype SDK, Google Talk, Google's Hangout/Helpout services,
etc.}. Like before with media streaming, the community is currently working
towards standardizing the Web-based Real-Time Communication (WebRTC) stack,
which will enable any web-service in need of providing real-time communication
with a few lines of code and without expecting the user to install a plugin.
Therefore, these forthcoming deployments of WebRTC services are going to kick
start the growth of real-time communication on the Internet.


There are some fundamental differences between media streaming and real-time
communication; media streaming is used in video on demand and IP television
(IPTV) services wherein the content is pre-encoded and played back directly
from storage, in streaming the challenge is to be able to simultaneously serve
multiple customers (\emph{scaling}) and consistently provide high quality
multimedia experience. The receiving endpoint in this case is mainly required
to avoid pausing the playback midstream, which it does by pre-buffering
several seconds of content. A larger pre-buffer not only removes the effect of
packet jitter but also helps in requesting lost packets (via
\emph{retransmissions}), therefore using a large playout buffer provides the
opportunity to deliver consistent video quality.
On the other hand, in real-time communication, there is no pre-stored content
and endpoints use small buffers. The small buffering duration maintains
interactivity by maintaining as low a delay as possible (limited only by the
network delay); hence, endpoints modify the sending rate to best match the
available capacity and not cause excessive delay.

This thesis is about enabling congestion control for real-time communication,
deploying it in heterogeneous networking environments containing a mixture of
wireless and wired links. Furthermore, the multimedia system running on an
endpoint may be connected to the Internet by one or more network interfaces,
having more than one IP address, thus enabling the multimedia application to
use multiple interfaces for increasing throughput or error-resilience. We
define a congestion control framework and categorize the observed congestion
cues and lastly, propose congestion control algorithms that work in diverse
situations. We use Real-time Transport Protocol (RTP)~\cite{rfc3550} to carry
media flows, the congestion control algorithm is therefore built within the
design constraints of RTP. Consequently, wherever possible we have attempted
to fix these deficiencies by proposing extensions to RTP. This thesis is a
bundle of scientific papers that discuss various parts of the framework and
this summary puts them in context.

% Video streaming endpoints overcome most of these challenges by using a larger
% playout buffers, which allows the endpoint to prebuffer the video in case
% there is an interruption in transmission, request retransmissions of lost
% packets and therefore provide consistent video quality. These features exist
% independent of streaming over UDP (e.g., RTSP or SIP) or TCP (e.g., DASH).

% Furthermore, in video streaming, when the path characteristics change, the
% client switches between files with the same media content pre-encoded at
% different quality levels (\emph{rate-switching}). However, this strategy is
% not applicable to real-time communication, where the sending endpoint needs 
% to immediately adapt the media encoding rate to the available path capacity.

% While video communication has existed for the last decade (via Skype, etc.),
% there is no standardized congestion-control, but many have been proposed in
% the past. To tackle congestion control, the IETF has chartered a new working
% group, RMCAT\footnote{http://tools.ietf.org/wg/rmcat/} that aims at
% standardizing congestion-control for real-time communication, but it is
% expected to be a multi-year process\cite{jennings:2013:webrtc}.

%Compared to video streaming, video communication is more challenging. First,
%it requires implementing a NAT traversal mechanism for communicating between
%peers that may be behind a NATs or firewalls. Second, it requires agreeing on
%a set of common codecs, protocols and formats to avoid negotiation failure.
%Lastly, to provide  a decent end-user experience, it requires either the end-
%point or a classifier to associate a DiffServ Code Point (DSCP) to the media
%packets; therby, enabling Quality of Service (QoS). Additionally, it should be
%able to adapt the media quality to the end-to-end path characteristics.

% video codecs routinely produce an I-frame (a full frame compressed picture)
% due to high motion (zooming, panning or the captured view itself contains
% motion) which causes a temporary burst on the network because these I-frames
% may be an order of magnitude larger than the other video packets (e.g.,
% P-frames are dependent frames).


% classification of cues 
% is a summary of papers

\section{Multimedia Congestion Control}
% \label{rg.ch.mcc}

Interactive real-time media applications use RTP~\cite{rfc3550} to encapsulate
multimedia content. RTP is capable of using TCP, UDP and DCCP for delivering
multimedia.  While TCP~\cite{rfc793} is extensively used to deliver video via
HTTP streaming, it is not very suitable for real-time communication;
especially when the path latencies are greater than 100 \emph{ms} \cite{Brosh:tcp-real-time}, 
hence, UDP~\cite{rfc768} carries real-time multimedia traffic. Since
UDP provides no form of congestion control, which is essential for deployment
on the Internet, multimedia applications have to implement their own congestion control.

Real-time multimedia communication on the Internet is subject to the
unpredictability of the best-effort IP network. The uncertainty is mainly due
to packet loss, packet re-ordering, and variable queuing delay.  Buffer-bloat
\cite{gettys:bufferbloat} and drop-tail queues in the router  can cause long
delays and bursty losses. The video is able to tolerate some amount of loss
either by concealing it or using some form of error-resilience technique but
bursty loss causes visual impairments which adversely affect the user's
quality of experience~\cite{Zink03subjectiveimpression}.

Additionally, in mobile networks  the capacity available to a user in a cell
varies due to mobility, cell-loading, interference, fading, and handover.
Similarly, the presence of cross-traffic on the bottleneck requires the real-time 
multimedia stream to compete for the available capacity, which may also
vary depending on the amount of cross-traffic.  Coupling the variability in
the available capacity to  the intrinsic variability in the captured media
(either due to motion or due  to voice activity detection), variability in the
size of the frames produced  by the video codec (I or P frames), the
responsiveness of the codec to produce the media stream at the requested bit
rate makes multimedia congestion control very challenging.

Instead of performing congestion control, the application may reserve capacity
for multimedia traffic, oftentimes this is done in IPTV deployments to
separate the operators' content from the customer's traffic, thereby
guaranteeing good performance for the IPTV media. Similarly, it is possible
for other multimedia services to attempt to reserve within an administrative
domain using Differentiated Services Code Points (DSCP). 

DiffServ assigns each data packet to a traffic class and the routers manage
each traffic class differently, thereby some traffic classes receive
preferential treatment (e.g., lower delay, lower losses) in the
network~\cite{rfc2475}. The routers overcome congestion between traffic
classes by implementing \emph{priority queuing}, \emph{fair queuing}, or
\emph{weighted fair queuing (WFQ)}~\cite{rfc4594}; for congestion within the
same traffic class the router discards packets using \emph{tail drop} or
\emph{Random Early Detection (RED)}~\cite{Floyd:RED}.

% generic markings

Consequently, DiffServ needs to be implemented on every router along the data
path and configured to have the same forwarding policy, i.e., the routers have
to belong to the same DiffServ administrative domain for the packets to be
treated in exactly the same way at each hop. In the case of interactive video,
it is likely that the media packets are sent to users attached not only to
different ISPs but also in different geographical locations. The packets
therefore traverse DiffServ domains, it is quite possible that the receiving
ISP to not implement a corresponding policy for each  traffic class; when this
happens the routers use the default policy to forward packets, and as a result
lose any opportunity of differentiated service.


% Especially with video traffic which can be marked in varying ways depending on
% the type of application. Multicast, broadcast, streaming, conversational, each
% category has its own policy, the ISP sometimes choose to ignore these
% different categories for video and marks all (or a subset of) the video
% packets with the same DSCP~\cite{rfc5865}. These \emph{generic} markings may
% be contrary to the intended DSCP of the multimedia system. Again DSCP does not
% guarantee the intended behavior for the endpoint. Despite the above listed
% challenges, marking packets with a DSCP code point helps in some
% environments~\cite{draft.rtcweb.qos}:

% % cases where it might work

% \begin{itemize}

%   \item If the bottleneck link is the broadband router that often connects a
%   residential or Small Office/Home Office (SOHO) to the ISP, the DSCP markings
%   helps in prioritizing the data traffic at this bottleneck.

%   \item If the packets traverse a congested wireless link and the service
%   provider did not scrub out the DSCP markings, the markings may help.

% \end{itemize}

In this thesis, we do not rely on the use of DSCP by any entity (endpoint or
middleboxes) along the path. The principal reason is that we would need to
implement congestion control anyway if there is insufficient capacity in the
service level agreement for the specific traffic class. Further, we assume
that the presence of these markings will make the performance of our
algorithms better. Consequently, the multimedia endpoints need to implement
congestion control, i.e., the endpoint monitors the media flows for congestion
and vary (increase or decrease) the media encoding rate based on the observed
congestion cues.


\section{Research Methodology}

This thesis aimed to produce original scientific work that is widely
applicable in the Internet community. The Association for Computing Machinery
(ACM) defines several cultural styles for conducting scientific
research~\cite{Denning:CS.Method}, from those our methodology falls into the
\textit{abstraction} and \textit{design} paradigms. In the \emph{abstraction}
paradigm the researcher iterates through ``modeling'' or ``experimentation''
to construct a model and make a prediction, then design an experiment and
collect data, lastly analyze the results. The \emph{design} paradigm is
related to  engineering and consists of the following steps: state the
requirements, write a specification, build and test the system. Our research
covers both these aspects, the results that make up the core of the thesis
were implemented as simulations, proof-of-concept prototypes and in test-beds. 

In order to make significant impact in the Internet community, researchers not
only have to produce significant results to motivate deployment but also solve
engineering issues. These engineering solutions are instead typically
described in standards documents, which facilitates interoperability and
enthuses deployment. In our research, wherever possible, we have contributed
back to the relevant standards body. To summarize, this thesis is made up
of both our research work and our standards work.

% For example, the protocol extensions to enable congestion control (i.e., to
% signal the congestion cues) are discussed mainly in the standards document,
% while the associated congestion control algorithm and the performance analysis
% is discussed in the corresponding scientific papers. Another example is
% Multipath RTP (MPRTP), the protocol extensions are defined in the appropriate
% standards document, but the performance of the scheduling algorithm using
% those extensions is a described in a scientific contribution.

\section{Contributions}

The following are the main contributions of this dissertation are:

\begin{itemize}
\setlength{\itemsep}{0pt}

% \item A criteria to evaluate multimedia congestion control algorithms in
% diverse usage scenarios and network topologies. These standardized scenarios
% form the basis of the performance evaluation in all our papers and improves
% the process of comparing our proposed solutions.

\item A mechanism to implement a rudimentary congestion control (circuit-breaker) 
that aborts communication when it encounters congestion. By
implementing such a mechanism the endpoints limit the impact of a non-adaptive
media flow on other elastic traffic.

\item A study on implementing the congestion controller for real-time media
communication at the sender, receiver, or both. Additionally, we look at the
possibility of reacting to congestion cues sent by the network elements on the
media path. We also evaluate the performance of the congestion control
currently implemented by the Chrome Web-browser.

\item Applicability of an error-resilience scheme from a suite of
error-resilience mechanisms based on latency and loss rate. Consequently, we
also propose using Forward Error Correction (FEC) to perform congestion
control instead of just using it for error-resilience.

\item A mechanism to use multiple interfaces to send and receive real-time
multimedia. We also propose a scheduling and an adaptive playout algorithm
that takes into account the variability in path characteristics across diverse
paths.

\item A mechanism to create coverage maps, i.e., associate throughput to a
geo-location. So that endpoints detect areas of good and poor coverage, and
when passing through these areas adapt their sending rate to best fit the
network conditions.

\end{itemize}

% \vspace{-1cm}
\section{Summary of the Publications}

This thesis consists of an introductory part and eight original publications.
In~\citepub{c:cb}, we propose a set of circuit-breaker conditions which are
applied to non-adaptive media flows. At the moment, these media flows do not
implement congestion control and if deployed on the Internet, are expected to
cause congestion. The circuit breaker triggers when the application appears to
be causing congestion.

\citepub{c:3grc}, \citepub{c:hetrc}, \citepub{c:eval}, and \citepub{c:fecrc}
discuss congestion control for interactive multimedia communication. The
congestion control algorithms proposed in \citepub{c:3grc} were developed for
a mobile environment. We additionally discuss three types of congestion
control:  \emph{sender-driven}, the sender decides the new sending rate; 
\emph{receiver-driven}, the receiver decides the new sending rate; and 
\emph{network-assisted}, the network notifies the endpoints about the available
rate (for e.g., by an in-path device, a 3G base station). \citepub{c:hetrc}
extends the sender-driven algorithm described in \citepub{c:3grc} for
deployment in heterogeneous environments. \citepub{c:eval} evaluates the
performance of Google's congestion control algorithm proposed for WebRTC and
comment on its deployability on the Internet.

\citepub{c:err} discusses error resilience for interactive multimedia
communication in a mobile (3G) environment. In this paper, we experiment with
using different types of error resilience schemes, namely, Negative
Acknowledgment (NACK) or Packet Loss Indication (PLI), Forward Error
Correction (FEC) or Unequal Level of Protection (ULP), adaptive video slice
sizes, and Reference Pictures Selection Indication (RPSI). Lastly it
discusses, the applicability of these schemes based on observed packet loss
ratio and delay.

In \citepub{c:fecrc}, we propose unifying the concept of error-resilience and
congestion control. This new congestion control algorithm uses FEC to probe
for available capacity and would replace the two separate algorithms currently
implemented by existing interactive multimedia applications (e.g., Skype,
Google Hangout, FaceTime). We compare the performance of the FEC and
Congestion Control with the algorithms discussed in \citepub{c:3grc},
\citepub{c:hetrc}, and \citepub{c:eval}. The paper also discusses the
interaction between the multimedia application, the RTP stack and the codec
for implementing congestion control.


In \citepub{c:mprtp}, we enable multihoming for real-time flows and extend the
capability of the current RTP system to send media over multiple paths. In
this paper, we propose a scheduling algorithm for Multipath RTP (MPRTP) that
sends media over paths with widely different path characteristics and also
propose a de-jitter buffer algorithm that plays out packets smoothly when the
path skew is large. The paper also discusses system and implementation related
issues involved in implementing MPRTP.

In \citepub{c:glass}, we propose a system to enable network-assisted
congestion control for mobile clients by building a network coverage maps
(mainly, measuring throughput). This paper builds on the initial results
presented in \citepub{c:3grc}, where the middleboxes in the media path assist
in congestion control. However, in \citepub{c:glass}, mobile clients report
their media throughput and geolocation to a \emph{coverage map service}, which
collects, stores and summarizes this information per geo-location. The mobile
clients query the coverage map service for available capacity at future
geo-locations and make appropriate congestion control decisions.

\section{Structure of the Thesis}

This thesis describes techniques to modify the sending rate to changing
network characteristics for different types of multimedia systems. The work is
mainly a summary of scientific-papers, but is also supported by additional
body of work. We have co-authored a number of Internet Drafts\footnote{at the
time of writing this thesis, several of these documents are still in the
Internet Draft state, but will be published as RFCs shortly.} that complement
the scientific results discussed in the thesis. The chapters describing the
various parts of the congestion control framework discuss both our scientific
and engineering work, while associating it with the relevant related work in
the area. The remainder of the thesis is organized as follows.

% Chapter~\ref{chap:rg} describes the research goal of the thesis, which
% consists of discussing terminology, challenges with DSCP markings,
% requirements for congestion control. This chapter is based on our contribution
% \cite{Singh:control.loops.api} and~\cite{draft.rmcat.evaluate}.

%\chapter{RTP: Real-time Transport Protocol}
% RTP, AVPF, CCM, XR, reduced-size, 

Chapter~\ref{chap:rtp} provides the necessary background information to 
Real-time Transport Protocol (RTP). RTP together with RTP Control Protocol (RTCP)
forms the control loop that adapts media to the reported path characteristics.
This chapter is based on the RTP protocol suite~\cite{rfc3550, rfc4585,
rfc3611, rfc5104, rfc5506} and our contributions to
it~\cite{rfc7097, rfc7005, draft.xr.bytes.discarded}.

% \chapter{Rate-control Framework}

Chapter~\ref{chap:cc.fw} provides an high-level overview to our proposed
`Congestion Cues Framework', discusses congestion cues, options for reporting
intervals, criteria for evaluating congestion control. We also discuss the
circuit breaker--a minimal congestion control--conditions under which a
multimedia stream will be terminated. The circuit breaker is applicable to
applications that do not currently implement congestion control, are about to
be deployed on the wide Internet and do not want to cause a congestion
collapse. This chapter is based on our contributions, which is documented
in~\cite{Singh:control.loops.api, draft.rmcat.app.interaction,
draft.rmcat.evaluate, Singh:PhDFw, draft.rtp.cb}, \citepub{c:cb}.


% \chapter{Rate-control for Interactive Multimedia}
% draft.rtp.cb, draft.rtp.tfrc, draft.rrtcc

Chapter~\ref{chap:cc} discusses the mechanisms available for congestion
control in interactive multimedia. We consider sender-driven, receiver-driven
and co-operative congestion control algorithms. The chapter is based on our
contributions, which is documented in \citepub{c:3grc}, \citepub{c:hetrc},
\cite{singh:2010.thesis} and \citepub{c:eval}.

% \chapter{Adaptive Error-Resilience and Congestion Control}
% +ECN

Chapter~\ref{chap:er-cc} discusses the applicability of error-resilience
mechanisms for real-time communication. We also discuss using these
error-resilience techniques for congestion control. The chapter is based on
our contributions, which is documented in \citepub{c:err}, and
\citepub{c:fecrc}.

% \chapter{Multihoming, Overlay and Mobility Consideration}

Chapter~\ref{chap:mprtp} discusses using multihoming for real-time media
delivery and introduces Multipath RTP (MPRTP). The chapter is mainly based on
our contributions, which is documented in~\cite{draft.mprtp, draft.mprtp.sdp,
Globisch:AsymGrpComm, draft.rtcp.overlay}, and \citepub{c:mprtp}.


% \chapter{Network-assisted Congestion Control}

Chapter~\ref{chap:cc.nw} discusses network-assisted congestion cues, i.e.,
from middleboxes in the media path or from a service providing a map of
network coverage (collected via active or passive measurements). The chapter
is based on our contributions, which is documented in \citepub{c:3grc},
\citepub{c:glass} and \cite{glass:patent}.

% \chapter{Conclusions}

Chapter~\ref{chap:conc} concludes the thesis and we analyze the proposed
congestion control algorithms with the areas in the framework.
