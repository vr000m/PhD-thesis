In this thesis, we have described our proposed classification of congestion
control cues (framework) in Chapter \ref{chap:cc.fw}. When discussing the
different parts of the framework, we highlighted the existing related work and
our contributions in those areas. In terms of innovativeness, the congestion
control framework provides options to look at congestion cues beyond those
reported by the receiver in an RTCP Receiver Report. The framework allows the
congestion control algorithm to use multiple paths for either aggregating
capacity or for increasing error-resilience, or to use capacity notifications from
middleboxes along the path, or to build a coverage map that provides
congestion notification as a third-party services. Each of these techniques
apply to one of the four areas described in the framework; however, the
definitions of each area are generalised enough to allow the application of a
broad range of techniques, not just those proposed in this thesis.

% insights and criticism

We defined a new class of co-operative congestion control algorithms and
believe that these algorithms will replace the purely sender-driven
mechanisms currently in use. 
The reason is that the co-operative algorithm lets the receiver
not only measure the end-to-end capacity but assess the quality of experience
(rendering quality) to decide its rate estimate. As with the 
sender-driven approach, the sender estimates the congestion based on the reported
cues and its current sending rate, but the sender can then factor in the
receiver's estimate before finally choosing a new target bit rate. We already
see an uptake of this general idea, with Google’s proposal for multimedia
congestion control in WebRTC~\cite{draft.rrtcc}.

Another observation that we made during our research was that many multimedia
systems implement the error-resilience and congestion control algorithms
separately. We believe the community has not explored the use of FEC for congestion
control in depth, partly because interactive multimedia flows have very tight
delay constraints and FEC may not arrive in time for recovering the lost
packet. The results outlined in this thesis show that FEC can be used for
congestion control and perform suitably when no bursty loss occurs.

\begin{figure}
    \centerline{
        {\includegraphics[width=\textwidth] %clip=true, trim=0 1cm 0 1.5cm]
        {chap9-fig-all-tech}}
    }
    \caption{Interworking of all the ideas presented in this thesis.}
    \label{chap9:all_in}
\end{figure}

\section{Synthesis}

While the techniques proposed in this thesis were only shown to work
independently, nothing prevents them from interworking.
Figure~\ref{chap9:all_in} depicts an example for a comprehensive architecture
wherein an endpoint will be able to use cues from all the above sources to
perform multimedia congestion control and packet scheduling. Endpoints will
always use the multipath extensions for RTP (MPRTP), even when using a single
path; this will allow the opportunity to offload or aggregate capacity when
new interfaces (or paths) appear (Block A of Figure~\ref{chap9:all_in}).

Since the circuit breaker algorithm relies on basic congestion cues (RTT and
loss) and periodic reception of RTP and RTCP, the circuit breaker sets the
boundary condition under which all future multimedia application will operate.
The congestion control will not just rely on the cues reported in an RTCP
RR/XRs, but gather hints from additional sources. For instance, mobile base
stations can help provide information whenever capacity allocation changes due
to mobility, an increase in active users, or handovers (Block B of
Figure~\ref{chap9:all_in}). Enabling Explicit Congestion Notification (ECN) in
the network and getting ECN-marked packets is another way of enabling
collaboration between the network and the endpoints. Lastly, using network
coverage maps to get information about prevailing network conditions would
also assist in congestion control (Block C of Figure~\ref{chap9:all_in}).
The main concern in this case is to ascertain the trustworthiness and
reliability of the indicated measurement values. Since the additional
congestion cues are just hints and part of a larger set of cues, an endpoint
may ignore cues that seem erroneous or provide false indication than the other
cues. 

% The recent disclosures of pervasive active monitoring has caused many services
% to take security more seriously. This has resulted in turning on encryption.
% Activating encryption means that middleboxes, which peeked into the traffic to
% provide differentiated quality of service, are now unable to do that anymore.

These additional hints are also essential at session
startup, because currently a media application typically starts sending media
at a pre-configured value (either set very low or set to the maximum media
rate). Receiving notifications from a NCMS server at the beginning of the
session may help the endpoint pick a better start rate.


Block D of Figure~\ref{chap9:all_in} shows that the receiver sends an RTCP
RR and XR containing typical congestion cues. It also sends any estimate made
by a receiver-side congestion control algorithm (e.g., using TMMBR, REMB).
Furthermore, it adds any throughput notifications it may have received from an
in-path middlebox (RTCP XR containing ECN markings or capacity changes
indicated by a base station) or from a NCMS server. The sender decides the new
target bit rate based on the congestion cues reported by the receiver, and the
notifications it has also been receiving from an in-path middlebox or from the
NCMS (Block E of Figure~\ref{chap9:all_in}).


Finally, depending on the underlying codec implementation, the new target rate
may result in: a change in the encoding rate of an audio and/or video stream,
a change in the number of layers produced by a scalable video codec (SVC),
packetization time (ptime) of an audio stream, or a change in the number of
simulcast multimedia streams (typically video streams).

\section{Future Directions}

Another aspect of MPRTP that is not discussed in detail in this thesis is the
role of MPRTP in overlay networks, or for processing media in data centers. Very
large conferences, with hundreds of participants can be arranged in complex
topologies containing combinations of cascaded meshes and trees. Such very
large media conferences (e.g., massive open online courses, seminars or
conferences), usually have a low peer churn because all participants arrive
and leave roughly at the same time. Also, the active participants produce the
media flows, while the dormant/passive participants consume it and occasionally
chime in with questions and comments. Instead of broadcasting to all the
participants, which may create load on a centralised server and require
scaling, we can exploit the asymmetric relationship between the participants
by using them as overlays. This would require participants to forward at least
as much as they receive, if not more, and using multiple paths eases this
requirement~\cite{Noh2008,Li2010a,Globisch:AsymGrpComm}.


% \section{Future Work}

% For each of the proposed congestion control techniques in this thesis, we make
% deployment recommendations and for some of these techniques the deployment
% considerations overlap. 

In the near-term, we see the following elements of emerging. First,
the emergence of WebRTC requires standardised congestion control.
Chapters~\ref{chap:cc} and~\ref{chap:er-cc} make some proposals that may fit
this purpose, but it requires more extensive evaluations to be deployment-ready.
Second, using FEC for congestion control needs to be generalised to work with
any other multimedia congestion control algorithm, to enhance the
applicability of the congestion control. Third, multipath scheduling needs to be reconsidered for
interactive multimedia communication. One way of achieving this would require
implementing a coupled congestion control that synthesises all the subflow
cues to arrive at an overall rate that would fit the combined paths and then
let the current scheduling algorithm allocate the bits per subflow. Fourth,
the scalability and robustness of the network coverage map service needs to
studied; furthermore, the location-throughput matching algorithm needs to be
studied in more detail to respond to diverse reporting from
various mobile devices, etc. Lastly, all these techniques need to be combined for
implementing a unified congestion control for interactive multimedia .



% Various conference architectures can be used to distribute the media in a
% many-to-many communication scenario: centralized, unicast receive with
% multicast send, full mesh, overlays, and trees
% \cite{Li2010a,Noh2008,Singh2001}. When developing a conferencing systems for a
% specific use-case, the scalability, reliability, quality and delay
% characteristic, for each of architecture needs to be considered. In the case
% of of very large video conferences, such as, massive open online course,
% seminars or conferences, we assume a low peer churn i.e., all participants
% arrive and leave roughly at the same time. Also, active participants produce
% the media flows, while the dormant/passive participant consume it. In
% \cite{Globisch:AsymGrpComm}, we propose using multiple Application Layer
% Multicast (ALM) distribution trees to broadcast the media from an active
% speaker to the participants listening in to the conference. The ALM tree
% network minimizes the end-to-end delay and reduces the load on the active
% participants.

% The use of multiple Application Layer Multicast (ALM) trees for media delivery
% minimizes the end-to-end delay and results in asymmetric relationships between
% participants and introduces complex forwarding. Chu \emph{et al.} show ALM as
% a viable solution for real-time conferencing over the Internet~\cite{Chu2001}.
% Banerjee et al.\cite{Banerjee2002} present an ALM protocol that has a
% hierarchical control structure with low overhead. Noh \emph{et
% al.}~\cite{Noh2008} use multiple trees to reduce the end-to-end delay and
% determine the optimal number of ALM trees depending on specific network
% characteristics. They conclude that the fan-out of a peer influences the
% trade-off between the propagation delay and the queuing delay. Li et
% al.~\cite{Li2010a} describes the use of multiple trees as a mechanism to scale
% to more clients by introducing multiple focus-mixer structures where each
% structure is dedicated to serving a set of clients in a region.

% In \cite{draft.rtcp.overlay}, instead of using a centralized conferencing
% server to maintain the media sessions and inserting peers or nodes in the
% appropriate location, we propose protocol extensions (e.g., to RTCP) to help
% nodes re-arrange themselves based on their pairwise connectivity, i.e.,
% reconstructing the tree by preferring links with better network
% characteristics.