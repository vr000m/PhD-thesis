
In this thesis, we describe our proposed classification of congestion control
cues (framework) in \ref{fw.fw}. When describing the different parts of the
framework, we also discuss the related work and our contributions in those
areas. In Chapter~\ref{chap:rg}, we aimed to classify congestion control cues
for real-time communication based on: \emph{where they are measured?} (by
in-path or out-of-path sources) and \emph{how they are reported?} (via in-band
or out-of-band signaling). In Chapter~\ref{chap:cc.fw}, we describe other
fundamental choices needed to implement congestion control. Choice of
congestion cues, reporting frequency and circuit-breakers. Additionally, we
also describe an evaluation suite for measuring the performance of the
proposed congestion control algorithms.

In chapter~\ref{chap:cc}, we describe congestion control implemented using
congestion cues from in-band sources and signaled in path (in RTCP). We
further categorize the congestion control algorithms based on \emph{where they
are implemented}: sender-driven scheme (e.g., TFRC), receiver-driven scheme
(e.g., TMMBR), or co-operative scheme (combination of sender and receiver,
e.g., C-NADU, RRTCC) and compare the performance of an algorithm in each
scheme. We observe that the performance of TFRC is bursty, which may lead to
poor user-experience, whilst TMMBR, C-NADU and RRTCC have a more stable
throughput. Lastly, TMMBR appears to be conservative and RRTCC appears to be
aggressive. C-NADU attempts to trade-off throughput for losses, but achieves
slightly higher throughput than TMMBR.


In chapter~\ref{chap:er-cc}, we address two problems: applicability of
error-resilience schemes, and using FEC to probe for available capacity to
implement congestion control for interactive multimedia flows. To the first
problem, we show that NACK, Reference Picture Selection, Unequal Error
Protection and adaptive slice-size can be used for different levels of
latencies and observed packet loss ratio. To the second problem: we propose a
congestion control scheme where instead of increasing the rate when network
conditions seem stable, the sender introduces FEC for one RTCP interval. If
the FEC and the media packets are received successfully, the sender increases
the sender rate with the amount of FEC rate. The trade-off is that we get a
smoother ramp-up and if a packet gets lost, it may be recovered by the FEC
packet. The sender also implements a variable FEC interval, i.e., it varies
the number of packet for which FEC is generated. Hence, if the sender thinks
that it is underutilizing the link by large margin, it introduces a shorter
FEC interval (up to 33\,\% redundancy) and therefore ramps up quickly.
Consequently when it thinks that it is closer to the bottleneck link capacity,
it introduces a longer FEC interval (up to 8\,\% redundancy) and therefore is
conservative in probing for available capacity. Our experiments show that by
using adaptive FEC for probing, the endpoints are able to recover 15-25\,\% of
the lost packets and $\approx$90\,\% of the time using FEC subsequently
results in an increase in the media rate. These results are comparable to our
earlier experiments using a fixed FEC interval throughput the duration of the
call for error-resilience, where we were able to 20-24\;\% of the lost
packets. We believe using FEC for congestion control in interactive multimedia
has not been explored in depth by the community, partly because interactive
multimedia flows have very tight delay constraints and FEC may not arrive in
time for recovering the packet.

In chapter~\ref{chap:mprtp}, we propose using the multiple interfaces of an
endpoint (e.g., mobile device, tablet, SOHO gateways) for increasing
throughput and robustness. It corresponds to using congestion cues from
out-of-band sources (different paths) and signaled in path (in RTCP). We
design a protocol extension (MPRTP) to RTP that is backwards compatible and
exploits multipath capabilities. We implement a scheduling algorithm that
takes application requirements and current path characteristics into
consideration to send packets over multiple paths. At the receiver, we
implement a per-path and aggregate de-jitter buffer, which interwork to
playout packets smoothly even when the path skew is high. Our experiments show
that the performance of MPRTP is not degraded compared to single path RTP, so
that it is safe to deploy. It enables load distribution and capacity
aggregation, which enables features like mobility, offloading, multihoming.

In chapter~\ref{chap:cc.nw}, we describe two congestion control mechanisms,
both use out-of-band signaling, but use in-path and out-of-path sources,
respectively. First, the in-path sources is based on receiving congestion cues
from middleboxes (e.g., 3G or LTE base-stations) and use standard RTP
extensions (e.g., TMMBR) for signaling . Second, the out-of-path sources are
crowd-sourced 3G coverage maps that collect throughput and geo-location
information from users and then notifies endpoints about the available
capacity in the region. We propose a system to collect and query coverage
maps. The endpoints query the coverage server using out-of-band signaling to
discover areas of poor coverage, based on these notifications the endpoints
vary their sending rate.

In terms of innovativeness, the congestion control framework provides options
to look beyond using congestion cues reported by the receiver on a given path.
The framework offers the congestion control algorithm to use multiple paths to
either aggregate capacity or for increasing resilience, to use feedback
notifications from smart middleboxes along the path or to build a coverage or
map that provides congestion notification as a third party services. Each of
these techniques apply to one of the four areas described in the framework.
However, the definitions of each area is generalized enough to allow
application of a broad-range of techniques not just those proposed in this
thesis.


\section{Future Work}

The algorithms described in each area of the framework currently works
independently of each-another and for each of the proposed technique, we make
deployment recommendations. However, making them interwork together deserves
further investigation.