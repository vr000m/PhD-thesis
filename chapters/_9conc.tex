
In this thesis, we have described our proposed classification of congestion
control cues (framework) in \ref{chap:cc.fw}. When discussing the different
parts of the framework, we highlight the existing related work to our
contributions in those areas. In terms of innovativeness, the congestion
control framework provides options to look beyond using congestion cues
reported by the receiver on a given path. The framework offers the congestion
control algorithm to use multiple paths to either aggregate capacity or for
increasing resilience, to use feedback notifications from intelligent
middleboxes along the path or to build a coverage or map that provides
congestion notification as a third party services. Each of these techniques
apply to one of the four areas described in the framework. However, the
definitions of each area is generalized enough to allow application of a
broad-range of techniques not just those proposed in this thesis. We believe
using FEC for congestion control in interactive multimedia has not been
explored in depth by the community, partly because interactive multimedia
flows have very tight delay constraints and FEC may not arrive in time for
recovering the lost packet.

% In chapter~\ref{chap:cc}, we describe congestion control implemented using
% congestion cues from in-band sources and signaled in path (in RTCP). We
% further categorize the congestion control algorithms based on \emph{where they
% are implemented}: sender-driven scheme (e.g., TFRC), receiver-driven scheme
% (e.g., TMMBR), or co-operative scheme (combination of sender and receiver,
% e.g., C-NADU, RRTCC) and compare the performance of an algorithm in each
% scheme. We observe that the performance of TFRC is bursty, which may lead to
% poor user-experience, whilst TMMBR, C-NADU and RRTCC have a more stable
% throughput. Lastly, TMMBR appears to be conservative and RRTCC appears to be
% aggressive. C-NADU attempts to trade-off throughput for losses, but achieves
% slightly higher throughput than TMMBR.


% In chapter~\ref{chap:er-cc}, we address two problems: applicability of
% error-resilience schemes, and using FEC to probe for available capacity to
% implement congestion control for interactive multimedia flows. To the first
% problem, we show that NACK, Reference Picture Selection, Unequal Error
% Protection and adaptive slice-size can be used for different levels of
% latencies and observed packet loss ratio. To the second problem: we propose a
% congestion control scheme where instead of increasing the rate when network
% conditions seem stable, the sender introduces FEC for one RTCP interval. If
% the FEC and the media packets are received successfully, the sender increases
% the sender rate with the amount of FEC rate. The trade-off is that we get a
% smoother ramp-up and if a packet gets lost, it may be recovered by the FEC
% packet. The sender also implements a variable FEC interval, i.e., it varies
% the number of packet for which FEC is generated. Hence, if the sender thinks
% that it is underutilizing the link by large margin, it introduces a shorter
% FEC interval (up to 33\,\% redundancy) and therefore ramps up quickly.
% Consequently when it thinks that it is closer to the bottleneck link capacity,
% it introduces a longer FEC interval (up to 8\,\% redundancy) and therefore is
% conservative in probing for available capacity. Our experiments show that by
% using adaptive FEC for probing, the endpoints are able to recover 15-25\,\% of
% the lost packets and $\approx$90\,\% of the time using FEC subsequently
% results in an increase in the media rate. These results are comparable to our
% earlier experiments using a fixed FEC interval throughput the duration of the
% call for error-resilience, where we were able to 20-24\;\% of the lost
% packets. We believe using FEC for congestion control in interactive multimedia
% has not been explored in depth by the community, partly because interactive
% multimedia flows have very tight delay constraints and FEC may not arrive in
% time for recovering the packet.

% In chapter~\ref{chap:mprtp}, we propose using the multiple interfaces of an
% endpoint (e.g., mobile device, tablet, SOHO gateways) for increasing
% throughput and robustness. It corresponds to using congestion cues from
% out-of-band sources (different paths) and signaled in path (in RTCP). We
% design a protocol extension (MPRTP) to RTP that is backwards compatible and
% exploits multipath capabilities. We implement a scheduling algorithm that
% takes application requirements and current path characteristics into
% consideration to send packets over multiple paths. At the receiver, we
% implement a per-path and aggregate de-jitter buffer, which interwork to
% playout packets smoothly even when the path skew is high. Our experiments show
% that the performance of MPRTP is not degraded compared to single path RTP, so
% that it is safe to deploy. It enables load distribution and capacity
% aggregation, which enables features like mobility, offloading, multihoming.

% In chapter~\ref{chap:cc.nw}, we describe two congestion control mechanisms,
% both use out-of-band signaling, but use in-path and out-of-path sources,
% respectively. First, the in-path sources is based on receiving congestion cues
% from middleboxes (e.g., 3G or LTE base-stations) and use standard RTP
% extensions (e.g., TMMBR) for signaling . Second, the out-of-path sources are
% crowd-sourced 3G coverage maps that collect throughput and geo-location
% information from users and then notifies endpoints about the available
% capacity in the region. We propose a system to collect and query coverage
% maps. The endpoints query the coverage server using out-of-band signaling to
% discover areas of poor coverage, based on these notifications the endpoints
% vary their sending rate.



% insights and criticism

% \section{Future Work}

For each of the proposed congestion control techniques in this thesis, we make
deployment recommendations and for some of these techniques the deployment
considerations overlap. Whereas the techniques in this thesis are only shown
to work independently and not interwork with each another. Therefore making
them interwork together deserves further investigation. For example, the
sending endpoint receiving congestion cues from a network coverage map service
and from the receiving endpoint can complement the capacity estimated by the
sender, hence, performing better congestion control.

Individually, each technique associated with the particular area of the
framework requires further work. First, emergence of WebRTC requires
standardized congestion control, chapters~\ref{chap:cc} and~\ref{chap:er-cc}
make some proposals that may fit the purpose, but require more extensive
evaluations to be deployment ready. Additionally, there are still some
proposals that are yet to be evaluated (e.g., D-FLow, NADA, etc). Second,
using FEC for congestion control needs to be generalized to work with other
congestion control algorithms. Third, multipath scheduling needs to be
reconsidered for low-delay scenarios. Fourth, the scalability and robustness
of the network coverage map service needs to studied, furthermore, the
location-throughput matching algorithm needs to also studied in more detail to
be able to respond to flash-crowds, diverse reporting from various mobile
devices, etc. Lastly, use these congestion cues for implementing congestion
control for interactive multimedia.


