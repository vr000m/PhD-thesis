In this thesis, we have described our proposed classification of congestion
control cues (framework) in Chapter \ref{chap:cc.fw}. When discussing the
different parts of the framework, we highlight the existing related work to
our contributions in those areas. In terms of innovativeness, the congestion
control framework provides options to look at congestion cues beyond those
reported by the receiver in an RTCP Receiver Report. The framework offers the
congestion control algorithm to use multiple paths for either aggregating
capacity or for increasing resilience, to use feedback notifications from
intelligent middleboxes along the path or to build a coverage map that
provides congestion notification as a third party services. Each of these
techniques apply to one of the four areas described in the framework; however,
the definitions of each area is generalized enough to allow the application of
a broad-range of techniques not just those proposed in this thesis.


We discuss and compare the performance of sender- and receiver-driven
congestion control algorithms; additionally define a hybrid class of
congestion control algorithms in which the multimedia application chooses the
new target media bit rate by comparing the bandwidth estimated by both the
sending and receiving endpoint. We believe that the hybrid algorithms are
possibly the best way forward; they provide a method to be backwards
compatible, i.e., interoperate with receivers that do not send receiver’s
estimate and in cases where the receiver co-operates the results are more
performant. We already see an uptake of this idea, with Google’s proposal for
congestion control for WebRTC.

Another observation that we made early on in our research was that many
multimedia systems implement the error-resilience and congestion control
algorithms independently. Furthermore, we believe the community has not
explored using FEC for congestion control in interactive multimedia in depth,
partly because interactive multimedia flows have very tight delay constraints
and FEC may not arrive in time for recovering the lost packet. The results
outlined in this thesis show that FEC can be used for congestion control and
perform suitably when no bursty loss occurs. We also observed that sending
smaller packets in mobile networks performed better than sending MTU sized
($\approx$ 1500 bytes) packets. Alternatively, the MTU sized packets were
better suited for wired or fixed networks where bit errors occurred less
often.


Another idea presented in the thesis is that the network provides notification
to endpoints to perform congestion control. We show that notification from
middleboxes such as, by base stations would help endpoints better congestion
control for example in mobile networks, for example when an LTE cell is
experiencing extreme load\footnote{Extreme load occurs when an LTE cell has
about 10x more users than when it is busy.}. Furthermore, we also show that
using congestion maps also aids in performing congestion control. Congestion
maps is a measurement service that aggregates throughput information from
multiple users and sends notification of areas with poor coverage to its
subscribers. We expect these notifications to be used as congestion cues and
not as a replacement to in-band, in-path congestion control. The main reason
for restricting the use of such a service to notify congestion cues is that
such a service is vulnerable to data pollution, i.e., the clients may report
incorrect measurements, not necessarily intentionally but because of
programming errors. Another reason is that the reported notifications may
depend on the way the aggregation is performed and how quickly the aggregation
converges to the prevailing network conditions in a region. A fast convergence
may be susceptible to misreporting leading to false positives and make the
user experience fluctuate when in reality there would be no reason for it to
fluctuate. A slow convergence would ignore minor spikes in load and endpoints
would experience poor quality of experience for a brief period until the
values converge. A slow convergence may also lead to inefficiency because once
an aggregation value in a region drops; it would stay there unless the
endpoints probe for additional capacity.


The advent of multi-view multi-party telepresence systems, these systems
demand more capacity than the current networks can currently provide and
instead of waiting for the capacity on a single path to increase, we developed
MPRTP; it is combines capacity across multiple paths. A common challenge with
using multiple paths is to reconcile the differences in the path latencies or
RTTs. A multimedia system typically contains a dejitter buffer that
reassembles out-of-order packets that can be used for this purpose but the
performance depends on the type of multimedia application and the maximum size
of the dejitter buffer. Consequently, in some cases the high latency path may
need to be ignored because it may not meet the requirements of the application
and therefore the capacity from that path would not be aggregated. Our
experiments showed that capacity could be aggregated across several mobile
interfaces, or mobile and WLAN, or multi-homed home gateway connected to
several ISPs.



% In chapter~\ref{chap:cc}, we describe congestion control implemented using
% congestion cues from in-band sources and signaled in path (in RTCP). We
% further categorize the congestion control algorithms based on \emph{where they
% are implemented}: sender-driven scheme (e.g., TFRC), receiver-driven scheme
% (e.g., TMMBR), or co-operative scheme (combination of sender and receiver,
% e.g., C-NADU, RRTCC) and compare the performance of an algorithm in each
% scheme. We observe that the performance of TFRC is bursty, which may lead to
% poor user-experience, whilst TMMBR, C-NADU and RRTCC have a more stable
% throughput. Lastly, TMMBR appears to be conservative and RRTCC appears to be
% aggressive. C-NADU attempts to trade-off throughput for losses, but achieves
% slightly higher throughput than TMMBR.


% In chapter~\ref{chap:er-cc}, we address two problems: applicability of
% error-resilience schemes, and using FEC to probe for available capacity to
% implement congestion control for interactive multimedia flows. To the first
% problem, we show that NACK, Reference Picture Selection, Unequal Error
% Protection and adaptive slice-size can be used for different levels of
% latencies and observed packet loss ratio. To the second problem: we propose a
% congestion control scheme where instead of increasing the rate when network
% conditions seem stable, the sender introduces FEC for one RTCP interval. If
% the FEC and the media packets are received successfully, the sender increases
% the sender rate with the amount of FEC rate. The trade-off is that we get a
% smoother ramp-up and if a packet gets lost, it may be recovered by the FEC
% packet. The sender also implements a variable FEC interval, i.e., it varies
% the number of packet for which FEC is generated. Hence, if the sender thinks
% that it is underutilizing the link by large margin, it introduces a shorter
% FEC interval (up to 33\,\% redundancy) and therefore ramps up quickly.
% Consequently when it thinks that it is closer to the bottleneck link capacity,
% it introduces a longer FEC interval (up to 8\,\% redundancy) and therefore is
% conservative in probing for available capacity. Our experiments show that by
% using adaptive FEC for probing, the endpoints are able to recover 15-25\,\% of
% the lost packets and $\approx$90\,\% of the time using FEC subsequently
% results in an increase in the media rate. These results are comparable to our
% earlier experiments using a fixed FEC interval throughput the duration of the
% call for error-resilience, where we were able to 20-24\;\% of the lost
% packets. We believe using FEC for congestion control in interactive multimedia
% has not been explored in depth by the community, partly because interactive
% multimedia flows have very tight delay constraints and FEC may not arrive in
% time for recovering the packet.

% In chapter~\ref{chap:mprtp}, we propose using the multiple interfaces of an
% endpoint (e.g., mobile device, tablet, SOHO gateways) for increasing
% throughput and robustness. It corresponds to using congestion cues from
% out-of-band sources (different paths) and signaled in path (in RTCP). We
% design a protocol extension (MPRTP) to RTP that is backwards compatible and
% exploits multipath capabilities. We implement a scheduling algorithm that
% takes application requirements and current path characteristics into
% consideration to send packets over multiple paths. At the receiver, we
% implement a per-path and aggregate de-jitter buffer, which interwork to
% playout packets smoothly even when the path skew is high. Our experiments show
% that the performance of MPRTP is not degraded compared to single path RTP, so
% that it is safe to deploy. It enables load distribution and capacity
% aggregation, which enables features like mobility, offloading, multihoming.

% In chapter~\ref{chap:cc.nw}, we describe two congestion control mechanisms,
% both use out-of-band signaling, but use in-path and out-of-path sources,
% respectively. First, the in-path sources is based on receiving congestion cues
% from middleboxes (e.g., 3G or LTE base-stations) and use standard RTP
% extensions (e.g., TMMBR) for signaling . Second, the out-of-path sources are
% crowd-sourced 3G coverage maps that collect throughput and geo-location
% information from users and then notifies endpoints about the available
% capacity in the region. We propose a system to collect and query coverage
% maps. The endpoints query the coverage server using out-of-band signaling to
% discover areas of poor coverage, based on these notifications the endpoints
% vary their sending rate.



% insights and criticism

% \section{Future Work}

For each of the proposed congestion control techniques in this thesis, we make
deployment recommendations and for some of these techniques the deployment
considerations overlap. The techniques proposed in this thesis are only shown
to work independently. While nothing prevents them to interwork, the
interworking is not discussed in the thesis, therefore, making them interwork
together deserves further investigation. For example, the sending endpoint
receiving congestion cues from a network coverage map service and from the
receiving endpoint can complement the capacity estimated by the sender, hence,
performing better congestion control.

Individually, each technique associated with the particular area of the
framework requires further work. First, emergence of WebRTC requires
standardized congestion control, chapters~\ref{chap:cc} and~\ref{chap:er-cc}
make some proposals that may fit the purpose, but require more extensive
evaluations to be deployment ready. Second, using FEC for congestion control
needs to be generalized to work with any other congestion control algorithms.
Third, multipath scheduling needs to be reconsidered for interactive
multimedia communication. Fourth, the scalability and robustness of the
network coverage map service needs to studied, furthermore, the 
location-throughput matching algorithm needs to be also studied in more 
detail to be able to respond to diverse reporting from various mobile 
devices, etc. Lastly, combine all these techniques for implementing 
congestion control for interactive multimedia.


